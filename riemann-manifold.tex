\documentclass{ltjsbook}
\usepackage[%
    textwidth=40\zw,
    lines=40,
    centering
]{geometry}
\usepackage{graphicx}
\usepackage{subcaption}
\usepackage{booktabs}
\usepackage[%
    hang,
    flushmargin
]{footmisc}
\usepackage{amssymb}
\usepackage{amsmath}
\usepackage{amsthm}
% \usepackage{mtpro2}
\usepackage{bussproofs}
\usepackage[
    math-style=ISO
]{unicode-math}
\usepackage{customdice}
\usepackage{tikz}
\usepackage{tikz-3dplot}
\usepackage{tikz-qtree}
\usepackage{pgfplots}
\usepackage{luamplib}
\usetikzlibrary{%
    patterns,
    intersections,
    calc,
    angles,
    quotes
}

\usepackage{xcolor}
\definecolor{sBlue}{HTML}{0095D9}
\colorlet{sOriginalCurve}{sBlue}

\definecolor{sRed}{HTML}{FF0033}
\colorlet{sApproxCurve}{sRed}

\definecolor{sGreen}{HTML}{138D75}
\colorlet{sSimpleCurve}{sGreen}

\definecolor{sFill}{HTML}{00A381}
\definecolor{sLightGray}{gray}{0.85}

\usepackage{emoji}
\usepackage{framed}

\usepackage[
    % no-math
]{fontspec}
\usepackage[
    deluxe,%
    expert,
    scale=0.95
]{luatexja-preset}

\usepackage{polyglossia}
\setdefaultlanguage{japanese}
\setotherlanguages{english,french,german,russian,hebrew,greek}

\usepackage{covington}
\setglossoptions{%
    fstl=\normalfont\upshape\rmfamily\mcfamily,
    enquotetl=false,
}

\usepackage{luatexja-ruby}
% \setmainfont[%
%     Extension=.otf,
%     UprightFont=*-Book,
%     UprightFont=*-Regular,
%     ItalicFont=*-Italic,
%     BoldFont=*-Bold,
%     BoldItalicFont=*-BoldItalic,
%     SmallCapsFeatures={Numbers=OldStyle},
% ]{NewCM10}
\setmainfont[%
    Extension=.ttf,
    UprightFont=*MT,
    ItalicFont=*-ItalicMT,
    BoldFont=*-BoldMT,
    BoldItalicFont=*-BoldItalicMT,
    Ligatures=Discretionary
]{TimesNewRomanPS}
% \setmainfont{STIX Two Text}
\setsansfont[
    Extension=.ttf,
    UprightFont=*-Regular,
    BoldFont=*-Medium
]{Roboto}
% \setmathfont[
%    math-style=ISO,
%    StylisticSet=1
% ]{STIX Two Math}
\setmathfont{NewCMMath-Book.otf}
\setmathfont[%
    range={%
        cal,
        bfcal
    },
    StylisticSet=1
]{KpMath-Regular.otf}

\ltjnewpreset{book}{%
    mc-m = A-OTF-RyuminPro-Regular.otf,
    % mc-m = NotoSerifJP-Regular.otf,
    % mc-m = HiraMinProN-W3.otf,
    gt-m = BIZUDGothic-Regular.ttf,
    gt-b = BIZUDGothic-Bold.ttf,
    mg-m = MotoyaMaruStd-W5.otf
}
\ltjapplypreset{book}
\setmonojfont{HiraMaruProN-W4}
\ltjsetparameter{%
    jacharrange={%
        -2, % Exclude Greek and Cyrillic letters.
        -3, % Punctuations and miscellaneous symbols.
        -7  % other CJK characters
    },
    alxspmode={`/,allow},
    alxspmode={`#,allow},
    alxspmode={92,allow} % backslash
}

\usepackage{wtref}
\newref{fig}
\setrefstyle{fig}{prefix=図}
\newref{eq}
\setrefstyle{eq}{prefix=式(, suffix=)}
\newref{definition}
\setrefstyle{definition}{prefix=定義}
\newref{axiom}
\setrefstyle{axiom}{prefix=公理}
\newref{proposition}
\setrefstyle{proposition}{prefix=命題}
\newref{lemma}
\setrefstyle{lemma}{prefix=補題}
\newref{example}
\setrefstyle{example}{prefix=例}
\newref{theorem}
\setrefstyle{theorem}{prefix=定理}
\newref{sec}
\setrefstyle{sec}{prefix={第},suffix={節}}
\newref{corollary}
\setrefstyle{corollary}{prefix=系}

\newtheoremstyle{jplain}%
    {0.6\baselineskip}%
    {0.6\baselineskip}%
    {\normalfont}%
    {}%
    {\bfseries\gtfamily\sffamily}%
    {}%
    {5pt}%
    {\thmname{#1}\thmnumber{#2}\thmnote{#3}\ignorespaces}
\theoremstyle{jplain}
\newtheorem{theorem}{定理}[chapter]
\newtheorem{proposition}[theorem]{命題}
\newtheorem{lemma}[theorem]{補題}
\newtheorem{definition}[theorem]{定義}
\newtheorem{exa}[theorem]{例}
\newtheorem{axiom}[theorem]{公理}
\newtheorem{corollary}[theorem]{系}
\newtheorem{numberedquote}[theorem]{引用}
\renewcommand{\proofname}{証明}
\renewcommand{\qedsymbol}{\rule{5pt}{10pt}}

\usepackage{tcolorbox}
\tcbuselibrary{%
    skins,
    breakable
}
\newtcolorbox{thmbox}{%
    colback=sLightGray,
    top=8pt,
    bottom=8pt,
    left=8pt,
    right=8pt,
    boxrule=0pt,
    sharp corners,
    frame hidden
}
\newtcolorbox{quotebox}{%
    breakable,
    colback=sLightGray,
    top=8pt,
    bottom=8pt,
    left=8pt,
    right=8pt,
    boxrule=0pt,
    sharp corners,
    frame hidden
}

\makeatletter
\renewenvironment{proof}[1][\proofname]{\par
  \pushQED{\qed}%
  \normalfont \topsep6\p@\@plus6\p@\relax
  \trivlist
  \item[\hskip\labelsep\bfseries\gtfamily#1]\ignorespaces
}{%
  \popQED\endtrivlist\@endpefalse
}
\makeatother
\AtBeginDocument{%
  \setlength{\abovedisplayskip}{5pt}%
  \setlength{\belowdisplayskip}{5pt}%
}

% https://note.com/yuw/n/n38dd54fb2169
\begingroup
\catcode`\,=\active
\def\@x@{\def,{\normalcomma\hskip.2em}}
\expandafter\endgroup\@x@%
\mathcode`\,="8000
\def\normalcomma{\mathchar"613B}

\usepackage{enumitem}
\newlist{conditions}{enumerate}{1}
\setlist[conditions]{label=(\arabic*)}

\newlist{inproofconditions}{enumerate}{1}
\setlist[inproofconditions]{label=(\alph*)}

\usepackage{hyperref}
\hypersetup{%
    colorlinks,
    citecolor=sBlue,
    linkcolor=sBlue,
    urlcolor=sBlue
}

\usepackage[%
    backend=biber,
    style=pecorarista,
    sorting=nyvt,
    urldate=long
]{biblatex}
\addbibresource{references.bib}

\newcommand\mainchapter[1]{\chapter{#1}\thispagestyle{empty}}
\renewcommand{\jsParagraphMark}{}
\renewcommand{\headfont}{\bfseries\gtfamily\sffamily}

\newcommand\tuple[1]{(#1)}
\newcommand\coord[1]{(#1)}
\newcommand\openinterval[1]{(#1)}
\newcommand\absolute[1]{\lvert #1 \rvert}
\newcommand\keyword[1]{{\bfseries\gtfamily #1}}
\newcommand\emphchar[1]{\ltjalchar`‹\thinspace{#1}\thinspace\ltjalchar`›}

\newcommand\NaturalNumber{\symbb{N}}
\newcommand\Integer{\symbb{Z}}
\newcommand\PositiveInteger{\symbb{N}_{\mathord{+}}}
\newcommand\SequenceLikeOpen{\langle}
\newcommand\SequenceLikeClose{\rangle}
\newcommand\Sequence[2]{{\SequenceLikeOpen {#1}_{#2} \SequenceLikeClose}_{#2 \in \PositiveInteger}}
\newcommand\Subsequence[1]{{\SequenceLikeOpen {#1}_{\varphi(n)} \SequenceLikeClose}_{n \in \PositiveInteger}}
\newcommand\Sequencen[1]{{\SequenceLikeOpen {#1}_n \SequenceLikeClose}_{n \in \PositiveInteger}}
\newcommand\SimpleSequencen[1]{{\SequenceLikeOpen {#1}_n \SequenceLikeClose}_{n}}
\newcommand\Family[3]{{\SequenceLikeOpen {#1}_{#2} \SequenceLikeClose}_{#2 \in #3}}
\newcommand\FamilyLambda[1]{{\SequenceLikeOpen {#1}_\lambda \SequenceLikeClose}_{\lambda \in \Lambda}}
\newcommand\FamilySimple[1]{{\SequenceLikeOpen {#1} \SequenceLikeClose}_\lambda}
\newcommand\FamilySimpleLambda[1]{{\SequenceLikeOpen {#1} \SequenceLikeClose}_\lambda}
\newcommand\NonNegReal{\symbb{R}_{\mathord{+}}}
\newcommand\PositiveReal{\symbb{R}_{\mathord{+}\mathord{+}}}
\newcommand\Generatedtopology[1]{\tau\left[#1\right]}
\newcommand\Neighborhood[1]{\symcal{N}_{#1}}
\newcommand\OpenNeighborhood[1]{\symcal{U}_{#1}}

\newcommand\placeholder{\mathord{\bullet}}
\newcommand\powerset[1]{\wp(#1)}
\newcommand\inlinefrac[2]{#1\mathbin{/}#2}
\newcommand\setcomp[1]{{#1}^{\mathsf{c}}}
% https://zrbabbler.hatenablog.com/entry/20120411/1334151482
\newcommand{\relmiddle}[1]{\mathrel{}\middle#1\mathrel{}}

\DeclareMathOperator{\identity}{id}
\DeclareMathOperator{\trace}{trace}
\DeclareMathOperator{\len}{len}
\newcommand\symdiffsymbol{\mathord{\triangle}}
\newcommand\symdiff[2]{#1\mathbin{\triangle}#2}
\newfontfamily\treefont[ItalicFont={Roboto Light Italic}]{Roboto Light}
\newfontfamily\suetterlin{Suetterlin-HJZ-Italic-1911.ttf}
\newcommand\formallang[1]{{\treefont\itshape #1}}

\newcommand\header[1]{\multicolumn{1}{c}{\bfseries\gtfamily #1}}
\newcommand\pronunciation[4]{#1 \texttt{#2} [#3] #4.}
\newcommand\primarystress{\mbox{}\char"02C8}
\newcommand\projection[1]{\pi_{#1}}

\newcommand\submin[1]{#1_{\text{min}}}

\newcommand\liaison{\hspace*{0.1em}\raisebox{-0.8ex}{\rotatebox{90}{(}}\hspace*{0.1em}}
\newcommand\shortunderscore{\scalebox{0.7}[1]{\_}}
\newcommand\invbreve[1]{#1{\char"032F}}
% \newcommand\symbb[1]{\mathbb{#1}}
% \newcommand\symcal[1]{\mathcal{#1}}
% \newcommand\symbf[1]{\mathbf{#1}}

\newcommand\iseventuallyin[2]{#1 \mathrel{\text{is eventually in}} #2}
\newcommand\justin[1]{\mathrel{\text{in}} #1}
\newcommand\isfrequentlyin[2]{#1 \mathrel{\text{is frequently in}} #2}
\newcommand\oxfordcomma[3]{#1, #2, \text{and }#3}

\newcommand\refimpliesref[2]{\(\text{#1} \Rightarrow \text{#2}\)}

\renewcommand{\labelitemii}{\(\circ\)}
\renewcommand{\labelitemiii}{\(\diamond\)}

\begin{document}
\mainchapter{多様体}

閉区間\(I := [a, b] \subseteq {\Real}\)から\({\Real}^n\)(一般には位相空間)への写像を\keyword{曲線}(curve)という.
像\(\gamma(I)\)を指して曲線ということもある.
\(\gamma(a)\)を\(\gamma\)の始点,\(\gamma(b)\)を\(\gamma\)の終点という.
\(p\)を始点,\(q\)を終点とする曲線を\(p\)から\(q\)への曲線という.
\(n\)を任意の自然数とする.
\(\gamma\)の\(m\)階までの導関数\(D^m \gamma\)が存在し,\(D^m \gamma\)が連続であるとき,すなわち\(C^m\)級であるとき,\(\gamma\)は\(C^m\)級曲線であるという.
\(\gamma\)が必要なだけ微分できる,すなわち\(C^\infty\)級であるとき,\(\gamma\)は\keyword{滑らか}(smooth)であるという.
\(C^1\)級曲線\(\gamma\colon I \to \symbb{R}^n\)が\(t \in I\)において\(D\gamma(t) \neq 0\)をみたすとき,\(\gamma\)は\keyword{正則}(regular)であるという.

\begin{example} \(r\)と\(m\)を正の定数とする.
次で定められる曲線\(\gamma \colon I \to {\Real}^n\)を常螺旋(ordinary helix)という.
\begin{align*}
    \gamma(t) = \begin{pmatrix}
        r \cos t \\
        r \sin t \\
        m t
    \end{pmatrix}.
\end{align*}
これは
\begin{align*}
    D\gamma(t) = \begin{pmatrix}
        - r \sin t \\
        r \cos t \\
        m
    \end{pmatrix} \neq 0
\end{align*}
をみたすので正則な曲線である.
\end{example}

\(\gamma_1 \colon I \to {\Real}^n\)と\(\gamma_2 \colon J \to {\Real}^n\)を\(C^1\)級曲線とする.
\(C^1\)級狭義単調増加関数\(\psi \colon I \to J\)で\(\gamma_2 = \gamma_1 \circ \psi\)となるようなものが存在することを\(\gamma_1 \sim \gamma_2\)と表すと,関係\(\sim\)は同値関係である.
これは逆関数定理より\(\psi^{-1}\)
\(\psi\)は曲線(の像)\(\tilde{\gamma} := \gamma_1(I) = \gamma_2(J)\)の同値なパラメーターであるという.

\begin{example} \(\gamma_1 \colon [0, \pi] \to {\Real}^2\)と\(\gamma_2 \colon [0, \inlinefrac{\pi}{2}] \to {\Real}^2\)をそれぞれ
\begin{align*}
    \gamma_1(\theta) = \begin{pmatrix}
        \cos \theta \\
        \sin \theta
    \end{pmatrix},
    \quad
    \gamma_2(\theta) = \begin{pmatrix}
        \cos 2 \theta \\
        \sin 2 \theta
    \end{pmatrix}
\end{align*}
で定めると\(\psi \colon \theta \mapsto 2 \theta\)によって\(\gamma_1 \circ \psi = \gamma_2\)となるので,これらは同値な曲線である.
\end{example}

\begin{thmbox}
\begin{proposition}
\(k_1, k_2\)を定数とする.
\(C^1\)級曲線\(\gamma \colon [a, b] \to {\Real}^n\)が\(\lVert \dot{\gamma}(t) \rVert = k_1\)をみたすならば以下が成り立つ.
\begin{align}
    \left(\int_a^b \lVert \dot{\gamma}(t) \rVert dt \right)^2
    = k_2 \int_a^b \lVert \dot{\gamma}(t) \rVert^2 dt.
    \mathlabel{l2-cauchy-schwarz-for-length}
\end{align}
\end{proposition}
\end{thmbox}

\begin{proof} \(\dot{\gamma}\)とノルムの連続性により\(t \mapsto \lVert \dot{\gamma}(t) \rVert, t \mapsto \lVert \dot{\gamma}(t) \rVert^2\)もまた連続であり,したがって可積分である.
\(f\colon t \mapsto \lVert \dot{\gamma}(t) \rVert\)と定数関数\(g \colon t \to 1\)について\(L^2([a, b])\)におけるCauchy--Schwarzの不等式から
\begin{gather}
    \left\lvert \int_a^b f(t) g(t) dt \right\rvert \leq \sqrt{\int_a^b \lvert f(t) \rvert^2 dt} \sqrt{\int_a^b \lvert g(t) \rvert^2 dt} \notag \\
    \left\lvert \int_a^b \lVert \dot{\gamma}(t) \rVert dt \right\rvert \leq \sqrt{b - a} \sqrt{\int_a^b \lVert \dot{\gamma}(t) \rVert^2 dt} \mathlabel{l2-cauchy-schwarz-for-length-intermediate}.
\end{gather}
\(f = k_1 = k_1 g\)であるから,\mathref{l2-cauchy-schwarz-for-length-intermediate}において等号が成立する.
\mathref{l2-cauchy-schwarz-for-length-intermediate}の両辺を\(2\)乗し,\(k_2 = b - a\)とすれば\mathref{l2-cauchy-schwarz-for-length}が得られる.
\end{proof}

\begin{thmbox}
\begin{proposition}
曲線\(\gamma\colon [a, b] \to {\Real}^n\)が
パラメーター\(\gamma\)
\(\psi\colon t \to s\)
\begin{align*}
    \int_a^b \lVert \dot{\gamma}(t) \rVert dt
    = \int_a^b \lVert \gamma'(s) \rVert ds
\end{align*}
\end{proposition}
\end{thmbox}

\begin{proof}
\(t = \tau(s)\)
\begin{align*}
    \int_{s_0}^{s_1} \left\lVert \frac{d(\gamma \circ \tau)}{ds}\right\rVert ds
    &= \int_{s_0}^{s_1} \left\lVert \frac{d\gamma}{dt}(\tau(s)) \frac{d\tau}{ds}(s) \right\rVert ds \\
    &= \int_{t_0}^{t_1} \left\lVert \frac{d\gamma}{dt}(t) \frac{1}{\dfrac{ds}{dt}(t)} \right\rVert \frac{ds}{dt}(t) dt \\
    &= \int_{t_0}^{t_1} \left\lVert \frac{d\gamma}{dt}(t) \right\rVert dt
\end{align*}
\end{proof}

\begin{thmbox}
\begin{proposition}
\(\gamma \colon [a, b] \to {\Real}^n\)を\(C^1\)級曲線とする.
\(s\colon [a, b] \to {\Real}\)を
\begin{align*}
    s(t) = \int_{t_0}^t \left\lVert \dot{\gamma}(\xi) \right\rVert d\xi
\end{align*}
で定義する.このとき\(\lVert \gamma' (s) \rVert = 1\)が成り立つ.
\end{proposition}
\end{thmbox}

\begin{proof}
\begin{align*}
    \lVert \gamma'(s) \rVert = \left\lVert \frac{d(\gamma \circ s)}{ds} \right\rVert
\end{align*}
\end{proof}

\(2\)点を結ぶ球面上の曲線の中で長さが最小となるのはそれらによって定まる大円の劣弧となることを示す.
計算を簡単にするため\(R = 1\)とする.
地球\(S^2 = S^2(1)\)上の点を球面座標\(\tuple{r, \theta, \varphi}\)で表すことを考える.
動径\(r\)は\(1\)に固定する.
天頂角\(\theta\) \((0 \leq \theta \leq \pi)\)は緯度を用いると
\begin{align*}
    \theta = \begin{cases}
        \displaystyle
        \frac{\pi}{2} - \frac{\alpha^\circ}{180^\circ} \pi
        & \text{北緯\(\alpha^\circ\)のとき,} \\[10pt]
        \displaystyle
        \frac{\pi}{2} + \frac{\alpha^\circ}{180^\circ} \pi
        & \text{南緯\(\alpha^\circ\)のとき.}
    \end{cases}
\end{align*}
これを測地学などの文脈では余緯度(colatitude)と呼ぶ.
方位角\(\varphi\) \((0 \leq \varphi < 2\pi)\)は緯度を用いると以下のように計算できる.
\begin{align*}
    \varphi = \begin{cases}
        \displaystyle
        \frac{\beta^\circ}{180^\circ} \pi
        & \text{東経\(\beta^\circ\)のとき,} \\[10pt]
        \displaystyle
        \pi - \frac{\beta^\circ}{180^\circ} \pi
        & \text{西経\(\beta^\circ\)のとき.}
    \end{cases}
\end{align*}
北極点\(\symup{N} = (0, 0, 1)\)と南極点\(\symup{S} = (0, 0, -1)\)は表し方が一意でないことに注意する.
極を含む本初子午線は
\begin{align*}
    M = \left\{\begin{pmatrix} \sin \theta \\ 0 \\ \cos \theta \end{pmatrix} \in {\Real}^3 \relmiddle{|} 0 \leq \theta \leq \pi\right\}
\end{align*}
と表される.
相違なる\(2\)点\(p, q \in S^2 \setminus M\)をとる.
\(p\)から\(q\)への正則\footnote{%
曲線\(\gamma\colon [a, b] \to {\Real}^n\)が正則(regular)であるとは\(t \in [a, b]\)において\(\dot{\gamma}(t) \neq 0\)であることをいう.
}で滑らかな\footnote{%
曲線\(\gamma\colon [a, b] \to {\Real}^n\)が滑らか(smooth)であるとは,\(\gamma\)が\([a, b]\)において\(C^\infty\)級であることをいう.
}曲線\(\gamma \colon [0, 1] \to S^2\)を,
\(\theta \colon [0, 1] \to (0, \pi)\)と\(\varphi\colon [0, 1] \to (0, 2\pi)\)を用いて
\begin{align*}
    \gamma \colon t \mapsto
        \begin{pmatrix}
            \sin \theta(t) \cos \varphi(t) \\
            \sin \theta(t) \sin \varphi(t) \\
            \cos \theta(t)
        \end{pmatrix}
\end{align*}
と表す.
\(p, q\)を最短で結ぶ曲線を求めるので\(\gamma\)は自身と交わらない,すなわち単射であると仮定する.
このとき,\(\gamma\)の長さ\(L(\gamma)\)は
\begin{align}
    L(\gamma) = \int_{0}^{1} \lVert \dot{\gamma}(t) \rVert  dt
    \mathlabel{great-circle-length-integral}
\end{align}
で求められる.\(\gamma\)の\(\theta, \varphi\)に関する偏導関数を
\(\partial_\theta, \partial_\varphi\)と略記することにする.
\begin{align*}
    \partial_\theta
    = \frac{\partial \gamma}{\partial \theta}(\theta, \varphi)
    =
    \begin{pmatrix}
        \cos \theta \cos \varphi \\
        \cos \theta \sin \varphi \\
        - \sin \theta
    \end{pmatrix},
    \quad
    \partial_\varphi
    = \frac{\partial \gamma}{\partial \varphi}(\theta, \varphi)
    =
    \begin{pmatrix}
        - \sin \theta \sin \varphi \\
        \sin \theta \cos \varphi \\
        0
    \end{pmatrix}.
\end{align*}
これらの内積をあらかじめ計算しておくと
\begin{align*}
    \partial_\theta \cdot \partial_\theta
    &=
    \begin{pmatrix}
        \cos \theta \cos \varphi \\
        \cos \theta \sin \varphi \\
        - \sin \theta
    \end{pmatrix}
    \cdot
    \begin{pmatrix}
        \cos \theta \cos \varphi \\
        \cos \theta \sin \varphi \\
        - \sin \theta
    \end{pmatrix} \\
    &= \cos^2 \theta \cos^2 \varphi
        + \cos^2 \theta \sin^2 \varphi
        + \sin^2 \theta \\
    &= 1,
\end{align*}
\begin{align*}
    \partial_\varphi \cdot \partial_\varphi
    &=
    \begin{pmatrix}
        - \sin \theta \sin \varphi \\
        \sin \theta \cos \varphi \\
        0
    \end{pmatrix}
    \cdot
    \begin{pmatrix}
        - \sin \theta \sin \varphi \\
        \sin \theta \cos \varphi \\
        0
    \end{pmatrix} \\
    &= \sin^2 \theta \sin^2 \varphi + \sin^2 \theta \cos^2 \varphi \\
    &= \sin^2 \theta,
\end{align*}
\begin{align*}
    \partial_\theta \cdot \partial_\varphi
    &=
    \begin{pmatrix}
        \cos \theta \cos \varphi \\
        \cos \theta \sin \varphi \\
        - \sin \theta
    \end{pmatrix}
    \cdot
    \begin{pmatrix}
        - \sin \theta \sin \varphi \\
        \sin \theta \cos \varphi \\
        0
    \end{pmatrix} \\
    &= - \cos \theta \cos \varphi \sin \theta \sin \varphi
       + \cos \theta \sin \varphi \sin \theta \cos \varphi \\
    &= 0.
\end{align*}
これらの結果を用いると被積分関数は以下のように計算できる.
\begin{align}
    \lVert \dot{\gamma}(t) \rVert^2
    &= \dot{\gamma}(t) \cdot \dot{\gamma}(t) \notag \\
    &= \begin{pmatrix}
            \dot{\theta} & \dot{\varphi}
        \end{pmatrix}
        \begin{pmatrix}
            \partial_\theta \\ \partial_\varphi
        \end{pmatrix}
        \begin{pmatrix}
            \partial_\theta & \partial_\varphi
        \end{pmatrix}
        \begin{pmatrix}
            \dot{\theta} \\ \dot{\varphi}
        \end{pmatrix} \notag \\
    &= \begin{pmatrix}
            \dot{\theta} & \dot{\varphi}
        \end{pmatrix}
        \begin{pmatrix}
            1 & 0 \\
            0 & \sin^2 \theta \\
        \end{pmatrix}
        \begin{pmatrix}
            \dot{\theta} \\ \dot{\varphi}
        \end{pmatrix} \notag \\
    &=  \dot{\theta}^2 + \dot{\varphi}^2 \sin^2 \theta \mathlabel{great-circle-dot-gamma}
\end{align}
正則性の仮定により\(\lVert \dot{\gamma}(t) \rVert^2 > 0\)である.
関数\(s \colon [0, 1] \to [0, L(\gamma)]\)を
\begin{align*}
    s \colon t \mapsto \int_0^t \lVert \dot{\gamma}(\tau) \rVert d\tau
\end{align*}
と定義する.逆関数定理より,\(t\)を\(s\)を用いて\(t(s)\)のように表すことができて,これは
\begin{align*}
    \frac{dt}{ds}(s)
    = \frac{1}{\dfrac{ds}{dt}(t(s))}
    = \frac{1}{\lVert \dot{\gamma}(t) \rVert}
\end{align*}
をみたす.
このようなパラメーターの置き換えを弧長によるパラメーター付け(arc length parametrization)という.
\begin{align}
    \lVert \dot{\gamma}(t(s)) \rVert
    = \left\lVert \frac{d\gamma}{dt}(t(s)) \frac{1}{\dfrac{ds}{dt}(t)} \right\rVert
    = \left\lvert \frac{1}{\lVert \dot{\gamma}(t) \rVert} \right\rvert \lVert \dot{\gamma}(t) \rVert
    = 1
\end{align}
この問題のEuler--Lagrange方程式は
\begin{align*}
    \frac{d}{ds}
    \begin{pmatrix}
        \displaystyle
        \frac{\partial F}{\partial \theta'}
        &
        \displaystyle
        \frac{\partial F}{\partial \varphi'}
    \end{pmatrix}
    -
    \begin{pmatrix}
        \displaystyle
        \frac{\partial F}{\partial \theta}
        &
        \displaystyle
        \frac{\partial F}{\partial \varphi}
    \end{pmatrix}
    = 0
\end{align*}
である.成分ごとに計算を進めて
\begin{numcases}
    {}
    \theta'' = (\varphi')^2 \sin \theta \cos \theta \mathlabel{great-circle-ode1} \\
    \frac{d}{ds}(\varphi' \sin^2 \theta) = 0 \mathlabel{great-circle-ode2}
\end{numcases}
を得る.\mathref{great-circle-ode2}より\(C\)を定数として
\begin{align*}
    \varphi' \sin^2 \theta = C
\end{align*}
が成り立つ.これを\mathref{great-circle-ode1}に代入して
\begin{align*}
    \theta'' = C^2 \frac{\cos \theta}{\sin^3 \theta}
\end{align*}


\mathref{great-circle-ode2}から
\begin{align*}
    \frac{\dot{\varphi}(t) \sin^2\theta(t)}{\sqrt{(\dot{\theta}(t))^2 + (\dot{\varphi}(t))^2 \sin^2 \theta(t)}}
    =
    \frac{\dot{\varphi}(0) \sin^2\theta(0)}{\sqrt{(\dot{\theta}(0))^2 + (\dot{\varphi}(0))^2 \sin^2 \theta(0)}}
\end{align*}
が成り立つ.\(\theta(0) = 0\)から\(0 < t < 1\)において\(\dot{\varphi}(t) = 0\)である.
\(q\)が球面座標で\(\tuple{1, \theta_1, \varphi_1}\)と表されているとき,定数関数\(\varphi\colon t \mapsto \varphi_1\)をとると,\(\dot{\varphi} = 0\)と\(\gamma(1) = q\)をみたす.
\(\theta\)としては例えば\(\theta\colon t \mapsto t \theta_1\)をとればよい.
以上の結果から\(\gamma\)が\(p\)と\(q\)を通る大円の劣弧であることがわかる.

このように\(\symbb{R}^3\)の図形(多様体)の上で\(2\)点を結ぶ最短の曲線を求める問題は,Riemann多様体\(\tuple{M, g}\)上での問題に一般化できる.集合\(\Omega(p, q)\)を
\begin{align*}
    \Omega(p, q) = \{%
        \gamma\colon [a, b] \to M \mid
        \text{\(\gamma\)は区分的に滑らかな曲線で\(\gamma(a) = p\)かつ\(\gamma(b) = q\)}
    \}
\end{align*}
とする.ただし\(\gamma\to [a, b] \to M\)が区分的に滑らかであるとは分割\(a = t_0 < t_1 < \cdots < t_n = b\)が存在して,各区間\([t_{i - 1}, t_i]\)への制限\(\gamma\restriction_{[t_{i - 1}, t_{i}]}\colon [t_{i - 1}, t_i] \to M\)が\(C^\infty\)級であることをいう.
\begin{align*}
    L(\gamma) = \int_{0}^{1} \sqrt{g\left(\frac{d\gamma}{dt}(t), \frac{d\gamma}{dt}(t)\right)} dt
\end{align*}

\begin{align}
    d_g(p, q) = \inf \{L(\gamma) \mid \gamma \in \Omega(p, q)\}
\end{align}
\keyword{Riemann距離}(Riemannian distance)という.
Riemann metricといってしまうと,それはRiemann計量を指すため注意が必要である.


\mathref{great-circle-distance}が距離の公理をみたすことを示す.
以下\(p, q, r\)は\(S^2(R) \subseteq \symbb{R}^3\)の点とする.
最初に非負性を示す.任意の\(x \in [-1 , 1]\)について
\begin{align*}
    0 \leq \arccos x \leq \pi
\end{align*}
から\(d(p, q) \geq 0\)である.
\(d(p, q) = 0\)となるのは\(p \cdot q = \lVert p \rVert \lVert q \rVert\)が成立するときかつそのときに限る.
これはCauchy--Schwarzの不等式において等号が成立する条件と同値であり,ある\(\lambda > 0\)が存在して\(p = \lambda q\)が成り立つということを意味する.
両辺のノルムをとると\(\lambda = 1\)がわかる.
したがって\(p = q\)が\(d(p, q) = 0\)の必要充分条件である.
対称性は\(\symbb{R}^3\)の内積が可換であることから明らかである.
最後に三角不等式を示す.
大円距離はRiemann距離の一例であるからRiemann距離について示せばよい.
任意の\(p, q, r\in M, \gamma_1 \in \Omega(p, r), \gamma_2 \in \Omega(r, q)\)をとる.
下限の定義から,任意の\(\varepsilon > 0\)について\(\gamma_1 \in \Omega(p, r), \gamma_2 \in \Omega(r, q)\)が存在して
\begin{gather*}
    L(\gamma_1) < d_g(p, r) + \frac{\varepsilon}{2},
    \quad
    L(\gamma_2) < d_g(r, q) + \frac{\varepsilon}{2}
\end{gather*}
が成り立つ.したがって
\begin{align*}
    d_g(p, q) \leq L(\gamma_1 + \gamma_2)
            = L(\gamma_1) + L(\gamma_2)
            < d_g(p, r) + d_g(r, q) + \varepsilon.
\end{align*}
\(\varepsilon > 0\)が任意であったから三角不等式\(d_g(p, q) \leq d_g(p, r) + d_g(r, q)\)が成り立つ.

\end{document}
