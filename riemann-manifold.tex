\documentclass{ltjsbook}
\usepackage[%
    textwidth=40\zw,
    lines=40,
    centering
]{geometry}
\usepackage{graphicx}
\usepackage{booktabs}
\usepackage[%
    hang,
    flushmargin
]{footmisc}
\usepackage{amssymb}
\usepackage{amsmath}
\usepackage{amsthm}
% \usepackage{mtpro2}
\usepackage{bussproofs}
\usepackage[
    math-style=ISO
]{unicode-math}
\usepackage{customdice}
\usepackage{tikz}
\usepackage{tikz-qtree}
\usepackage{pgfplots}
\usetikzlibrary{%
    patterns,
    intersections,
    calc,
    angles,
    quotes
}

\usepackage{xcolor}
\definecolor{sBlue}{HTML}{0095D9}
\colorlet{sOriginalCurve}{sBlue}

\definecolor{sRed}{HTML}{FF0033}
\colorlet{sApproxCurve}{sRed}

\definecolor{sGreen}{HTML}{138D75}
\colorlet{sSimpleCurve}{sGreen}

\definecolor{sFill}{HTML}{00A381}
\definecolor{sLightGray}{gray}{0.85}

\usepackage{emoji}
\usepackage{framed}

\usepackage[
    % no-math
]{fontspec}
\usepackage[
    deluxe,%
    expert,
    scale=0.95
]{luatexja-preset}

\usepackage{polyglossia}
\setdefaultlanguage{japanese}
\setotherlanguages{english,french,german,russian,hebrew,greek}

\usepackage{covington}
\setglossoptions{%
    fstl=\normalfont\upshape\rmfamily\mcfamily,
    enquotetl=false,
}

\usepackage{luatexja-ruby}
% \setmainfont[%
%     Extension=.otf,
%     UprightFont=*-Book,
%     UprightFont=*-Regular,
%     ItalicFont=*-Italic,
%     BoldFont=*-Bold,
%     BoldItalicFont=*-BoldItalic,
%     SmallCapsFeatures={Numbers=OldStyle},
% ]{NewCM10}
\setmainfont[%
    Extension=.ttf,
    UprightFont=*MT,
    ItalicFont=*-ItalicMT,
    BoldFont=*-BoldMT,
    BoldItalicFont=*-BoldItalicMT,
    Ligatures=Discretionary
]{TimesNewRomanPS}
% \setmainfont{STIX Two Text}
\setsansfont[
    Extension=.ttf,
    UprightFont=*-Regular,
    BoldFont=*-Medium
]{Roboto}
% \setmathfont[
%    math-style=ISO,
%    StylisticSet=1
% ]{STIX Two Math}
\setmathfont{NewCMMath-Book.otf}
\setmathfont[%
    range={%
        cal,
        bfcal
    },
    StylisticSet=1
]{KpMath-Regular.otf}

\ltjnewpreset{book}{%
    mc-m = A-OTF-RyuminPro-Regular.otf,
    % mc-m = NotoSerifJP-Regular.otf,
    % mc-m = HiraMinProN-W3.otf,
    gt-m = BIZUDGothic-Regular.ttf,
    gt-b = BIZUDGothic-Bold.ttf,
    mg-m = MotoyaMaruStd-W5.otf
}
\ltjapplypreset{book}
\setmonojfont{HiraMaruProN-W4}
\ltjsetparameter{%
    jacharrange={%
        -2, % Exclude Greek and Cyrillic letters.
        -3, % Punctuations and miscellaneous symbols.
        -7  % other CJK characters
    },
    alxspmode={`/,allow},
    alxspmode={`#,allow},
    alxspmode={92,allow} % backslash
}

\usepackage{wtref}
\newref{fig}
\setrefstyle{fig}{prefix=図}
\newref{eq}
\setrefstyle{eq}{prefix=式(, suffix=)}
\newref{definition}
\setrefstyle{definition}{prefix=定義}
\newref{axiom}
\setrefstyle{axiom}{prefix=公理}
\newref{proposition}
\setrefstyle{proposition}{prefix=命題}
\newref{lemma}
\setrefstyle{lemma}{prefix=補題}
\newref{example}
\setrefstyle{example}{prefix=例}
\newref{theorem}
\setrefstyle{theorem}{prefix=定理}
\newref{sec}
\setrefstyle{sec}{prefix={第},suffix={節}}

\newtheoremstyle{jplain}%
    {0.6\baselineskip}%
    {0.6\baselineskip}%
    {\normalfont}%
    {}%
    {\bfseries\gtfamily\sffamily}%
    {}%
    {5pt}%
    {\thmname{#1}\thmnumber{#2}\thmnote{#3}\ignorespaces}
\theoremstyle{jplain}
\newtheorem{theorem}{定理}[chapter]
\newtheorem{proposition}[theorem]{命題}
\newtheorem{lemma}[theorem]{補題}
\newtheorem{definition}[theorem]{定義}
\newtheorem{exa}[theorem]{例}
\newtheorem{axiom}[theorem]{公理}
\newtheorem{numberedquote}[theorem]{引用}
\renewcommand{\proofname}{証明}
\renewcommand{\qedsymbol}{\rule{5pt}{10pt}}

\usepackage{tcolorbox}
\tcbuselibrary{%
    skins,
    breakable
}
\newtcolorbox{thmbox}{%
    colback=sLightGray,
    top=8pt,
    bottom=8pt,
    left=8pt,
    right=8pt,
    boxrule=0pt,
    sharp corners,
    frame hidden
}
\newtcolorbox{quotebox}{%
    breakable,
    colback=sLightGray,
    top=8pt,
    bottom=8pt,
    left=8pt,
    right=8pt,
    boxrule=0pt,
    sharp corners,
    frame hidden
}

\makeatletter
\renewenvironment{proof}[1][\proofname]{\par
  \pushQED{\qed}%
  \normalfont \topsep6\p@\@plus6\p@\relax
  \trivlist
  \item[\hskip\labelsep\bfseries\gtfamily#1]\ignorespaces
}{%
  \popQED\endtrivlist\@endpefalse
}
\makeatother
\AtBeginDocument{%
  \setlength{\abovedisplayskip}{5pt}%
  \setlength{\belowdisplayskip}{5pt}%
}

% https://note.com/yuw/n/n38dd54fb2169
\begingroup
\catcode`\,=\active
\def\@x@{\def,{\normalcomma\hskip.2em}}
\expandafter\endgroup\@x@%
\mathcode`\,="8000
\def\normalcomma{\mathchar"613B}

\usepackage{enumitem}
\newlist{conditions}{enumerate}{1}
\setlist[conditions]{label=(\arabic*)}

\newlist{inproofconditions}{enumerate}{1}
\setlist[inproofconditions]{label=(\alph*)}

\usepackage{hyperref}
\hypersetup{%
    colorlinks,
    citecolor=sBlue,
    linkcolor=sBlue,
    urlcolor=sBlue
}

\usepackage[%
    backend=biber,
    style=pecorarista,
    sorting=nyvt,
    urldate=long
]{biblatex}
\addbibresource{references.bib}

\newcommand\mainchapter[1]{\chapter{#1}\thispagestyle{empty}}
\renewcommand{\jsParagraphMark}{}
\renewcommand{\headfont}{\bfseries\gtfamily\sffamily}

\newcommand\tuple[1]{(#1)}
\newcommand\keyword[1]{{\bfseries\gtfamily #1}}
\newcommand\emphchar[1]{\ltjalchar`‹#1\ltjalchar`›}

\newcommand\NaturalNumber{\symbb{N}}
\newcommand\Integer{\symbb{Z}}
\newcommand\PositiveInteger{\symbb{N}_{\mathord{+}}}
\newcommand\Sequence[2]{{\langle {#1}_{#2} \rangle}_{#2 \in \PositiveInteger}}
\newcommand\Subsequence[1]{{\langle {#1}_{\varphi(n)} \rangle}_{n \in \PositiveInteger}}
\newcommand\Sequencen[1]{{\langle {#1}_n \rangle}_{n \in \PositiveInteger}}
\newcommand\SimpleSequencen[1]{{\langle {#1}_n \rangle}_{n}}
\newcommand\Family[1]{{({#1}_\lambda)}_{\lambda \in \Lambda}}
\newcommand\NonNegReal{\symbb{R}_{\mathord{+}}}
\newcommand\PositiveReal{\symbb{R}_{\mathord{+}\mathord{+}}}
\newcommand\Generatedtopology[1]{\tau\left[#1\right]}
\newcommand\Neighborhood[1]{\symcal{N}(#1)}
\newcommand\OpenNeighborhood[1]{\symcal{U}(#1)}

\newcommand\placeholder{\mathord{\cdot}}
\newcommand\powerset[1]{\wp(#1)}
\newcommand\inlinefrac[2]{#1\mathbin{/}#2}
\newcommand\setcomp[1]{{#1}^{\mathsf{c}}}
% https://zrbabbler.hatenablog.com/entry/20120411/1334151482
\newcommand{\relmiddle}[1]{\mathrel{}\middle#1\mathrel{}}

\DeclareMathOperator{\identity}{id}
\DeclareMathOperator{\trace}{trace}
\DeclareMathOperator{\len}{len}
\newcommand\symdiffsymbol{\mathord{\triangle}}
\newcommand\symdiff[2]{#1\mathbin{\triangle}#2}
\newfontfamily\treefont[ItalicFont={Roboto Light Italic}]{Roboto Light}
\newfontfamily\suetterlin{Suetterlin-HJZ-Italic-1911.ttf}
\newcommand\formallang[1]{{\treefont\itshape #1}}

\newcommand\header[1]{\multicolumn{1}{c}{\bfseries\gtfamily #1}}
\newcommand\pronunciation[4]{#1 \texttt{#2} [#3] #4.}
\newcommand\primarystress{\mbox{}\char"02C8}

\newcommand\submin[1]{#1_{\text{min}}}

\newcommand\liaison{\hspace*{0.1em}\raisebox{-0.8ex}{\rotatebox{90}{(}}\hspace*{0.1em}}
\newcommand\shortunderscore{\scalebox{0.7}[1]{\_}}
\newcommand\invbreve[1]{#1{\char"032F}}
% \newcommand\symbb[1]{\mathbb{#1}}
% \newcommand\symcal[1]{\mathcal{#1}}
% \newcommand\symbf[1]{\mathbf{#1}}

\newcommand\iseventuallyin[2]{#1 \mathrel{\text{is eventually in}} #2}
\newcommand\justin[1]{\mathrel{\text{in}} #1}
\newcommand\isfrequentlyin[2]{#1 \mathrel{\text{is frequently in}} #2}
\newcommand\oxfordcomma[3]{#1, #2, \text{and }#3}

\newcommand\proofimplies[2]{\(\text{\ref{#1}} \Rightarrow \text{\ref{#2}}\):}

\renewcommand{\labelitemii}{\(\circ\)}
\renewcommand{\labelitemiii}{\(\diamond\)}

\begin{document}
\mainchapter{多様体}

\(2\)点を結ぶ球面上の曲線の中で長さが最小となるのはそれらによって定まる大円の劣弧となることを示す.
計算を簡単にするため\(R = 1\)とする.
地球\(S^2 = S^2(1)\)上の点を球面座標\(\tuple{r, \theta, \varphi}\)で表すことを考える.
動径\(r\)は\(1\)に固定する.
天頂角\(\theta\) \((0 \leq \theta \leq \pi)\)は緯度を用いると
\begin{align*}
    \theta = \begin{cases}
        \displaystyle
        \frac{\pi}{2} - \frac{\alpha^\circ}{180^\circ} \pi
        & \text{北緯\(\alpha^\circ\)のとき,} \\[10pt]
        \displaystyle
        \frac{\pi}{2} + \frac{\alpha^\circ}{180^\circ} \pi
        & \text{南緯\(\alpha^\circ\)のとき.}
    \end{cases}
\end{align*}
これを測地学などの文脈では余緯度(colatitude)と呼ぶ.
方位角\(\varphi\) \((0 \leq \varphi < 2\pi)\)は緯度を用いると以下のように計算できる.
\begin{align*}
    \varphi = \begin{cases}
        \displaystyle
        \frac{\beta^\circ}{180^\circ} \pi
        & \text{東経\(\beta^\circ\)のとき,} \\[10pt]
        \displaystyle
        \pi - \frac{\beta^\circ}{180^\circ} \pi
        & \text{西経\(\beta^\circ\)のとき.}
    \end{cases}
\end{align*}
北極点\(\symup{N} = (0, 0, 1)\)と南極点\(\symup{S} = (0, 0, -1)\)は表し方が一意でないことに注意する.

大円の劣弧が最短距離を与えることを示すにあたり,始点\(p\)を\(\symup{N}\)としても一般性を失わない.
例えば\(p\)の余緯度が\(\theta\),経度が\(\varphi\)であるとする.
\(\symbb{R}^3\)の点を\(z\)軸まわりに\(-\varphi\)だけ回転させる行列\(R_z(-\varphi)\)と,\(y\)軸まわりに\(-\theta\)だけ回転させる行列\(R_y(-\theta)\)の合成\(A := R_y(-\theta) R_z(-\varphi)\)は長さを保つ.
変換後の\(\symup{N} = Ap\)と\(Aq\)を最短で結ぶ曲線上の点を求め,それに逆の変換\(A^{-1} = R_z(\varphi)R_y(\theta)\)を作用させて元の\(p, q\)を最短で結ぶ曲線が得られる.
\(x = \symup{N}\)から\(p\)とは異なる点\(q\)への曲線\(\gamma \colon [0, 1] \to S^2\)をとる.
これは連続微分可能な関数\(\theta \colon [0, 1] \to [0, \pi], \varphi \colon [0, 1] \to \symbb{R}\)を用いて
\begin{align*}
    \gamma \colon t \mapsto
        \begin{pmatrix}
            \sin \theta(t) \cos \varphi(t) \\
            \sin \theta(t) \sin \varphi(t) \\
            \cos \theta(t)
        \end{pmatrix}
\end{align*}
と表されているとする.
また任意の\(t \in [0, 1]\)で
\begin{align}
    \begin{pmatrix}
        \dot{\theta}(t) \\
        \dot{\varphi}(t)
    \end{pmatrix}
    \neq 0
    \mathlabel{great-circle-regular}
\end{align}
であると仮定する.これは直感的にいうと「停まらない」ということである.
また最短で結ぶ曲線を求めるので\(\gamma\)は自身と交わらない,すなわち単射であると仮定してよい.
曲線の長さ\(L(\gamma)\)は
\begin{align}
    L(\gamma) = \int_{0}^{1} \sqrt{\frac{d\gamma}{dt}(t) \cdot \frac{d\gamma}{dt}(t)} dt
    \mathlabel{great-circle-length-integral}
\end{align}
で求められる.
\begin{align*}
    \frac{\partial \gamma}{\partial \theta}(\theta, \varphi)
    =
    \begin{pmatrix}
        \cos \theta \cos \varphi \\
        \cos \theta \sin \varphi \\
        - \sin \theta
    \end{pmatrix},
    \quad
    \frac{\partial \gamma}{\partial \varphi}(\theta, \varphi)
    =
    \begin{pmatrix}
        - \sin \theta \sin \varphi \\
        \sin \theta \cos \varphi \\
        0
    \end{pmatrix}
\end{align*}
であるから,
\begin{align*}
    \frac{\partial \gamma}{\partial \theta}(\theta, \varphi)
    \cdot
    \frac{\partial \gamma}{\partial \theta}(\theta, \varphi)
    &=
    \begin{pmatrix}
        \cos \theta \cos \varphi \\
        \cos \theta \sin \varphi \\
        - \sin \theta
    \end{pmatrix}
    \cdot
    \begin{pmatrix}
        \cos \theta \cos \varphi \\
        \cos \theta \sin \varphi \\
        - \sin \theta
    \end{pmatrix} \\
    &= \cos^2 \theta \cos^2 \varphi
        + \cos^2 \theta \sin^2 \varphi
        + \sin^2 \theta \\
    &= 1,
\end{align*}
\begin{align*}
    \frac{\partial \gamma}{\partial \varphi}(\theta, \varphi)
    \cdot
    \frac{\partial \gamma}{\partial \varphi}(\theta, \varphi)
    &=
    \begin{pmatrix}
        - \sin \theta \sin \varphi \\
        \sin \theta \cos \varphi \\
        0
    \end{pmatrix}
    \cdot
    \begin{pmatrix}
        - \sin \theta \sin \varphi \\
        \sin \theta \cos \varphi \\
        0
    \end{pmatrix} \\
    &= \sin^2 \theta \sin^2 \varphi + \sin^2 \theta \cos^2 \varphi \\
    &= \sin^2 \theta,
\end{align*}
\begin{align*}
    \frac{\partial \gamma}{\partial \theta}(\theta, \varphi)
    \cdot
    \frac{\partial \gamma}{\partial \varphi}(\theta, \varphi)
    &=
    \begin{pmatrix}
        \cos \theta \cos \varphi \\
        \cos \theta \sin \varphi \\
        - \sin \theta
    \end{pmatrix}
    \cdot
    \begin{pmatrix}
        - \sin \theta \sin \varphi \\
        \sin \theta \cos \varphi \\
        0
    \end{pmatrix} \\
    &= - \cos \theta \cos \varphi \sin \theta \sin \varphi
       + \cos \theta \sin \varphi \sin \theta \cos \varphi \\
    &= 0.
\end{align*}
積分の計算中に現れる行列の積をあらかじめ計算しておくと
\begin{align*}
    \begin{pmatrix}
        \displaystyle
        \frac{\partial \gamma}{\partial \theta}(\theta, \varphi)
        &
        \displaystyle
        \frac{\partial \gamma}{\partial \varphi}(\theta, \varphi)
    \end{pmatrix}^{\transpose}
    \begin{pmatrix}
        \displaystyle
        \frac{\partial \gamma}{\partial \theta}(\theta, \varphi)
        &
        \displaystyle
        \frac{\partial \gamma}{\partial \varphi}(\theta, \varphi)
    \end{pmatrix}
    =
    \begin{pmatrix}
        1 & 0 \\
        0 & \sin^2 \theta(t) \\
    \end{pmatrix}
\end{align*}
となる.
\(s\)を
\begin{align*}
    s = \int_0^t
        \sqrt{(\dot{\theta}(t))^2 + (\dot{\varphi}(t))^2 \sin^2 \theta(t)}
    dt
\end{align*}
と置くと
\begin{align*}
    \frac{ds}{dt} = \sqrt{(\dot{\theta}(t))^2 + (\dot{\varphi}(t))^2 \sin^2 \theta(t)}.
\end{align*}
以降,表記を簡潔にするため,ニュートンの記法(Newton's notation)を用いる.
これは関数\(f\colon t \mapsto \placeholder\)の\(t\)に関する導関数を\(\dot{f}\)のように表すものである.
\mathref{great-circle-length-integral}の被積分関数を\(F \colon [0, 1] \times \symbb{R}^2 \times \symbb{R}^2 \to \symbb{R}\)は
\begin{align*}
    F\left(
        s,
        \begin{pmatrix}
            \theta(t(s)) \\
            \varphi(t(s))
        \end{pmatrix},
        \begin{pmatrix}
            \dot{\theta}(t(s)) \\
            \dot{\varphi}(t(s)) \\
        \end{pmatrix}
    \right)
    &=
    \sqrt{%
        \begin{pmatrix}
            \dot{\theta} & \dot{\varphi}
        \end{pmatrix}
        \begin{pmatrix}
            1 & 0 \\
            0 & \sin^2 \theta \\
        \end{pmatrix}
        \begin{pmatrix}
            \dot{\theta} \\ \dot{\varphi}
        \end{pmatrix}
    } \\
    &= \sqrt{\dot{\theta}^2 + \dot{\varphi}^2 \sin^2 \theta}.
\end{align*}
この問題のEuler--Lagrange方程式は
\begin{align*}
    \begin{pmatrix}
        \displaystyle
        \frac{\partial F}{\partial \theta}
        &
        \displaystyle
        \frac{\partial F}{\partial \varphi}
    \end{pmatrix}
    -
    \frac{d}{ds}
    \begin{pmatrix}
        \displaystyle
        \frac{\partial F}{\partial \dot{\theta}}
        &
        \displaystyle
        \frac{\partial F}{\partial \dot{\varphi}}
    \end{pmatrix}
    = 0
\end{align*}
である.
\mathref{great-circle-regular}と\(t > 0\)において\(\sin \theta(t) > 0\)であるから,成分ごとに計算して
\begin{numcases}
    \displaystyle
    \frac{%
        \dot{\varphi}^2 \sin \theta \cos \theta
    }{%
        \dfrac{ds}{dt}
    }
    =
    \frac{d}{ds}
    \frac{%
        \dot{\theta}
    }{%
        \dfrac{ds}{dt}
    } \mathlabel{great-circle-ode1} \\[10pt]
    \displaystyle
    0 =
    \frac{d}{ds}
    \frac{%
        \dot{\varphi}(t) \sin^2\theta(t)
    }{%
        \dfrac{ds}{dt}
    } \mathlabel{great-circle-ode2}
\end{numcases}
を得る.
\begin{align*}
    \text{(\mathref{great-circle-ode1}の右辺)}
    = \frac{d}{dt}\left(\frac{d \theta}{dt}\frac{1}{\displaystyle \frac{ds}{dt}}\right)
    = \frac{d}{dt}\left(\frac{d\theta}{ds}(t(s))\right)
\end{align*}
\begin{align*}
    \text{(\mathref{great-circle-ode2}の右辺)}
    = \frac{d}{dt}\left(\frac{d \varphi}{dt}\frac{1}{\displaystyle \frac{ds}{dt}}\sin^2 \theta(t(s))\right)
    = \frac{d}{dt}\left(\frac{d \varphi}{ds}(t(s)) \sin^2 \theta(t(s))\right)
\end{align*}

\mathref{great-circle-ode2}から
\begin{align*}
    \frac{\dot{\varphi}(t) \sin^2\theta(t)}{\sqrt{(\dot{\theta}(t))^2 + (\dot{\varphi}(t))^2 \sin^2 \theta(t)}}
    =
    \frac{\dot{\varphi}(0) \sin^2\theta(0)}{\sqrt{(\dot{\theta}(0))^2 + (\dot{\varphi}(0))^2 \sin^2 \theta(0)}}
\end{align*}
が成り立つ.\(\theta(0) = 0\)から\(0 < t < 1\)において\(\dot{\varphi}(t) = 0\)である.
\(q\)が球面座標で\(\tuple{1, \theta_1, \varphi_1}\)と表されているとき,定数関数\(\varphi\colon t \mapsto \varphi_1\)をとると,\(\dot{\varphi} = 0\)と\(\gamma(1) = q\)をみたす.
\(\theta\)としては例えば\(\theta\colon t \mapsto t \theta_1\)をとればよい.
以上の結果から\(\gamma\)が\(p\)と\(q\)を通る大円の劣弧であることがわかる.

このように\(\symbb{R}^3\)の図形(多様体)の上で\(2\)点を結ぶ最短の曲線を求める問題は,Riemann多様体\(\tuple{M, g}\)上での問題に一般化できる.集合\(\Omega(p, q)\)を
\begin{align*}
    \Omega(p, q) = \{%
        \gamma\colon [a, b] \to M \mid
        \text{\(\gamma\)は区分的に滑らかな曲線で\(\gamma(a) = p\)かつ\(\gamma(b) = q\)}
    \}
\end{align*}
とする.ただし\(\gamma\to [a, b] \to M\)が区分的に滑らかであるとは分割\(a = t_0 < t_1 < \cdots < t_n = b\)が存在して,各区間\([t_{i - 1}, t_i]\)への制限\(\gamma\restriction_{[t_{i - 1}, t_{i}]}\colon [t_{i - 1}, t_i] \to M\)が\(C^\infty\)級であることをいう.
\begin{align*}
    L(\gamma) = \int_{0}^{1} \sqrt{g\left(\frac{d\gamma}{dt}(t), \frac{d\gamma}{dt}(t)\right)} dt
\end{align*}

\begin{align}
    d_g(p, q) = \inf \{L(\gamma) \mid \gamma \in \Omega(p, q)\}
\end{align}
\keyword{Riemann距離}(Riemannian distance)という.
Riemann metricといってしまうと,それはRiemann計量を指すため注意が必要である.


\mathref{great-circle-distance}が距離の公理をみたすことを示す.
以下\(p, q, r\)は\(S^2(R) \subseteq \symbb{R}^3\)の点とする.
最初に非負性を示す.任意の\(x \in [-1 , 1]\)について
\begin{align*}
    0 \leq \arccos x \leq \pi
\end{align*}
から\(d(p, q) \geq 0\)である.
\(d(p, q) = 0\)となるのは\(p \cdot q = \lVert p \rVert \lVert q \rVert\)が成立するときかつそのときに限る.
これはCauchy--Schwarzの不等式において等号が成立する条件と同値であり,ある\(\lambda > 0\)が存在して\(p = \lambda q\)が成り立つということを意味する.
両辺のノルムをとると\(\lambda = 1\)がわかる.
したがって\(p = q\)が\(d(p, q) = 0\)の必要充分条件である.
対称性は\(\symbb{R}^3\)の内積が可換であることから明らかである.
最後に三角不等式を示す.
大円距離はRiemann距離の一例であるからRiemann距離について示せばよい.
任意の\(p, q, r\in M, \gamma_1 \in \Omega(p, r), \gamma_2 \in \Omega(r, q)\)をとる.
下限の定義から,任意の\(\varepsilon > 0\)について\(\gamma_1 \in \Omega(p, r), \gamma_2 \in \Omega(r, q)\)が存在して
\begin{gather*}
    L(\gamma_1) < d_g(p, r) + \frac{\varepsilon}{2},
    \quad
    L(\gamma_2) < d_g(r, q) + \frac{\varepsilon}{2}
\end{gather*}
が成り立つ.したがって
\begin{align*}
    d_g(p, q) \leq L(\gamma_1 + \gamma_2)
            = L(\gamma_1) + L(\gamma_2)
            < d_g(p, r) + d_g(r, q) + \varepsilon.
\end{align*}
\(\varepsilon > 0\)が任意であったから三角不等式\(d_g(p, q) \leq d_g(p, r) + d_g(r, q)\)が成り立つ.

\end{document}
