\usepackage{booktabs}
\usepackage[%
    hang,
    flushmargin
]{footmisc}
\usepackage{amsmath}
\usepackage{amssymb}
\usepackage{amsthm}
\usepackage{bussproofs}
\usepackage{unicode-math}
\usepackage{tikz}
\usepackage{pgfplots}

\usepackage{xcolor}
\definecolor{sBlue}{HTML}{3498DB}
\colorlet{sOriginalCurve}{sBlue}

\definecolor{sRed}{HTML}{FF0033}
\colorlet{sApproxCurve}{sRed}

\definecolor{sGreen}{HTML}{00CC00}
\colorlet{sSimpleCurve}{sGreen}

\definecolor{sFill}{HTML}{00A381}
\definecolor{sLightGray}{gray}{0.85}

\usepackage{polyglossia}
\setdefaultlanguage{japanese}
\setotherlanguages{english,german,russian,hebrew,greek}

\usepackage[
    deluxe,%
    expert,
    scale=0.95
]{luatexja-preset}
\usepackage{luatexja-ruby}
\setmainfont[%
    ItalicFont=NewCM10-Italic.otf,
    BoldFont=NewCM10-Bold.otf,
    BoldItalicFont=NewCM10-BoldItalic.otf,
    SmallCapsFeatures={Numbers=OldStyle},
]{NewCM10-Regular.otf}
\setmathfont[math-style=ISO]{New Computer Modern Math}
\setmathfont[
    range={
        cal,%
        bfcal
    },
    StylisticSet=1
]{KpMath}
\setsansfont[BoldFont={Roboto-Medium}]{Roboto}
\newfontfamily\sourcecodepro{Source Code Pro}

\ltjnewpreset{book}{%
    mc-m = A-OTF-RyuminPro-Regular.otf,
    gt-m = BIZUDGothic-Regular.ttf,
    gt-b = BIZUDGothic-Bold.ttf,
    mg-m = MotoyaMaruStd-W5.otf
}
\ltjapplypreset{book}
\setmonojfont{HiraMaruProN-W4}
\ltjsetparameter{%
    jacharrange={%
        -2, % Exclude Greek and Cyrillic letters.
        -3, % Punctuations and miscellaneous symbols.
        -7  % other CJK characters
    },
    alxspmode={`/,allow},
    alxspmode={`#,allow},
    alxspmode={92,allow} % backslash
}

\usepackage{wtref}
\newref{fig}
\setrefstyle{fig}{prefix=図}
\newref{eq}
\setrefstyle{eq}{prefix=式(, suffix=)}
\newref{def}
\setrefstyle{def}{prefix=定義}
\newref{axiom}
\setrefstyle{axiom}{prefix=公理}
\newref{proposition}
\setrefstyle{proposition}{prefix=命題}
\newref{lemma}
\setrefstyle{lemma}{prefix=補題}
\newref{example}
\setrefstyle{example}{prefix=例}
\newref{theorem}
\setrefstyle{theorem}{prefix=定理}
\newref{sec}
\setrefstyle{sec}{prefix={第},suffix={節}}


\newtheoremstyle{jplain}%
    {0.6\baselineskip}%
    {0.6\baselineskip}%
    {\normalfont}%
    {}%
    {\bfseries\gtfamily\sffamily}%
    {}%
    {5pt}%
    {\thmname{#1}\thmnumber{#2}\thmnote{#3}\ignorespaces}
\theoremstyle{jplain}
\newtheorem{theorem}{定理}[chapter]
\newtheorem{proposition}[theorem]{命題}
\newtheorem{lemma}[theorem]{補題}
\newtheorem{definition}[theorem]{定義}
\newtheorem{example}[theorem]{例}
\newtheorem{axiom}[theorem]{公理}
\renewcommand{\proofname}{証明}
\renewcommand{\qedsymbol}{\rule{5pt}{10pt}}

\usepackage{tcolorbox}
\tcbuselibrary{%
    skins,
    breakable
}
\newtcolorbox{thmbox}{%
    colback=sLightGray,
    top=8pt,
    bottom=8pt,
    left=8pt,
    right=8pt,
    boxrule=0pt,
    sharp corners,
    frame hidden
}

\makeatletter
\renewenvironment{proof}[1][\proofname]{\par
  \pushQED{\qed}%
  \normalfont \topsep6\p@\@plus6\p@\relax
  \trivlist
  \item[\hskip\labelsep\bfseries\gtfamily#1]\ignorespaces
}{%
  \popQED\endtrivlist\@endpefalse
}
\makeatother
\AtBeginDocument{%
  \setlength{\abovedisplayskip}{5pt}%
  \setlength{\belowdisplayskip}{5pt}%
}

% https://note.com/yuw/n/n38dd54fb2169
\begingroup
\catcode`\,=\active
\def\@x@{\def,{\normalcomma\hskip.2em}}
\expandafter\endgroup\@x@%
\mathcode`\,="8000
\def\normalcomma{\mathchar"613B}

\usepackage{enumitem}
\newlist{conditions}{enumerate}{1}
\setlist[conditions]{label=(\arabic*)}

\usepackage{hyperref}
\hypersetup{%
    colorlinks,
    citecolor=sBlue,
    linkcolor=sBlue,
    urlcolor=sBlue
}

\usepackage[%
    backend=biber,
    style=pecorarista,
    sorting=nyvt,
    urldate=long
]{biblatex}
\addbibresource{references.bib}

\usepackage{titleps}
\newpagestyle{main}{%
    \headrule
    \sethead[\thepage][][{\sffamily\gtfamily {\thesection}\hspace{1\zw}\sectiontitle}]{\sffamily\gtfamily 第{\thechapter}章\hspace{1\zw}\chaptertitle}{}{\thepage}
}

\newcommand\mainchapter[1]{\chapter{#1}\thispagestyle{empty}}
\renewcommand{\jsParagraphMark}{}
\renewcommand{\headfont}{\bfseries\gtfamily\sffamily}

\newcommand\keyword[1]{{\bfseries\gtfamily #1}}

\newcommand\Integer{\symbb{N}}
\newcommand\PositiveInteger{\symbb{N}_{\mathord{+}}}
\newcommand\NonNegReal{\symbb{R}_{\mathord{+}}}
\newcommand\PositiveReal{\symbb{R}_{\mathord{+}\mathord{+}}}
\newcommand\placeholder{\mathord{\cdot}}
\newcommand\powerset[1]{\wp(#1)}
\newcommand\inlinefrac[2]{#1\mathbin{/}#2}
\DeclareMathOperator{\identity}{id}

\newcommand\header[1]{\multicolumn{1}{c}{\bfseries\gtfamily #1}}
