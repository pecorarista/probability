\documentclass{ltjsarticle}
\usepackage{graphicx}
\usepackage{fontspec}
\setmainfont{Times LT Pro}
\newfontfamily\phonetic{Doulos SIL}
\newcommand\liaison{\hspace*{0.1em}\raisebox{-0.8ex}{\rotatebox{90}{(}}\hspace*{0.1em}}
\usepackage{covington}
\setglossoptions{%
    fsi=\itshape,
    fsii=\phonetic,
    fstl=\normalfont\upshape\rmfamily\mcfamily,
    enquotetl=false,
}
\begin{document}
\trigloss{L-a théorie des hasard-s consist-e à réduire tous l-es événement-s}
         {la teɔʁi de azaʁ {k\~{ɔ}sist\liaison} a ʁedɥiʁ tu {lez\liaison} evɛnm\~{ɑ}}
         {the-\textsc{f}.\textsc{sg} theory of{\textunderscore}the\textbackslash\textsc{m}.\textsc{pl} chance-\textsc{pl} consist-3\textsc{sg}.\textsc{pres} to reduce all the-\textsc{m}.\textsc{pl} event-\textsc{pl}}
         {確率論は}
\trigloss{du même genre à un certain nombre de cas égal-e-ment possible-s,}
         {dy mɛm ʒ\~{ɑ}ʁ a \~{œ} sɛʁt\~{ɛ} n\~{ɔ}bʁ də kɑ eɡalm\~{ɑ} pɔsibl}
         {{of{\textunderscore}the\textbackslash\textsc{m}.\textsc{sg}} same kind to a certain number of {case{\textbackslash}\textsc{pl}} equal-\textsc{f}.\textsc{sg}-\textsc{adv} possible-\textsc{pl}}
         {同種のものからなる事象のすべてを,同程度起こりやすいいくつかの場合,}
\trigloss{c'-est-à-dire tel-s que nous soyons égal-e-ment indécis sur}
         {sɛtadiʁ tɛl kə nu swaj\~{ɔ} egalm\~{ɑ} \~{ɛ}desi syʁ}
         {that-is-to-say such-\textsc{pl} that we be.1\textsc{pl}.\textsc{pres}.\textsc{subj} equal-\textsc{f}.\textsc{sg}-\textsc{adv} {uncertain{\textbackslash}\textsc{m}.\textsc{sg}} on}
         {すなわち私たちがそれらの存在について同程度に確信をもてない場合に帰着させること,}
\trigloss{leur existence, et à déterminer le nombre de cas favorable-s}
         {{lœʁ\liaison} ɛgzist\~{ɑ}s e a detɛʁmine lə n\~{ɔ}bʁə də kɑ favɔʁabl}
         {their existence and to determine the number of {case{\textbackslash}\textsc{pl}} favorable-\textsc{pl}}
         {そして好ましい場合の数}
\trigloss{à l' événement dont on cherche la probabilité. Le rapport de ce nombre à celui de tous}
         {lə ʁapɔʁ də sə {n\~{ɔ}bʁ{\liaison}} a səlɥi də tu}
         {a the.\textsc{m}.\textsc{sg} event x x ch the.\textsc{f}.\textsc{sg} probability. The ratio of the number}
         {}
\end{document}
