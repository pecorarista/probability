\documentclass{ltjsarticle}
\usepackage{graphicx}
\usepackage{fontspec}
\setmainfont{Times LT Pro}
\newfontfamily\phonetic{Doulos SIL}
\newcommand\liaison{\hspace*{0.1em}\raisebox{-0.8ex}{\rotatebox{90}{(}}\hspace*{0.1em}}
\usepackage{covington}
\setglossoptions{%
    fsi=\itshape,
    fsii=\phonetic,
    fstl=\normalfont\upshape\rmfamily\mcfamily,
    enquotetl=false,
}
\begin{document}
\trigloss{L-a théorie des hasard-s consist-e à réduire tous l-es événement-s}
         {la teɔʁi de azaʁ {k\~{ɔ}sist\liaison} a ʁedɥiʁ tu {lez\liaison} evɛnm\~{ɑ}}
         {the-\textsc{f}.\textsc{sg} theory {of{\textunderscore}the[\textsc{m}.\textsc{pl}]} chance-\textsc{pl} consist-3\textsc{sg}.\textsc{pres} to {reduce[\textsc{inf}]} all the-\textsc{m}.\textsc{pl} event-\textsc{pl}}
         {確率論とは}
\trigloss{du même genre à un certain nombre de cas égal-e-ment possible-s,}
         {dy mɛm {ʒ\~{ɑ}ʁ{\liaison}} a \~{œ} sɛʁt\~{ɛ} n\~{ɔ}bʁ də kɑ eɡalm\~{ɑ} pɔsibl}
         {{of{\textunderscore}the[\textsc{m}.\textsc{sg}]} same kind to a certain number of {case[\textsc{pl}]} equal-\textsc{f}.\textsc{sg}-\textsc{adv} possible-\textsc{pl}}
         {同じ種類からなる事象すべてを,同程度起こりやすいいくつかの場合,}
\trigloss{c'-est-à-dire tel-s que nous soyons égal-e-ment indécis sur}
         {sɛtadiʁ tɛl kə nu swaj\~{ɔ} egalm\~{ɑ} \~{ɛ}desi syʁ}
         {that-is-to-say such-\textsc{pl} that we {be[1\textsc{pl}.\textsc{pres}.\textsc{subj}]} equal-\textsc{f}.\textsc{sg}-\textsc{adv} {uncertain[\textsc{m}.\textsc{pl}]} on}
         {すなわち私たちがそれらの存在について同程度に確信をもてない場合に還元すること,}
\trigloss{leur existence, et à déterminer l-e nombre de cas favorable-s}
         {{lœʁ\liaison} ɛgzist\~{ɑ}s e a detɛʁmine lə n\~{ɔ}bʁ də kɑ {favɔʁabl\liaison}}
         {their existence and to {determine[\textsc{inf}]} the-\textsc{m}.\textsc{sg} number of {case[\textsc{pl}]} favorable-\textsc{pl}}
         {そして確からしさを知りたい事象に関連する場合の数を決定すること}
\trigloss{à l' événement dont on cherch-e l-a probabilité.}
         {a {l{\liaison}} evɛnəm\~{ɑ} d\~{ɔ} \~{ɔ} ʃɛʁʃə la prɔbabilite}
         {to {the[\textsc{m}.\textsc{sg}]} event of{\textunderscore}which {people (3\textsc{sg})} seek-3\textsc{sg}.\textsc{pres} {the-\textsc{f}.\textsc{sg}} probability.}
         {に帰着される.}
\trigloss{L-e rapport de ce nombre à celui de tous l-es cas}
         {lə ʁapɔʁ də sə {n\~{ɔ}bʁ{\liaison}} a səlɥi də tu le kɑ}
         {the-\textsc{m}.\textsc{sg} ratio of {this[\textsc{m}.\textsc{sg}]} number to {that[\textsc{m}.\textsc{sg}]} of all {the-\textsc{m}.\textsc{pl}} {cases[\textsc{pl}]}}
         {この場合の数が全部の場合の数に占める割合が}
\trigloss{possible-s est l-a mesure de cette probabilité,}
         {{pɔsibl{\liaison}} ɛ la məzyʁ də sɛt pʁɔbabilite}
         {possible-\textsc{m}.\textsc{pl} {be[3\textsc{sg}.\textsc{pres}]} the-\textsc{f}.\textsc{sg} measure of {this[\textsc{f}.\textsc{sg}]} probability}
         {その事象の確からしさの尺度である.}
\trigloss{qui n' est ainsi qu' une fraction dont l-e numérateur}
         {ki n {ɛt{\liaison}} \~{ɛ}si {k{\liaison}} yn fʁaksj\~{ɔ} d\~{ɔ} lə {nymeʁatœʁ\liaison}}
         {which \textsc{neg} {be[3\textsc{sg}.\textsc{pres}]} thus only one fraction whose the numerator}
         {すなわちそれは分子が}
\trigloss{est l-e nombre des cas favorable-s, et dont}
         {ɛ lə n\~{ɔ}bʁ de kɑ favɔʁabl e d\~{ɔ}}
         {{be[3\textsc{sg}.\textsc{pres}]} the-\textsc{m}.\textsc{sg} number {of{\textunderscore}the[\textsc{m}.\textsc{pl}]} {case[\textsc{pl}]} favorable and whose}
         {関連する場合の数であり,}
\trigloss{l-e dénominateur est l-e nombre de tous les cas possibles}
         {lə {denɔminatœʁ\liaison} ɛ lə}
         {the-\textsc{m}.\textsc{sg} denominator {be[3\textsc{sg}.\textsc{pres}]} the-\textsc{m}.\textsc{sg} number of all the-\textsc{m}.\textsc{sg} case[\textsc{pl}] possible-\textsc{pl}}
         {分母が起こり得る場合すべての数であるような分数にすぎない.}
\end{document}
