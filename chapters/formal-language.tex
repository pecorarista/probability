\documentclass[../main.tex]{subfiles}
\begin{document}
\mainchapter{形式言語}

\begin{thmbox}
\begin{definition}
空ではない有限集合のことを\keyword{アルファベット}(alphabet)といい,
しばしば\(\Sigma\)で表す.
アルファベットの元を\keyword{記号}(symbol)と呼ぶ.
\definitionlabel{alphabet}
\end{definition}
\end{thmbox}

\begin{exa} \(\Sigma = \{\symtt{0}, \symtt{1}\}\)はアルファベットである.
\end{exa}

\begin{exa} 現代英語で主に用いられるラテン小文字の集合\(\Sigma = \{\symtt{a}, \ldots, \symtt{z}\}\)はアルファベットである.
\end{exa}

\begin{exa} 記号が1文字で表される必要はないので\(\symcal{V} = \{\text{\ttfamily 株価}, \text{\ttfamily 上がる}, \text{\ttfamily 下がる}\}\)\footnote{語彙(vocabulary)のような意味合いをこめて\(\symcal{V}\)と表した.}などもまたアルファベットである.
\end{exa}

\begin{thmbox}
\begin{definition}
\(n\)を任意の自然数とする.アルファベット\(\Sigma\)の元からなる\(n\)-タプル全体の集合を
\(\Sigma^*\)と書くとき,\(\Sigma^*\)の元を\keyword{文字列}(string)と呼ぶ.
特に\(0\)-タプル\(\tuple{}\)を\(\Sigma\)上の\keyword{空文字列}(empty string)と呼び,しばしば\(\varepsilon\)で表される.
\end{definition}
\end{thmbox}

\(\Sigma\)上の文字列全体の集合とはすなわち
\begin{align*}
    \Sigma^{\mathord{*}}
    &:= \{\tuple{}\} \cup \{\tuple{a_1} \mid a_1 \in \Sigma \} \cup \{\tuple{a_1, a_2} \mid a_1, a_2 \in \Sigma \} \cup \cdots \\
    &= \{\tuple{}\} \cup \bigcup_{\ell \in \PositiveInteger} \{\tuple{a_1, \ldots, a_\ell} \mid a_1, \ldots, a_\ell \in \Sigma\} \\
    &= \bigcup_{\ell \in \symbb{N}} \prod_{i = 1}^\ell \Sigma^i \\
\end{align*}
である.\(2\)行目から\(3\)行目の変形では
\begin{align*}
    \prod_{i = 1}^0 \Sigma^i
    = \prod_{i \in \{n \mid n \in \symbb{N}, 1 \leq n \leq 0\}} \Sigma^i
    = \prod_{i \in \emptyset} \Sigma^i
    = \{\tuple{}\}
\end{align*}
を利用している.

文字列\(\alpha = \tuple{a_1, a_2,\ldots, a_n}\)をしばしば\(a_1 a_2 \cdots a_n\)のように表す.
\(\Sigma^{\mathord{*}}\)の2元\(s = a_1 \cdots a_n\), \(t = b_1 \cdots b_m\)に対して
\keyword{連結}(concatenation)と呼ばれる二項演算を
\begin{align}
    s \doubleplus t = a_1 \cdots a_n b_1 \cdots b_m
    % \in \Sigma^{n + m} \subseteq \Sigma^{\mathord{+}}
    \eqlabel{concatenation}
\end{align}
で定義する.\(s \doubleplus t\)は単に\(st\)と書かれることが多い.
任意の\(s \in \Sigma^{\mathord{*}}\)に対して
\begin{align}
    \varepsilon s = s \varepsilon = s \eqlabel{empty-string}
\end{align}
が成り立つ.したがって\(\tuple{ \Sigma^*, \doubleplus, \varepsilon }\)はモノイドである.

\begin{exa} \(\Sigma = \{\symtt{0}, \symtt{1}\}\)とすると,
\(\symtt{0000}\)や\(\symtt{0101}\)は文字列である.
\end{exa}

\begin{thmbox}
\begin{definition}
アルファベット\(\Sigma\)上の任意の文字列について,その\keyword{長さ}(length)を以下のように定義する:
\begin{align*}
    \len (s) = n \; \Longleftrightarrow \; s \in \Sigma^n.
\end{align*}
\end{definition}
\end{thmbox}

\begin{exa} \(\Sigma = \{\symtt{0}, \symtt{1}\}\)とすると\(\len(\symtt{0110}) = 4\)であり,
\(\Sigma = \{\symtt{01}, \symtt{10}\}\)とすると\(\len(\symtt{0110}) = 2\)である.
\end{exa}


\begin{definition} \(\Sigma^{\mathord{*}}\)の部分集合を\(\Sigma\)上の\keyword{言語}(language)という.
\begin{itemize}
    \item \(L_1, L_2\)が言語ならば\(L_1L_2 = \{ w \doubleplus v \mid w \in L_1, v \in L_2 \}\)
    \item \(L^{\symsf{R}} = \{ \}\)
\end{itemize}
\end{definition}


\begin{definition}
\(\langle Q, \Sigma, \delta, q_0, F \rangle\)
\(Q\)は有限集合で,\(Q\)の元は\keyword{状態}(state)と呼ばれる.
\(\Sigma\)は有限集合で,\(\Sigma\)の元は\keyword{入力記号}(input symbol)と呼ばれる.
\(q_0 \in Q\)は初期状態
\(F \subseteq Q\)は
\(\delta\colon Q \times \Sigma \to \wp(Q)\)
\end{definition}

\(Q = \{q_0, q_1, q_2\}\)
\(\Sigma = \{0, 1\}\)

\begin{definition}
言語\(L = \Sigma^{\mathord{*}}\)
言語の連結\(L\)
\end{definition}

\begin{itemize}
    \item \formallang{S} \(\to\) \formallang{NP} \formallang{VP}
    \item \formallang{VP} \(\to\) \formallang{V}
    \item \formallang{V} \(\to\) \formallang{V}
\end{itemize}

\begin{figure}
    \centering
    \begin{tikzpicture}[every node/.append style={font=\treefont\itshape}]
    \Tree [.S [.NP [.Det the ] [.N cat ] ]
    [.VP [.V sat ]
    [.PP [.P on ]
    [.NP [.Det the ] [.N mat ] ] ] ] ];
    \end{tikzpicture}
\end{figure}

\begin{definition} \(G = \langle V, \Sigma, S \rangle\)を\keyword{文脈自由文法}(context-free grammar, CFG)と呼ぶ.
\begin{itemize}
    \item \(\Sigma\)は終端記号(terminal symbol)と呼ばれる記号の集合
    \item \(N\)は\keyword{非終端記号}(nonterminal symbol)と呼ばれる全体の集合
    \item \(S \in N\)は開始記号(start symbol)と呼ばれる非終端記号
    \item \(S \in N\)は開始記号(start symbol)と呼ばれる非終端記号
\end{itemize}

\(V = N \cup \Sigma\)

\end{definition}

\textgreek{Καραθεοδωρή}

\end{document}
