\documentclass[../main.tex]{subfiles}
\begin{document}
\mainchapter{代数学}

この章では代数学の用語について紹介する.
その目的は以降の章でそれらを用いて説明を簡略にすることである.
したがって用語に親しむために最低限の具体例の提示は行うものの,
代数学的におもしろい部分にはほとんど踏み込まないことをご了承いただきたい.

\section{代表的な代数構造}

\begin{thmbox}
\begin{definition}
\(X\)を集合とし,\(n\)を正整数とする.
\(X^n\)から\(X\)への写像を\(X\)上の\keyword{\(n\)項演算}(\(n\)-ary operation)と呼ぶ{\footnotemark}.
\end{definition}
\end{thmbox}

\footnotetext{\(n\)が小さいときは1-ary, 2-ary, 3-aryなどとはいわず,unary, binary, ternaryなどというのが普通である.}

\(n\)項演算を単に演算ということもある.
\(2\)項演算\(f \colon X \times X \to X\)による\(x, y\)の像\(f(x, y)\)はしばしば中置記法によって\(a \mathbin{f} b\)のように書かかれる.
今は\(f\)という記号を用いたが,通常は\(\mathord{+}\)や\(\mathord{\times}\)などの記号を用いる.
記号\(\mathord{+}\)で表される演算を\keyword{和}または\keyword{加法}(addition)と呼ぶ.
同様に\(\mathord{\times}\)や\(\mathord{\cdot}\)などで表される演算を\keyword{積}(product)または\keyword{乗法}(multiplication)と呼ぶ.
積の記号はよく省略される.すなわち\(a \times b\)や\(a \cdot b\)は多くの場合\(ab\)のように書かれる.

集合\(X\)の部分集合\(S\)と\(X\)上の\(n\)項演算\(f\)が与えられたとき,
任意の\(x_1, \ldots, x_n \in S\)について\(f(x_1, \ldots, x_n) \in S\)が成り立つ,
すなわち\(f(S^n) \subseteq S\)であるとき,
\keyword{\(S\)は\(f\)について閉じている}(\(S\) is closed under \(f\))
あるいは\keyword{\(f\)-閉}(\(f\)-closed)であるという.

\begin{exa} \(\symbb{Z}\)の部分集合\(\symbb{N}\)は,加法\(\mathord{+}\)について閉じている.
しかし1項演算\(\mathord{-}\colon n \mapsto -n\)については閉じていない.
例えば\(-1 \not\in \symbb{N}\)である.
\end{exa}

\begin{thmbox}
\begin{definition} 集合\(X\)とその上の演算\(\mathord{\cdot}\colon S \times S \to S\)が以下の条件をみたすとき,組\(\langle X, \mathord{\cdot} \rangle\)を\keyword{モノイド}(monoid)という:
\begin{conditions}
    \item (結合律)\(X\)の任意の元\(a, b, c\)に対して,\((a \cdot b) \cdot c = a \cdot (b \cdot c)\)が成り立つ.
    \item (単位元の存在)\(X\)の元\(e\)が存在して,\(S\)の任意の元に対して\(e \cdot a = a \cdot e = a\)が成り立つ.
\end{conditions}
さらに次の可換律をみたすとき,\(\langle X, \mathord{\cdot} \rangle\)を\keyword{可換モノイド}(commutative monoid)という:
\begin{conditions}[resume]
    \item (可換律)\(X\)の任意の元\(a, b\)に対して,\(a \cdot b = b \cdot a\)が成り立つ.
\end{conditions}
\end{definition}
\end{thmbox}

モノイド\(\langle X, \mathord{\star} \rangle\)の単位元が\(e\)であるとき,それを\(\tuple{ X, \mathord{\star}, e }\)のように表すこともある.
どのような演算や単位元を考えているのか,書き手・読み手間あるいは話し手・聞き手間で共有されている場合,集合と同じく単に\(X\)と表すことが多い.以下この種の注釈は省く.

\begin{exa} 整数全体の集合\(\symbb{Z}\)とその上の通常の加法\(\mathord{+}\)の組\(\tuple{ \symbb{Z}, \mathord{+} }\)はモノイドである.
\end{exa}

\begin{exa} アルファベット\(\Sigma\)のKleeneスター\(\Sigma^{\mathord{*}}\)(\eqref{kleene-star}),連結(\eqref{concatenation}),空文字列(\eqref{empty-string})からなる3つ組\(\langle \Sigma^{\mathord{*}}, \doubleplus, \varepsilon \rangle\)はモノイドである.
\end{exa}

\begin{thmbox}
\begin{definition} 以下の性質をみたすとき,組\(\langle X, \mathord{+}, \mathord{\cdot} \rangle\)を\keyword{半環}(semiring)\index{半環 semiring@はんかん}という.
\begin{conditions}
    \item \(\langle X, \mathord{+}, 0 \rangle\)は可換モノイドである.
    \item \(\langle X, \mathord{\cdot}, 1 \rangle\)はモノイドである.
    \item 分配律が成り立つ.
    \item 乗法によって加法単位元\(0\)が\(R\)を\keyword{消す}(annihilate).すなわち
        任意の\(a \in R\)について
        \begin{align}
            0 \cdot a = a \cdot 0 = 0
        \end{align}
        が成り立つ.
\end{conditions}
\definitionlabel{semiring}
\end{definition}
\end{thmbox}

\begin{definition} 記号\(+\infty\)で表される「数」を正の無限大,
\(- \infty\)で表される「数」を負の無限大という.
\(\symbb{R} \cup \{\pm \infty\}\)の順序および演算を以下にように定義する:
\begin{itemize}
    \item 任意の\(a \in \symbb{R}\)について\(- \infty < a < + \infty\).
    \item 任意の\(a \in \symbb{R} \cup \{\infty\}\)に対して\(a + \infty = (+ \infty) + a = \infty\).
    \item 任意の\(a \in \symbb{R} \cup \{-\infty\}\)に対して\(a - \infty = (- \infty) + a = -\infty\).
    \item \(a \in \symbb{R} \cup \{\pm \infty\}\)とする.
        \begin{align*}
            a \cdot (\pm \infty) = (\pm \infty) \cdot a =
            \begin{cases}
                \pm \infty & \text{if \(a > 0\),}\\
                0          & \text{if \(a = 0\),}\\
                \mp \infty & \text{otherwise.} \\
            \end{cases}
        \end{align*}
    \item \(a \in (\symbb{R} \setminus \{0\}) \cup \{\pm \infty\}\)とする.
        \begin{align*}
            \frac{\pm \infty}{a} =
            \begin{cases}
                \pm \infty & \text{if \(a > 0\),}\\
                \mp \infty & \text{otherwise.}
            \end{cases}
        \end{align*}
    \item 任意の\(a \in \symbb{R}\)について\(a/(\pm \infty) = 0\).
\end{itemize}
ここで式\(a + \infty\)は\(a + (+\infty)\)の意味でもあり\(a - (- \infty)\)の意味でもある.
また式\(a - \infty\)は\(a - (+ \infty)\)の意味でもあり,\(a + (- \infty)\)の意味でもある.
このような解釈の下\(+ \infty\)を単に\(\infty\)とも書く.集合\(\symbb{R} \cup \{\pm \infty\}\)を
\(\overline{\symbb{R}}\)と表し,\keyword{拡張された実数}(extended real numbers)と呼ぶ.
拡張された実数においても\(\infty - \infty\)のような不定形に具体的な計算規則を与えることはしない.
ただし\(0 \cdot (\pm \infty)\)の形の不定形については,既に上で示したように\(0 \cdot (\pm \infty) = 0\)とするのが(測度論
\definitionlabel{extended-real}
\end{definition}


\begin{exa} 正の無限大を含む実数全体の集合\(\symbb{R} \cup \{\infty\}\)上の加法\(\land\)を
\begin{align*}
    a \land b = \min \{a, b\}
\end{align*}
とすると,\(\langle \symbb{R} \cup \{\infty\}, \land, \infty\rangle\)は可換モノイドである.
また乗法\(\mathord{\ast}\)を\(\symbb{R}\)における通常の加法に,無限大に関する規則(\definitionref{extended-real})を付け加えたものとする.このとき\(\langle \symbb{R} \cup \{\infty\}, \mathord{\ast}, 0 \rangle\)はモノイドである.
分配律が成り立つことは以下のように確認できる:
\begin{align*}
    a \mathbin{\ast} (b \land c)
    &= a + \min \{b, c\} \\
    &= a +
    \begin{cases}
        b & \text{if \(b \geq c\),} \\
        c & \text{otherwise}
    \end{cases} \\
    &=
    \begin{cases}
        a + b  & \text{if \(b \geq c\),} \\
        a + c & \text{otherwise}
    \end{cases} \\
    &= (a \mathbin{\ast} b) \land (a \mathbin{\ast} c).
\end{align*}
    加法単位元\(\infty\)は任意の\(a \in \symbb{R} \cup \{\infty\}\)について\(a \mathbin{\ast} \infty = a + \infty = \infty\)をみたすので\(\symbb{R}\cup\{\infty\}\)を消す.したがって\(\tuple{ \symbb{R} \cup \{\infty\}, \mathord{\land}, \mathord{\ast} }\)は半環である.この半環をトロピカル半環(tropical semiring)という.
\end{exa}

\begin{thmbox}
\begin{definition}[(ベクトル空間)]
\end{definition}
\end{thmbox}

\begin{thmbox}
\begin{definition}[(代数)]
\(A\)を\(\symbb{K}\)上のベクトル空間とする.
積または乗法と呼ばれる\(A\)上の二項演算\((x, y) \mapsto x \mathbin{\times} y\)(通常\(x \mathbin{\times} y\)を単に\(xy\)を書く)が以下の性質をみたすとき,
組\((A, \mathord{\times})\)は\(\symbb{K}\)上の\keyword{代数}(algebra)であるという.
\begin{conditions}
    \item\label{algebra-mult-assoc} 任意の\(x, y, z \in A\)について\(x (yz) = (xy) z\).
    \item 任意の\(x, y, z \in A\)について\((x + y) z = xz + yz, x (y + z) = xz + xz\).
    \item\label{algebra-mult-scalar-assoc} 任意の\(\alpha \in \symbb{K}\)と任意の\(x, y \in X\)について
        \(\alpha (xy) = (\alpha x) y = x (\alpha y)\).
\end{conditions}
どういう演算であるかを強調したい場合以外は組\((A, \mathord{\times})\)を単に\(A\)と書くことが多い.
上の\ref{algebra-mult-assoc}--\ref{algebra-mult-scalar-assoc}に加えて次をみたすとき,
代数\(A\)は可換であるという.
\begin{conditions}[resume]
    \item 任意の\(x, y\in A\)について\(x y  = y x\).
\end{conditions}
また\ref{algebra-mult-assoc}--\ref{algebra-mult-scalar-assoc}に加えて次をみたすとき,
代数\(A\)は単位元をもつ(with unit)または単位的(unital)であるという.
\begin{conditions}[resume]
    \item ある元\(1_A \in A \setminus \{0_A\}\)が存在して,
        任意の\(x \in A\)に対して\(1_A x  = x 1_A = x\)が成り立つ.
        これを乗法単位元と呼ぶ.
\end{conditions}
\end{definition}
\end{thmbox}

\noindent 混乱のおそれがない場合は\(0_A\)や\(1_A\)を単に\(0\)や\(1\)と表す.

\section{部分集合族の代数的性質}

\(\symcal{F}\)を集合\(X\)の部分集合族とする.
\(\symcal{F}\)上の演算について
\begin{itemize}
    \item (\(\mathord{\cap}\)-閉性)\(A, B \in \symcal{F}\)ならば\(A \cap B \in \symcal{F}\)が成り立つ.
    \item (\(\mathord{\cup}\)-閉性)\(A, B \in \symcal{F}\)ならば\(A \cup B \in \symcal{F}\)が成り立つ.
    \item (\(\mathord{\setminus}\)-閉性)\(A, B \in \symcal{F}\)ならば\(A \setminus B \in \symcal{F}\)が成り立つ.
    \item (\(\setcomp{\placeholder}\)-閉性)\(A \in \symcal{F}\)ならば\(\setcomp{A} = X \setminus A \in \symcal{F}\)が成り立つ.
    \item (\(\sigma\)-\(\mathord{\cup}\)-閉性)\(\symcal{F}\)の任意の列\(\Sequence{A}\)について
        \begin{align*}
            \bigcup_{n \in \PositiveInteger} A_n \in \symcal{F}.
        \end{align*}
\end{itemize}
\(\mathord{*}\)を\(\symcal{F}\)上の演算とする.
\(\symcal{F}\)が\(\mathord{*}\)-閉性をみたすとき,\(\symcal{F}\)は\(\mathord{*}\)-閉(\(\mathord{*}\)-closed)である,あるいは\(\mathord{*}\)について閉じている(closed under \(\mathord{*}\))ということもある.
「\(\sigma\)-\(\mathord{\cup}\)-閉」の〈\(\sigma\)〉は可算個の集合の和集合に関係しているものにつけられる接辞で,ドイツ語において和や和集合を意味する\begin{german}«Summe»\end{german}\footnote{\begin{german}Summe\end{german} [ˈzʊmə].現代のドイツ語で和集合を意味する語としては\textgerman{«Vereinigung»} [fɛ\invbreve{ɐ}ˈʔa\invbreve{ɪ}nɪgʊŋ]のほうが一般的である.
% 可算個の集合の共通部分に関係しているものには\textgerman{«Durchschnitt»}
% [ˈdʊʁçʃnɪt]
% にちなんで\(\delta\)が使われる.
% 例えば可算個の開集合\textgerman{«Gebiet»} [ɡə\/ˈbiːt]の共通部分であるような集合を\(G_\delta\)と呼ぶ.
}の\emphchar{s}に対応するギリシャ文字である.

\begin{thmbox}
\begin{definition}
集合\(X\)の部分集合族\(\symcal{F}\)が演算\(\mathord{\cap}\)について閉じているとき,\(\symcal{F}\)を
\keyword{\(\symbf{\pi}\)-システム}(\(\pi\)-system)という.
\end{definition}
\end{thmbox}

\begin{thmbox}
\begin{definition}
集合\(X\)の部分集合族\(\symcal{F}\)が次の性質をみたすとき,\(\symcal{F}\)を\keyword{集合半環}(semiring of sets)という:
\begin{conditions}
    \item\label{semiring-emptyset} \(\emptyset \in \symcal{F}\).
    \item (\(\mathord{\cap}\)-閉性)\(A, B \in \symcal{F}\)ならば\(A \cap B \in \symcal{F}\)が成り立つ.
    \item\label{semiring-setminus}(区分的\(\mathord{\setminus}\)-閉性\footnotemark)
        \(A, B \in \symcal{F}\)ならば,互いに素な\(C_1, \ldots, C_n \in \symcal{F}\)が存在して\(A \setminus B = \bigcup_{i = 1}^n C_i\)が成り立つ.
\end{conditions}
\definitionlabel{semiring-of-sets}
\end{definition}
\end{thmbox}

\footnote{「区分的\(\mathord{\setminus}\)-閉性」は便宜的につけたもので一般的な名称ではない.}

\begin{thmbox}
\begin{definition}
\(X\)を集合とし,\(\symcal{F}\)をその部分集合族とする.
\(\symcal{F}\)が以下の性質\ref{sigma-algebra-x-in-f}--\ref{sigma-algebra-sigma-cup-closed}をみたすとき,
\(\symcal{F}\)を\(X\)上の\keyword{\(\symbf{\sigma}\)-代数}(\(\sigma\)-algebra)という{\footnotemark}.
\begin{conditions}
    \item\label{sigma-algebra-x-in-f}
        \(X \in \symcal{F}\).
    \item\label{sigma-algebra-comp-closed} (\(\setcomp{\placeholder}\)-閉性) \(A \in \symcal{F}\)ならば\(\setcomp{A} \in \symcal{F}\).
    \item\label{sigma-algebra-sigma-cup-closed} (\(\sigma\)-\(\mathord{\cup}\)-閉性)\(\symcal{F}\)の任意の列\(\Sequence{A}\)について
        \begin{align*}
            \bigcup_{n \in \PositiveInteger} A_n \in \symcal{F}.
        \end{align*}
\end{conditions}
\end{definition}
\end{thmbox}

\footnotetext{\inhibitglue 「\(\sigma\)-加法族」という訳語もよく用いられる.}

一般に集合半環は代数学の意味での半環ではないが,よく似た性質をもっている.
まず\(\symcal{F}\)上の加法を\keyword{対称差}(symmetric difference)とする.
対称差とは\(\symdiff{A}{B} = (A \setminus B) \cup (B \setminus A)\)と定義されるもので,図示すると\figref{symmetric-difference}のようになる.
\(\symcal{F}\)が集合半環の条件をみたすならば,\(\symcal{F}\)は\(\symdiffsymbol\)について閉じている.
また\(\symdiffsymbol\)について結合律と可換律が成り立つ.
この演算の単位元\(\emptyset\)は定義から\(\symcal{F}\)に含まれている.
よって\(\langle \symcal{F}, \mathord{\triangle}, \emptyset \rangle\)は可換モノイドである.
次に\(\symcal{F}\)上の乗法を\(\mathord{\cap}\)とする.
これは結合律・可換律・分配律をみたしている.
また加法単位元\(\emptyset\)は乗法\(\mathord{\cap}\)によって\(\symcal{F}\)を零化する.
しかし乗法単位元\(X\)が\(\symcal{F}\)に含まれるかどうかは上の定義からわからない.
したがって集合半環は全体集合を含んでいれば代数学の意味での半環でもあり,
そうでなければ代数学の意味での半環にはあと一歩足りないということになる.

\begin{figure}[h]
    \centering
    \begin{tikzpicture}[draw opacity=1, fill opacity=0.4, text opacity=1]
        \draw[thick] (-2.35, 1.9) rectangle (2.35, -1.9);
        \draw[thick, fill=sRed, even odd rule] (-0.65, 0) circle (1) (0.65, 0) circle (1);
        \node at (0, 1.3) {\(\symdiff{A}{B}\)};
        \node at (-2, 1.5) {\(X\)};
        \node at (-1.0, 0.6) {\(A\)};
        \node at (0.9, 0.6) {\(B\)};
    \end{tikzpicture}
    \caption{\(X\)の部分集合\(A\)と\(B\)の対称差\(\symdiff{A}{B}\).}
    \figlabel{symmetric-difference}
\end{figure}

\begin{example}
\(\symbb{R}\)の左開右閉区間の集合\(\symcal{I} = \{(a, b] \mid a \leq b\}\)は集合半環である.
\examplelabel{interval-semiring}
\end{example}

\begin{proof} \(\emptyset = (a, a] \in \symcal{I}\)である.
\(\mathord{\cap}\)-閉であることは
\begin{align*}
    (a, b] \cap (c, d] =
    \begin{cases}
        \emptyset          & \text{\((a, b] \cap (c, d] = \emptyset\)のとき,} \\
        (a, b]             & \text{\((a, b] \subseteq (c, d]\)のとき,} \\
        (c, d]             & \text{\((a, b] \supseteq (c, d]\)のとき,} \\
        (c, b]             & \text{それ以外のとき}
    \end{cases}
\end{align*}
からわかる.また区分的\(\mathord{\setminus}\)-閉であることは
\begin{align}
    (a, b] \setminus (c, d] =
    \begin{cases}
        (a, b]             & \text{\((a, b] \cap (c, d] = \emptyset\)のとき,} \\
        \emptyset          & \text{\((a, b] \subseteq (c, d]\)のとき,} \\
        (a, c] \cup (d, b] & \text{\((a, b] \supseteq (c, d]\)のとき,} \\
        (a, c] \cup (b, d] & \text{それ以外のとき}
        \eqlabel{setminus-cases}
    \end{cases}
\end{align}
からわかる.
\end{proof}


\end{document}
