\documentclass[../main.tex]{subfiles}
\begin{document}
\mainchapter{位相}

\begin{thmbox}
\begin{definition}
\(X\)を集合とし,\(\symcal{O}\)を\(X\)の部分集合族とする.
\(\symcal{O}\)が以下の\ref{topology-empty}--\ref{topology-union}をみたすとき,
\(\symcal{O}\)の\keyword{位相}(topology)という.
\begin{conditions}
    \item\label{topology-empty} \(\emptyset \in \symcal{O}\)かつ\(X \in \symcal{O}\).
    \item\label{topology-intersection} \(O_1,\,O_2 \in \symcal{O}\)ならば\(O_1 \cap O_2 \in \symcal{O}\).
    \item\label{topology-union} 集合族\({(O_\lambda)}_{\lambda \in \Lambda}\)が各\(\lambda \in \Lambda\)について\(O_\lambda \in \symcal{O}\)をみたすならば
        \begin{align}
            \bigcup_{\lambda \in \Lambda} \in \symcal{O}.
        \end{align}
\end{conditions}
集合\(X\)とその位相\(\symcal{O}\)の組\((X, \symcal{O})\)を\keyword{位相空間}(topological space)という.
\end{definition}
\end{thmbox}

距離空間において,連続性は以下のように定義される:
\begin{thmbox}
\begin{definition}
\((X, d_{X})\), \((Y,\,d_{Y})\)を距離空間とする.
写像\(f\colon X \to Y\)が\keyword{点}\(x \in X\)\keyword{において連続}(continuous at \(x\))であるとは,
任意の\(\varepsilon > 0\)に対して,ある\(\delta > 0\)が存在して,
任意の\(x' \in X\)に対して,\(d_{X}(x', x) < \delta\)ならば\(d_{Y}(f(x'), f(x)) < \varepsilon\)となることをいう.
\definitionlabel{dist-continuity}
\end{definition}
\end{thmbox}

\begin{thmbox}
\begin{definition}
\((X, \symcal{O}_{X})\), \((Y, \symcal{O}_{Y})\)を位相空間とする.
写像\(f\colon X \to Y\)が点\(x \in X\)において連続であるとは,
\(f(x)\)の\((Y, \symcal{O}_{Y})\)における任意の近傍\(V\)に対して,
\((X, \symcal{O}_{X})\)における\(x\)のある近傍\(U\)が存在して,\(f(U) \subseteq V\)となることをいう.
\definitionlabel{topology-continuity}
\end{definition}
\end{thmbox}

\(X\)を集合とし,\(\symcal{O}_1\)と\(\symcal{O}_2\)をそれぞれ\(X\)の位相とする.
\(\symcal{O}_1 \subseteq \symcal{O}_2\)が成り立つとき,
\(\symcal{O}_1\)は\(\symcal{O}_2\)より\keyword{粗い}(coarser)\index{粗い@あらい},
または\(\symcal{O}_2\)は\(\symcal{O}_1\)より\keyword{細かい}(finer)\index{細かい@こまかい}という.
\(X\)の位相全体の集合は関係\(\subseteq\)を順序とみなすことにより,半順序集合となる.

任意の集合\(X\)について,最も粗い(coarsest)位相は密着位相であり,最も細かい(finest)位相は離散位相である.
簡単な例として異なる2点からなる集合\(X = \{a, b\}\)を考える.
密着位相\(\{\emptyset, X\}\)に,任意の1点だけからなる集合を1つだけ含めた集合,
すなわち\(\{\emptyset, \{a\}, X\}\)と\(\{\emptyset, \{b\}, X\}\)もまた位相である.
これらを\keyword{Sierpiński位相}(Sierpiński topology)と呼ぶ.
密着位相\(\{0,\,X\}\),任意のSierpiński位相\(\symcal{S}\),離散位相\(\wp(X) = \{\emptyset,\,\{a\},\,\{b\},\,X\}\)について,\(\{\emptyset,\,X\} \preceq \symcal{S} \preceq \wp(X)\)が成り立つ.これらを象徴的に図示すると\figref{sierpinski}のようになる.
2つの異なるSierpiński位相は比較可能でないことから,位相の細かさは一般的には全順序にならないことがわかる.

\begin{figure}
    \centering
    \vspace{5.2mm}
    \begin{tikzpicture}
        \draw[thick] (0, 0) ellipse (1.2 and 0.6);
        \node[circle, draw=black, fill=black, inner sep=0pt, outer sep=1pt, minimum size=3pt] at (-0.5, 0) {};
        \node[circle, draw=black, fill=black, inner sep=0pt, outer sep=1pt, minimum size=3pt] at (0.5, 0) {};
        \node at (0, -1.3) {密着位相};

        \draw[thick] (4, 0) ellipse (1.2 and 0.6);
        \node[circle, draw=black, fill=black, inner sep=0pt, outer sep=1pt, minimum size=3pt] at (3.5, 0) {};
        \node[circle, draw=black, fill=black, inner sep=0pt, outer sep=1pt, minimum size=3pt] at (4.5, 0) {};
        \draw[thick] (3.5, 0) circle (0.3);
        \node at (4, -1.3) {Sierpiński位相};

        \draw[thick] (8, 0) ellipse (1.2 and 0.6);
        \node[circle, draw=black, fill=black, inner sep=0pt, outer sep=1pt, minimum size=3pt] at (7.5, 0) {};
        \node[circle, draw=black, fill=black, inner sep=0pt, outer sep=1pt, minimum size=3pt] at (8.5, 0) {};
        \draw[thick] (7.5, 0) circle (0.3);
        \draw[thick] (8.5, 0) circle (0.3);
        \node at (8, -1.3) {離散位相};
    \end{tikzpicture}
    \vspace{4mm}
    \caption{各位相を象徴的に表した図.図における点は集合の元,点を囲む円はその元を含む開集合に対応している.}
    \figlabel{sierpinski}
\end{figure}

\begin{thmbox}
\begin{definition}
位相空間\((X, \symcal{O})\)の部分集合\(A\)が\(\overline{A} = X\)をみたすとき,
\(A\)は\(X\)において\keyword{稠密}(dense)であるという.
\end{definition}
\end{thmbox}

稠密であることは次のように言い換えられる.

\begin{thmbox}
\begin{proposition}
位相空間\((X, \symcal{O})\)の部分集合\(A\)が稠密であることは,
任意の\(O \in \symcal{O} \setminus \{\emptyset\}\)について\(A \cap O \neq \emptyset\)が成り立つことと同値である.
\propositionlabel{dense-paraphrase}
\end{proposition}
\end{thmbox}

\begin{proof}(\(\Rightarrow\))任意の\(O \in \symcal{O} \setminus \{\emptyset\}\)をとる.
\(O\)は空でないから,ある\(x \in O\)が存在する.
\(x \in O \subset X\)であり,仮定\(X \subseteq \overline{A}\)より\(x \in \overline{A}\)である.
したがって触点の性質から\(A \cap O \neq \emptyset\)が成り立つ.

\noindent (\(\Leftarrow\))任意の\(x \in X\)をとる.
仮定から\(x\)の任意の開近傍\(V\)について\(A \cap V \neq \emptyset\)が成り立つ.
したがって\(x \in \overline{A}\),すなわち\(X \subseteq \overline{A}\)が示された.
\end{proof}

\end{document}
