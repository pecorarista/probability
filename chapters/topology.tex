\documentclass{ltjsbook}
\usepackage[%
    textwidth=40\zw,
    lines=38,
    centering
]{geometry}
\usepackage{graphicx}
\usepackage{booktabs}
\usepackage[%
    hang,
    flushmargin
]{footmisc}
\usepackage{amssymb}
\usepackage{amsmath}
\usepackage{amsthm}
% \usepackage{mtpro2}
\usepackage{bussproofs}
\usepackage[
    math-style=ISO
]{unicode-math}
\usepackage{customdice}
\usepackage{tikz}
\usepackage{tikz-qtree}
\usepackage{pgfplots}
\usetikzlibrary{%
    patterns,
    intersections,
    calc,
    angles,
    quotes
}

\usepackage{xcolor}
\definecolor{sBlue}{HTML}{0095D9}
\colorlet{sOriginalCurve}{sBlue}

\definecolor{sRed}{HTML}{FF0033}
\colorlet{sApproxCurve}{sRed}

\definecolor{sGreen}{HTML}{138D75}
\colorlet{sSimpleCurve}{sGreen}

\definecolor{sFill}{HTML}{00A381}
\definecolor{sLightGray}{gray}{0.85}

\usepackage{emoji}
\usepackage{framed}

\usepackage[
    % no-math
]{fontspec}
\usepackage[
    deluxe,%
    expert,
    scale=0.95
]{luatexja-preset}

\usepackage{polyglossia}
\setdefaultlanguage{japanese}
\setotherlanguages{english,french,german,russian,hebrew,greek}

\usepackage{covington}
\setglossoptions{%
    fstl=\normalfont\upshape\rmfamily\mcfamily,
    enquotetl=false,
}

\usepackage{luatexja-ruby}
% \setmainfont[%
%     Extension=.otf,
%     UprightFont=*-Book,
%     UprightFont=*-Regular,
%     ItalicFont=*-Italic,
%     BoldFont=*-Bold,
%     BoldItalicFont=*-BoldItalic,
%     SmallCapsFeatures={Numbers=OldStyle},
% ]{NewCM10}
\setmainfont[%
    Extension=.ttf,
    UprightFont=*MT,
    ItalicFont=*-ItalicMT,
    BoldFont=*-BoldMT,
    BoldItalicFont=*-BoldItalicMT,
    Ligatures=Discretionary
]{TimesNewRomanPS}
% \setmainfont{STIX Two Text}
\setsansfont[
    Extension=.ttf,
    UprightFont=*-Regular,
    BoldFont=*-Medium
]{Roboto}
% \setmathfont[
%    math-style=ISO,
%    StylisticSet=1
% ]{STIX Two Math}
\setmathfont{NewCMMath-Book.otf}
\setmathfont[%
    range={%
        cal,
        bfcal
    },
    StylisticSet=1
]{KpMath-Regular.otf}

\ltjnewpreset{book}{%
    mc-m = A-OTF-RyuminPro-Regular.otf,
    % mc-m = NotoSerifJP-Regular.otf,
    % mc-m = HiraMinProN-W3.otf,
    gt-m = BIZUDGothic-Regular.ttf,
    gt-b = BIZUDGothic-Bold.ttf,
    mg-m = MotoyaMaruStd-W5.otf
}
\ltjapplypreset{book}
\setmonojfont{HiraMaruProN-W4}
\ltjsetparameter{%
    jacharrange={%
        -2, % Exclude Greek and Cyrillic letters.
        -3, % Punctuations and miscellaneous symbols.
        -7  % other CJK characters
    },
    alxspmode={`/,allow},
    alxspmode={`#,allow},
    alxspmode={92,allow} % backslash
}

\usepackage{wtref}
\newref{fig}
\setrefstyle{fig}{prefix=図}
\newref{eq}
\setrefstyle{eq}{prefix=式(, suffix=)}
\newref{definition}
\setrefstyle{definition}{prefix=定義}
\newref{axiom}
\setrefstyle{axiom}{prefix=公理}
\newref{proposition}
\setrefstyle{proposition}{prefix=命題}
\newref{lemma}
\setrefstyle{lemma}{prefix=補題}
\newref{example}
\setrefstyle{example}{prefix=例}
\newref{theorem}
\setrefstyle{theorem}{prefix=定理}
\newref{sec}
\setrefstyle{sec}{prefix={第},suffix={節}}

\newtheoremstyle{jplain}%
    {0.6\baselineskip}%
    {0.6\baselineskip}%
    {\normalfont}%
    {}%
    {\bfseries\gtfamily\sffamily}%
    {}%
    {5pt}%
    {\thmname{#1}\thmnumber{#2}\thmnote{#3}\ignorespaces}
\theoremstyle{jplain}
\newtheorem{theorem}{定理}[chapter]
\newtheorem{proposition}[theorem]{命題}
\newtheorem{lemma}[theorem]{補題}
\newtheorem{definition}[theorem]{定義}
\newtheorem{exa}[theorem]{例}
\newtheorem{axiom}[theorem]{公理}
\newtheorem{numberedquote}[theorem]{引用}
\renewcommand{\proofname}{証明}
\renewcommand{\qedsymbol}{\rule{5pt}{10pt}}

\usepackage{tcolorbox}
\tcbuselibrary{%
    skins,
    breakable
}
\newtcolorbox{thmbox}{%
    colback=sLightGray,
    top=8pt,
    bottom=8pt,
    left=8pt,
    right=8pt,
    boxrule=0pt,
    sharp corners,
    frame hidden
}
\newtcolorbox{quotebox}{%
    breakable,
    colback=sLightGray,
    top=8pt,
    bottom=8pt,
    left=8pt,
    right=8pt,
    boxrule=0pt,
    sharp corners,
    frame hidden
}

\makeatletter
\renewenvironment{proof}[1][\proofname]{\par
  \pushQED{\qed}%
  \normalfont \topsep6\p@\@plus6\p@\relax
  \trivlist
  \item[\hskip\labelsep\bfseries\gtfamily#1]\ignorespaces
}{%
  \popQED\endtrivlist\@endpefalse
}
\makeatother
\AtBeginDocument{%
  \setlength{\abovedisplayskip}{5pt}%
  \setlength{\belowdisplayskip}{5pt}%
}

% https://note.com/yuw/n/n38dd54fb2169
\begingroup
\catcode`\,=\active
\def\@x@{\def,{\normalcomma\hskip.2em}}
\expandafter\endgroup\@x@%
\mathcode`\,="8000
\def\normalcomma{\mathchar"613B}

\usepackage{enumitem}
\newlist{conditions}{enumerate}{1}
\setlist[conditions]{label=(\arabic*)}

\newlist{inproofconditions}{enumerate}{1}
\setlist[inproofconditions]{label=(\alph*)}

\usepackage{hyperref}
\hypersetup{%
    colorlinks,
    citecolor=sBlue,
    linkcolor=sBlue,
    urlcolor=sBlue
}

\usepackage[%
    backend=biber,
    style=pecorarista,
    sorting=nyvt,
    urldate=long
]{biblatex}
\addbibresource{references.bib}

\newcommand\mainchapter[1]{\chapter{#1}\thispagestyle{empty}}
\renewcommand{\jsParagraphMark}{}
\renewcommand{\headfont}{\bfseries\gtfamily\sffamily}

\newcommand\tuple[1]{(#1)}
\newcommand\keyword[1]{{\bfseries\gtfamily #1}}
\newcommand\emphchar[1]{\ltjalchar`‹#1\ltjalchar`›}

\newcommand\NaturalNumber{\symbb{N}}
\newcommand\Integer{\symbb{Z}}
\newcommand\PositiveInteger{\symbb{N}_{\mathord{+}}}
\newcommand\Sequence[2]{{\langle {#1}_{#2} \rangle}_{#2 \in \PositiveInteger}}
\newcommand\Subsequence[1]{{\langle {#1}_{\varphi(n)} \rangle}_{n \in \PositiveInteger}}
\newcommand\Sequencen[1]{{\langle {#1}_n \rangle}_{n \in \PositiveInteger}}
\newcommand\SimpleSequencen[1]{{\langle {#1}_n \rangle}_{n}}
\newcommand\Family[1]{{({#1}_\lambda)}_{\lambda \in \Lambda}}
\newcommand\NonNegReal{\symbb{R}_{\mathord{+}}}
\newcommand\PositiveReal{\symbb{R}_{\mathord{+}\mathord{+}}}
\newcommand\Generatedtopology[1]{\tau\left[#1\right]}
\newcommand\Neighborhood[1]{\symcal{N}(#1)}
\newcommand\OpenNeighborhood[1]{\symcal{U}(#1)}

\newcommand\placeholder{\mathord{\cdot}}
\newcommand\powerset[1]{\wp(#1)}
\newcommand\inlinefrac[2]{#1\mathbin{/}#2}
\newcommand\setcomp[1]{{#1}^{\mathsf{c}}}
% https://zrbabbler.hatenablog.com/entry/20120411/1334151482
\newcommand{\relmiddle}[1]{\mathrel{}\middle#1\mathrel{}}

\DeclareMathOperator{\identity}{id}
\DeclareMathOperator{\trace}{trace}
\DeclareMathOperator{\len}{len}
\newcommand\symdiffsymbol{\mathord{\triangle}}
\newcommand\symdiff[2]{#1\mathbin{\triangle}#2}
\newfontfamily\treefont[ItalicFont={Roboto Light Italic}]{Roboto Light}
\newfontfamily\suetterlin{Suetterlin-HJZ-Italic-1911.ttf}
\newcommand\formallang[1]{{\treefont\itshape #1}}

\newcommand\header[1]{\multicolumn{1}{c}{\bfseries\gtfamily #1}}
\newcommand\pronunciation[4]{#1 \texttt{#2} [#3] #4.}
\newcommand\primarystress{\mbox{}\char"02C8}

\newcommand\submin[1]{#1_{\text{min}}}

\newcommand\liaison{\hspace*{0.1em}\raisebox{-0.8ex}{\rotatebox{90}{(}}\hspace*{0.1em}}
\newcommand\shortunderscore{\scalebox{0.7}[1]{\_}}
\newcommand\invbreve[1]{#1{\char"032F}}
% \newcommand\symbb[1]{\mathbb{#1}}
% \newcommand\symcal[1]{\mathcal{#1}}
% \newcommand\symbf[1]{\mathbf{#1}}

\newcommand\iseventuallyin[2]{#1 \mathrel{\text{is eventually in}} #2}
\newcommand\justin[1]{\mathrel{\text{in}} #1}
\newcommand\isfrequentlyin[2]{#1 \mathrel{\text{is frequently in}} #2}
\newcommand\oxfordcomma[3]{#1, #2, \text{and }#3}

\newcommand\proofimplies[2]{\(\text{\ref{#1}} \Rightarrow \text{\ref{#2}}\):}

\renewcommand{\labelitemii}{\(\circ\)}
\renewcommand{\labelitemiii}{\(\diamond\)}

\begin{document}
\mainchapter{位相}
\nocite{uchida}
\nocite{introduction-to-topology}

\section{距離空間}
\begin{thmbox}
\begin{definition}
\(\tuple{X, d_{X}}\)と\(\tuple{Y, d_{Y}}\)を距離空間とする.
写像\(f\colon X \to Y\)が\keyword{点}\(x \in X\)\keyword{において連続}(continuous at \(x\))であるとは,
任意の\(\varepsilon > 0\)に対して,ある\(\delta > 0\)が存在して,
任意の\(x' \in X\)に対して,\(d_{X}(x', x) < \delta\)ならば\(d_{Y}(f(x'), f(x)) < \varepsilon\)となることをいう.
\definitionlabel{dist-continuity}
\end{definition}
\end{thmbox}

\section{位相空間}
\begin{thmbox}
\begin{definition}
\(X\)を集合とし,\(\symcal{O}\)を\(X\)の部分集合族とする.
\(\symcal{O}\)が以下の\ref{topology-empty}--\ref{topology-union}をみたすとき,
\(\symcal{O}\)の\keyword{位相}(topology)という.
\begin{conditions}
    \item\label{topology-empty} \(\emptyset \in \symcal{O}\)かつ\(X \in \symcal{O}\).
    \item\label{topology-intersection} \(O_1, O_2 \in \symcal{O}\)ならば\(O_1 \cap O_2 \in \symcal{O}\).
    \item\label{topology-union} 集合族\({(O_\lambda)}_{\lambda \in \Lambda}\)が各\(\lambda \in \Lambda\)について\(O_\lambda \in \symcal{O}\)をみたすならば
        \begin{align}
            \bigcup_{\lambda \in \Lambda} \in \symcal{O}.
        \end{align}
\end{conditions}
集合\(X\)とその位相\(\symcal{O}\)の組\(\tuple{X, \symcal{O}}\)を\keyword{位相空間}(topological space)という.\definitionlabel{topology}
\end{definition}
\end{thmbox}

\begin{thmbox}
\begin{definition}
\(\symbb{R}^n\)において
\end{definition}
\end{thmbox}

\(X\)を集合とし,\(\symcal{O}_1\)と\(\symcal{O}_2\)をそれぞれ\(X\)の位相とする.
\(\symcal{O}_1 \subseteq \symcal{O}_2\)が成り立つとき,
\(\symcal{O}_1\)は\(\symcal{O}_2\)より\keyword{粗い}(coarser)\index{粗い@あらい},
または\(\symcal{O}_2\)は\(\symcal{O}_1\)より\keyword{細かい}(finer)\index{細かい@こまかい}という.
\(X\)の位相全体の集合は関係\(\subseteq\)を順序とみなすことにより,半順序集合となる.

任意の集合\(X\)について,最も粗い(coarsest)位相は密着位相であり,最も細かい(finest)位相は離散位相である.
簡単な例として異なる2点からなる集合\(X = \{a, b\}\)を考える.
密着位相\(\{\emptyset, X\}\)に,任意の1点だけからなる集合を1つだけ含めた集合,
すなわち\(\{\emptyset, \{a\}, X\}\)と\(\{\emptyset, \{b\}, X\}\)もまた位相である.
これらを\keyword{Sierpiński\footnote{\pronunciation{Sierpiński}{pl}{ɕɛr{\primarystress}pi\~{\j}sk\textsuperscript{j}i}{シェルピンスキー}}位相}(Sierpiński topology)と呼ぶ.
密着位相\(\{0, X\}\),任意のSierpiński位相\(\symcal{S}\),離散位相\(\wp(X) = \{\emptyset, \{a\}, \{b\}, X\}\)について,\(\{\emptyset, X\} \preceq \symcal{S} \preceq \wp(X)\)が成り立つ.これらを象徴的に図示すると\figref{sierpinski}のようになる.
2つの異なるSierpiński位相は比較可能でないことから,位相の細かさは一般的には全順序にならないことがわかる.

\begin{figure}
    \centering
    \vspace{5.2mm}
    \begin{tikzpicture}
        \draw[thick] (0, 0) ellipse (1.2 and 0.6);
        \node[circle, draw=black, fill=black, inner sep=0pt, outer sep=1pt, minimum size=3pt] at (-0.5, 0) {};
        \node[circle, draw=black, fill=black, inner sep=0pt, outer sep=1pt, minimum size=3pt] at (0.5, 0) {};
        \node at (0, -1.3) {密着位相};

        \draw[thick] (4, 0) ellipse (1.2 and 0.6);
        \node[circle, draw=black, fill=black, inner sep=0pt, outer sep=1pt, minimum size=3pt] at (3.5, 0) {};
        \node[circle, draw=black, fill=black, inner sep=0pt, outer sep=1pt, minimum size=3pt] at (4.5, 0) {};
        \draw[thick] (3.5, 0) circle (0.3);
        \node at (4, -1.3) {Sierpiński位相};

        \draw[thick] (8, 0) ellipse (1.2 and 0.6);
        \node[circle, draw=black, fill=black, inner sep=0pt, outer sep=1pt, minimum size=3pt] at (7.5, 0) {};
        \node[circle, draw=black, fill=black, inner sep=0pt, outer sep=1pt, minimum size=3pt] at (8.5, 0) {};
        \draw[thick] (7.5, 0) circle (0.3);
        \draw[thick] (8.5, 0) circle (0.3);
        \node at (8, -1.3) {離散位相};
    \end{tikzpicture}
    \vspace{4mm}
    \caption{各位相を象徴的に表した図.図における点は集合の元,点を囲む円はその元を含む開集合に対応している.}
    \figlabel{sierpinski}
\end{figure}

\begin{thmbox}
\begin{definition}
\((X, \symcal{O})\)を位相空間とする.\(\symcal{O}\)の部分集合\(\symcal{B}\)について,
任意の\(O \in \symcal{O}\)に対して,ある\(\symcal{B}_0 \subseteq \symcal{B}\)が存在して,
\begin{align*}
    O = \bigcup \symcal{B}_0
\end{align*}
とできるとき,
\(\symcal{B}\)を\(\symcal{O}\)の\keyword{開基}(basis)という.
\end{definition}
\end{thmbox}

\begin{example} \(\symbb{R}\)において開区間全体の集合\(\symcal{I} = \{ (a, b) \mid a,  b \in \symbb{R}\}\)は開基である.
\end{example}

% Singh 1.4.5
\begin{thmbox}
\begin{theorem}
\(X\)を集合とする.部分集合族\(\symcal{B} \subseteq \wp(X)\)が
以下をみたすとき,
\begin{conditions}
    \item \(X = \bigcup \symcal{B}\).
    \item 任意の\(B_1, B_2 \in \symcal{B}\)と\(x \in B_1 \cap B_2\)に対して,
        ある\(B_3 \in \symcal{B}\)が存在して,\(B_1 \cap B_2\)
\end{conditions}
\end{theorem}
\end{thmbox}

\section{直積集合と位相}
位相空間\(\tuple{X, \symcal{O}_X}\)と \(\tuple{Y, \symcal{O}_Y}\)が与えられたとき,
直積\(X \times Y\)に位相を定める方法はいくつかある.
その1つは
\begin{align*}
    \symcal{R} = \{ U \times V \mid U \in \symcal{O}_X, V \in \symcal{O}_Y\}
\end{align*}
が生成する位相\(\Box := \Generatedtopology{\symcal{R}}\)を\(X \times Y\)の位相とする方法である.
\(\symcal{R}\)自身は一般的には位相にならない.
例えば\(X = Y = \symbb{R}\)において,
\(X\)の開区間\(U_1, U_2\)と
\(Y\)の開区間\(V_1, V_2\)を
\figref{box-topology}のような位置関係になるようにとる.
このとき\(U_1 \times V_1\)と\(U_2 \times V_2\)は\(\symcal{R}\)の元であるが,
その和集合\((U_1 \times V_1) \cup (U_2 \times V_2)\)は\(\symcal{R}\)の元ではない.

\begin{figure}
    \centering
    \begin{tikzpicture}
        \begin{axis}[
            axis lines=middle,
            axis line style=thick,
            xlabel={$x$},
            ylabel={$y$},
            xtick=\empty,
            ytick=\empty,
            xmin=-0.5,
            xmax=5,
            ymin=-0.5,
            ymax=3,
            samples=200,
            clip=false
        ]
        % vertical lines
        \addplot[dotted, sBlue] coordinates {(0.5, -0.25) (0.5, 2.5)};
        \addplot[dotted, sRed] coordinates {(1.5, -0.25) (1.5, 2.5)};
        \addplot[dotted, sBlue] coordinates {(2, -0.25) (2, 3)};
        \addplot[dotted, sRed] coordinates {(3, -0.25) (3, 3)};
        \addplot[ultra thick, sBlue] coordinates {(0.03, 0.5) (0.03, 1.5)};
        \addplot[ultra thick, sRed] coordinates {(-0.03, 1) (-0.03, 2)};
        % horizontal lines
        \addplot[dotted, sBlue] coordinates {(-0.25, 0.5) (3.5, 0.5)};
        \addplot[dotted, sRed] coordinates {(-0.25, 1) (3.5, 1)};
        \addplot[dotted, sBlue] coordinates {(-0.25, 1.5) (3.5, 1.5)};
        \addplot[dotted, sRed] coordinates {(-0.25, 2) (3.5, 2)};
        \addplot[ultra thick, sBlue] coordinates {(0.5, 0.03) (2, 0.03)};
        \addplot[ultra thick, sRed] coordinates {(1.5, -0.03) (3, -0.03)};

        \node at (axis cs:-0.2, -0.2) {$O$};
        \node at (axis cs:1.25, -0.2) {\textcolor{sBlue}{$U_1$}};
        \node at (axis cs:2.25, -0.2) {\textcolor{sRed}{$U_2$}};
        \node at (axis cs:0.3, 1) {\textcolor{sBlue}{$V_1$}};
        \node at (axis cs:-0.2, 1.5) {\textcolor{sRed}{$V_2$}};
        \node at (axis cs:0.8, 1.7) {\textcolor{sBlue}{$U_1 \times V_1$}};
        \node at (axis cs:2.25, 2.2) {\textcolor{sRed}{$U_2 \times V_2$}};
        \node[anchor=west] at (axis cs:3.2, 1.25) {$(U_1 \times V_1) \cup (U_2 \times V_2)$};

        \fill[pattern=north east lines, pattern color=sBlue] (axis cs:0.5, 1.5) rectangle (axis cs:2, 0.5);
        \fill[pattern=north west lines, pattern color=sRed] (axis cs:1.5, 2) rectangle (axis cs:3, 1);

        \addplot[thick] coordinates {%
            (0.5, 1.5)
            (0.5, 0.5)
            (2, 0.5)
            (2, 1)
            (3, 1)
            (3, 2)
            (1.5, 2)
            (1.5, 1.5)
        } -- cycle;

        \end{axis}
    \end{tikzpicture}
    \caption{\(\symbb{R} \times \symbb{R}\)の箱位相の元の例.}\figlabel{box-topology}
\end{figure}

一般に位相空間の族\(({(X_\lambda, \symcal{O}_\lambda))}_{\lambda \in \Lambda}\)が与えられたとき,
\begin{align*}
    \symcal{R} = \left\{
        \prod_{\lambda \in \Lambda} O_\lambda
        \relmiddle{|}
        \text{任意の\(\lambda \in \Lambda\)について\(O_\lambda \in \symcal{O}_\lambda\)}
        \right\}
\end{align*}
が生成する位相\(\Box := \Generatedtopology{\symcal{R}}\)を\({((X_\lambda, \symcal{O}_\lambda))}_{\lambda \in \Lambda}\)の\keyword{箱位相}(box topology)という.

\begin{thmbox}
\begin{proposition}
\(\symbb{R} \times \symbb{R}\)の箱位相\(\Box\)は\(\symbb{R}^2\)の通常の位相\(\symcal{O}\)と一致する.
\end{proposition}
\end{thmbox}

\begin{proof} (\(\Box \subseteq \symcal{O}\))任意の\(O \in \Box\)をとる.
\end{proof}

\begin{thmbox}
\begin{definition}
\(\tuple{X, \symcal{O}}\)を位相空間とする.
\(X\)の部分集合\(N\)が点\(x \in X\)の\keyword{近傍}(neighborhood)であるとは\(x\)が\(N\)の内点となることをいう.すなわち開集合\(U\)で\(x \in U\)かつ\(U \subseteq N\)をみたすものが存在することをいう.
特に\(N\)自身が開集合であるとき,\(N\)は\keyword{開近傍}(open neighborhood)であるという.
\end{definition}
\end{thmbox}

\begin{exa}
ユークリッド空間\(\symbb{R}^2\)の点\(x\)をとる.
\(r\)を任意の正の実数として\(N = [x - r, x + r]^2\)と定める.
このとき例えば開集合\(U = B(x, \inlinefrac{r}{2})\)をとると\(x \in U\)かつ\(U \subseteq N\)が成り立つので,\(N\)は\(x\)の近傍である(\figref{neighborhood}).\examplelabel{neighborhood}
\end{exa}

\begin{figure}
    \centering
    \begin{tikzpicture}
        \tikzset{%
            Point/.style={%
                circle,
                draw=black,
                fill=black,
                inner sep=0pt,
                outer sep=1pt,
                minimum size=3pt
            }
        }
        \node[Point] at (0, 0) {};
        \node at (0.2, 0.2) {\(x\)};
        \draw[thick, dotted] (0, 0) circle (0.7);
        \draw[thick] (-1.4, 1.4) rectangle (1.4, -1.4);
        \node at (0.8, 0.8) {\(U\)};
        \node at (1.8, 1.2) {\(N\)};
    \end{tikzpicture}
    \caption{\exampleref{neighborhood}を図示したもの.}\figlabel{neighborhood}
\end{figure}

\begin{thmbox}
\begin{proposition}
\(\tuple{X, \symcal{O}}\)を位相空間とする.
点\(x \in X\)の近傍全体の集合\(\Neighborhood{x}\)について以下が成り立つ.
\begin{conditions}
    \item\label{neighborhood-cap} 任意の\(N_1, N_2 \in \Neighborhood{x}\)について\(N_1 \cap N_2 \in \Neighborhood{x}\).
\end{conditions}
\end{proposition}
\propositionlabel{neighborhood-properties}
\end{thmbox}

\begin{proof} (1) \(N_1 \in \Neighborhood{x}\)より,ある\(U_1 \in \symcal{O}\)が存在して,
\(x \in U_2 \subseteq N_1\)が成り立つ.
同様に\(N_2 \in \Neighborhood{x}\)より,ある\(U_2 \in \symcal{O}\)が存在して,
\(x \in U_2 \subseteq N_2\)が成り立つ.
位相の定義から\(U_1 \cap U_2 \in \symcal{O}\)であり,
これは\(x \in U_1 \cap U_2 \subseteq N_1 \cap N_2\)をみたすので,
\(N_1 \cap N_2 \in \Neighborhood{x}\)が成り立つ.
\end{proof}

\begin{thmbox}
\begin{definition}
位相空間\(\tuple{X, \symcal{O}}\)の部分集合\(A\)が\(\overline{A} = X\)をみたすとき,
\(A\)は\(X\)において\keyword{稠密}(dense)であるという.
\end{definition}
\end{thmbox}

稠密であることは次のように言い換えられる.

\begin{thmbox}
\begin{proposition}
位相空間\(\tuple{X, \symcal{O}}\)の部分集合\(A\)が稠密であることは,
任意の\(O \in \symcal{O} \setminus \{\emptyset\}\)について\(A \cap O \neq \emptyset\)が成り立つことと同値である.
\propositionlabel{dense-paraphrase}
\end{proposition}
\end{thmbox}

\begin{proof}(\(\Rightarrow\))任意の\(O \in \symcal{O} \setminus \{\emptyset\}\)をとる.
\(O\)は空でないから,ある\(x \in O\)が存在する.
\(x \in O \subset X\)であり,仮定\(X \subseteq \overline{A}\)より\(x \in \overline{A}\)である.
したがって触点の性質から\(A \cap O \neq \emptyset\)が成り立つ.

\noindent (\(\Leftarrow\))任意の\(x \in X\)をとる.
仮定から\(x\)の任意の開近傍\(V\)について\(A \cap V \neq \emptyset\)が成り立つ.
したがって\(x \in \overline{A}\),すなわち\(X \subseteq \overline{A}\)が示された.
\end{proof}

\begin{thmbox}
\begin{definition}
\({\tuple{\tuple{X_\lambda, \symcal{O}_\lambda}}}_{\lambda \in \Lambda}\)を位相空間の族とし,
\(X = \prod_{\lambda \in \Lambda} X_\lambda\)とする.
\(\Lambda\)の任意の有限部分集合全体の集合を\(\symcal{I}\)とする.任意の\(I \in \symcal{I}\)について
\begin{align*}
    \Pi_I = \left\{ f\colon \Lambda \to \bigcup_{\lambda \in \Lambda} X_\lambda \relmiddle{|} \text{\parbox{7cm}{\(f(i) \in \symcal{O}_i\) for all \(i \in I\) and \(f(\lambda) \in X_\lambda\) for all \(\lambda \in \Lambda \setminus I\)}}\right\}
    % \Pi_I = \left\{ f\colon \Lambda \to \bigcup_{\lambda \in \Lambda} X_\lambda \relmiddle{|} \text{各\(i \in I\)について\(\symcal{O}_i\)かつ\(\lambda \in \Lambda \setminus I\)について\(f(\lambda) \in X_\lambda\)}\right\}
\end{align*}
と定義する.このとき\(\bigcup_{I \in \symcal{I}} \Pi_I\)が生成する位相\(\symcal{O}:=\tau[\bigcup_{I \in \symcal{I}}\Pi_I]\)を\(X\)の\keyword{直積位相}(product topology)という.
\end{definition}
\end{thmbox}

直積\(X = \prod_{\lambda \in \Lambda} X_\lambda\)について,その位相としては特に断らない限り上で定義した\(\symcal{O}\)をとる.組\(\tuple{X, O}\)を\({\tuple{\tuple{X_\lambda, \symcal{O}_\lambda}}}_{\lambda \in \Lambda}\)の\keyword{直積位相空間}(product topological space)と呼ぶ.

\section{連続写像}
\begin{thmbox}
\begin{definition}
\(\tuple{X, \symcal{O}_{X}}\), \(\tuple{Y, \symcal{O}_{Y}}\)を位相空間とする.
写像\(f\colon X \to Y\)が点\(x \in X\)において連続であるとは,
\(f(x)\)の\(\tuple{Y, \symcal{O}_{Y}}\)における任意の近傍\(V\)に対して,
\(\tuple{X, \symcal{O}_{X}}\)における\(x\)のある近傍\(U\)が存在して,\(f(U) \subseteq V\)となることをいう.
\definitionlabel{topology-continuity}
\end{definition}
\end{thmbox}

\section{コンパクト性}
\begin{thmbox}
\begin{definition}
    位相空間\(\tuple{X, \symcal{O}}\)を位相空間とする.\(X\)の部分集合族\(\symcal{U}\)が
\begin{align}
    X = \bigcup \symcal{U}
    \eqlabel{cover}
\end{align}
をみたすとき,\(\symcal{U}\)は\(X\)を\keyword{覆う}(\(\symcal{U}\) covers \(X\))\index{覆う@おおう},あるいは
\(\symcal{U}\)は\(X\)の\keyword{被覆}(cover)\index{被覆@ひふく}であるという.
\eqref{cover}が成り立ち,かつ\(\symcal{U}\)の元がすべて開集合であるとき,
すなわち\(\symcal{U} \subseteq \symcal{O}\)であるとき,
\(\symcal{U}\)は\(X\)の\keyword{開被覆}(open cover)\index{かいひふく@開被覆}であるという.
開被覆の部分集合\(\symcal{U}'\)が有限個の要素からなり,かつ\(X\)を覆うとき,
\(\symcal{U}'\)は\keyword{有限部分被覆}(finite subcover)\index{有限部分被覆@ゆうげんぶぶんひふく}であるという.

位相空間\(\tuple{X, \symcal{O}}\)の任意の開被覆\(\symcal{U}\)について,有限部分被覆\(\symcal{U}' \subseteq \symcal{U}\)が存在するとき,\(\tuple{X, \symcal{O}}\)は\keyword{コンパクト}(compact)である,あるいは\keyword{コンパクト空間}(compact space)であるという.
\end{definition}
\end{thmbox}

\begin{exa} \(X\)を有限個の要素をもつ集合とする.
このとき離散空間\(\tuple{X, \wp(X)}\)はコンパクトである.
なぜならば,任意の開被覆\(\symcal{U}\)をとると,
その要素の個数は\(\lvert\symcal{U}\rvert \leq \lvert \wp(X)\rvert \leq 2^{\lvert X \rvert}\)となるので,
\(\symcal{U}\)自身が有限部分被覆になっているからである.

一方で,\(X\)が無限個の要素をもつ場合,離散空間\(\tuple{X, \wp(X)}\)はコンパクトにはならない.なぜならば,開被覆として\(\symcal{U} = \{\{x\} \mid x \in X\}\)をとると,その有限個の要素からなる部分集合で\(X\)を覆うことはできないからである.
\end{exa}


\begin{thmbox}
\begin{theorem}
\(\tuple{X, \symcal{O}_X}, \tuple{Y, \symcal{O}_Y}\)を位相空間,\(f\colon X \to Y\)を連続写像とする.
\(A \subseteq X\)がコンパクトならば像\(f(A) \subseteq Y\)もまたコンパクトである.
\end{theorem}
\end{thmbox}

\begin{proof}
\(\symcal{V} \subseteq \symcal{O}_Y\)を\(f(A)\)の任意の開被覆とする.
\propositionref{f-set}~\ref{f-inv-cup}, \ref{f-inv-increase}, \ref{f-inv-f-a}より
\begin{align*}
    \bigcup_{V \in \symcal{V}} f^{-1}(V)
    = f^{-1} \left(\bigcup  \symcal{V}\right)
    \supseteq f^{-1}(f(A))
    \supseteq A
\end{align*}
となるので,\(\symcal{U} = \{f^{-1}(V) \mid V \in \symcal{V}\}\)は\(A\)の開被覆である.
\(A\)がコンパクトなので\(\{f^{-1}(V_1), \ldots, f^{-1}(V_n)\} \subseteq \symcal{U}\)が存在して
\begin{align*}
    A \subseteq \bigcup_{i = 1}^n f^{-1}(V_i)
\end{align*}
が成り立つ.したがって\propositionref{f-set}~\ref{f-increase}, \ref{f-cup}, \ref{f-f-inv-a}より
\begin{align*}
    f(A)
    \subseteq f \left(\bigcup_{i = 1}^n f^{-1}(V_i)\right)
    \subseteq \bigcup_{i = 1}^n f (f^{-1}(V_i))
    = \bigcup_{i = 1}^n V_i
\end{align*}
となり,\(f(A)\)もまたコンパクトである.
\end{proof}

\section{ネット}
\begin{thmbox}
\begin{definition}
空でない集合\(\Lambda\)上と,以下の\ref{directed-set-reflex}--\ref{directed-set-upper}を満たす
関係\(\preceq\)の組\(\tuple{\Lambda, \preceq}\)を\keyword{有向集合}(directed set)という.
関係\(\preceq\)が明らかな場合は\(\tuple{\Lambda, \preceq}\)を単に\(\Lambda\)と書く.
\begin{conditions}
    \item\label{directed-set-reflex} (反射律)任意の\(\lambda \in \Lambda\)について\(\lambda \preceq \lambda\).
    \item\label{directed-set-trans} (推移律)任意の\(\alpha, \beta, \gamma \in \Lambda\)について\(\alpha \preceq \beta\)かつ\(\beta \preceq \gamma\)ならば\(\alpha \preceq \gamma\).
    \item\label{directed-set-upper} 任意の\(\alpha, \beta \in \Lambda\)に対して\(\alpha \preceq \gamma\)かつ\(\beta \preceq \gamma\)をみたす\(\gamma \in \Lambda\)が存在する.
\end{conditions}
\definitionlabel{directed-set}
\end{definition}
\end{thmbox}

上の定義の\ref{directed-set-reflex}と\ref{directed-set-trans}は順序の公理から反対称律を除いたものである.
この\(2\)つを満たす関係を\keyword{前順序}(preorder)と呼ぶ\footnote{%
擬順序(quasiorder)と呼ぶこともある.
「前順序」という用語を「ぜんじゅんじょ」と読んでしまうと全順序と紛らわしいので,英語のままpreorderと言ったほうがよいかもしれない.}.

\begin{exa} \(\symbb{N}\), \(\symbb{Z}\), \(\symbb{Q}\), \(\symbb{R}\)はそれぞれ有向集合である.
\end{exa}

\begin{exa} 位相空間\(X\)の任意の点\(x\)をとる.
集合\(\Neighborhood{x}\)に関係\(\mathord{\preceq}\)を\(N \supseteq N' \Leftrightarrow N' \preceq N\)で定める.
これが反射律と推移律をみたすことは明らかである.
\definitionref{directed-set-uppper}~\ref{directed-set-upper}については,
\(x\)の任意の近傍\(N_1, N_2\)について,
\(N_1 \cap N_2\)をとると,これもまた\(x\)の近傍であり(\propositionref{neighborhood-properties}の\ref{neighborhood-cap}),\(N_1 \cap N_2 \subseteq N_1\)かつ\(N_1 \cap N_2 \subseteq N_2\)をみたすことからわかる.\examplelabel{neighborhood-directed-set}
\end{exa}

\begin{thmbox}
\begin{definition}
有向集合\(\Lambda\)から集合\(X\)への写像\(f \colon \Lambda \to X\)を\(X\)の\keyword{ネット}(net in \(X\))という.
\end{definition}
\end{thmbox}

ネットはしばしば列や族と同様に\(\Familyl{x}\)のように書かれる.
以降,厳密に使い分けることはせず,\(f, g\)のような表記と\(\Familyl{x}\)のような表記を併せて用いる.

\begin{exa}
Riemann\footnote{\pronunciation{Riemann}{de}{ˈʀiːman}{リーマン}}積分の定義に用いられる\keyword{分割}(partition)\index{ぶんかつ@分割 partition}は有向集合の例である.
分割とは,区間\(I = [a, b]\)が与えられたとき,\(a = x_0 < x_1 < \cdots < x_n = b\)を満たすような点の集合\(\{x_0, \ldots, x_n\}\)のことをいう.
2つの分割\(\Delta = \{x_0, \ldots, x_n\}\)と\(\Delta' = \{y_0, \ldots, y_m\}\)について\(\Delta \subseteq \Delta' \)が成り立つとき,\(\Delta'\)は\(\Delta\)の\keyword{細分}(refinement)\index{さいぶん@細分 refinement}であるという.
\(\Delta'\)が\(\Delta\)の細分であるとき関係\(\Delta \preceq \Delta'\)が成り立つと定める.このとき\(I\)の分割全体の集合は有向集合である.有向集合の定義\ref{directed-set-reflex}と\ref{directed-set-trans}が成り立つのは明らかである.
\ref{directed-set-upper}については,分割\(\Delta\)と\(\Delta'\)が与えられたとき,\(\Delta \cup \Delta'\)の元を用いた分割\(\Delta^*\)について\(\Delta \preceq \Delta^*\)と\(\Delta' \preceq \Delta^*\)となることからわかる.

閉区間\(I\)上で定義される実数値関数\(f\)について,\(I\)の分割\(\Delta = (x_0, \ldots, x_n)\)による\keyword{過剰和}(upper sum)と\keyword{不足和}(lower sum)(\figref{partition-of-interval})を以下で定義する:
\begin{gather}
    U(f, \Delta) := \sum_{i = 1}^n \sup \{ f(x) \mid x \in [x_{i - 1}, x_i] \} (x_i - x_{i - 1}), \\
    L(f, \Delta) := \sum_{i = 1}^n \inf \{ f(x) \mid x \in [x_{i - 1}, x_i] \} (x_i - x_{i - 1}).
\end{gather}
このとき\(\Delta \mapsto U(f, \Delta)\), \(\Delta \mapsto L(f, \Delta)\)はそれぞれネットになっている.
\end{exa}

\pgfmathdeclarefunction{curve}{1}{%
    \pgfmathparse{1.7 * sin(0.72 * deg(#1)) + 2.2}%
}

\begin{figure}
    \centering
    \begin{tikzpicture}
        \pgfdeclarepatternformonly{north east lines wide}
           {\pgfqpoint{-1pt}{-1pt}}
           {\pgfqpoint{10pt}{10pt}}
           {\pgfqpoint{9pt}{9pt}}
           {%
             \pgfsetlinewidth{0.4pt}
             \pgfpathmoveto{\pgfqpoint{0pt}{0pt}}
             \pgfpathlineto{\pgfqpoint{9.1pt}{9.1pt}}
             \pgfusepath{stroke}
           }
        \begin{axis}[
            domain=-0.5:8,
            samples=200,
            axis lines=middle,
            xlabel={\(x\)},
            ylabel={\(y\)},
            xmin=-0.8,
            xmax=8.5,
            ymin=-2.0,
            ymax=5,
            xtick=\empty,
            xticklabels=\empty,
            ytick=\empty,
            yticklabels=\empty,
            width=15cm,
            height=7cm,
        ]
        \node at (axis cs: -0.25, -0.4) {\(O\)};

        \pgfplotsinvokeforeach{0, ..., 4}{%
            \node at (axis cs: {#1 + 1}, -0.1) [anchor=north] {\(x_#1\)};
            \node at (axis cs: {#1 + 1}, -0.6) [anchor=north] {\rotatebox[origin=c]{90}{\(=\)}};
        }
        \pgfplotsinvokeforeach{2, ..., 5}{%
            \node at (axis cs: #1, -1.2) [anchor=north] {\(y_#1\)};
        }

        \node at (axis cs: 1, -1.2) [anchor=north] {\(y_0\)};
        \node at (axis cs: 1.5, -1.2) [anchor=north] {\(y_1\)};

        \node at (axis cs: 5.5, -0.6) [anchor=west] {\(\Delta = \{x_0, \ldots, x_4\}\)};
        \node at (axis cs: 5.5, -1.4) [anchor=west] {\(\Delta' = \{y_0, \ldots, y_5\}\)};

        \addplot[sBlue, thick] {curve(x)};
        \node at (axis cs: 5.7, 1.8)  {\(y = f(x)\)};

        \pgfmathsetmacro\dx{1};
        \foreach \k in {1, ..., 4} {%
            \pgfmathsetmacro\xleft{\k * \dx};
            \pgfmathsetmacro\xright{\k * \dx + \dx};
            \pgfmathsetmacro\yleft{curve(\xleft)};
            \pgfmathsetmacro\yright{curve(\xright)};
            \pgfmathsetmacro\yminimum{min(\yleft, \yright)};
            \addplot [
                pattern=north east lines wide,
                pattern color=sRed,
                draw,
            ] coordinates {%
                (\xleft, 0)
                (\xleft, \yminimum)
                (\xright, \yminimum)
                (\xright, 0)
                (\xleft, 0)
            };
        }
        \addplot [
            pattern=north east lines wide,
            pattern color=sRed,
            draw,
        ] coordinates {%
            (1.5, 0)
            (1.5, {curve(1.5)})
            (2, {curve(1.5)})
            (2, 0)
            (1.5, 0)
        };
        \draw[dashed] (axis cs: 1.5, -0) -- (axis cs: 1.5, -1.2);
        \end{axis}
    \end{tikzpicture}
    \caption{分割\(\Delta\)とその細分\(\Delta'\),そしてそれらの作る不足和の例.}\figlabel{partition-of-interval}
\end{figure}

\begin{thmbox}
\begin{definition}
ネット\(f\colon \Lambda \to X\)について,ある\(\lambda_0 \in \Lambda\)が存在して,任意の\(\lambda \succeq \lambda_0\)について\(f(\lambda) \in Y\)が成り立つとき,\(f\)は\keyword{最終的に\(Y\)に入る}\((f\) is eventually in \(Y)\)という.
一般的な表記ではないが以降はこのことを\(\iseventuallyin{f}{Y}\)と書く.
点\(x \in X\)の任意の開近傍\(U\)について\(\iseventuallyin{f}{U}\)が成り立つとき,
\keyword{\(\Familyl{x}\)は\(x\)に収束する}(\(\Familyl{x}\) converges to \(x\))といい,\(x_\lambda \to x\)と書く.
この\(x\)を\(\Familyl{x}\)の\keyword{極限}(limit point)という.
\end{definition}
\end{thmbox}

一般の位相空間において,ネットは複数の極限をもつことがある.
たとえば\(X = \{0, 1\}\)に密着位相\(\symcal{O} = \{\emptyset, X\}\)を入れた\(\tuple{X, \symcal{O}}\)を考える.
このとき任意の\(f\colon \PositiveInteger \to \{0, 1\}\)は\(0\)と\(1\)に収束する.
したがって極限が一意に定まるためには位相に条件を付ける必要がある.
それが次のHausdorff空間\footnote{\pronunciation{Hausdorff}{de}{ˈha\invbreve{ʊ}sdɔʀf}{ハウスドルフ}}である.

\begin{thmbox}
\begin{definition}
\(X\)を位相空間とする.
\(X\)の任意の異なる\(2\)点\(x, y\)に対し,
\(x\)の近傍\(U\)と\(y\)の近傍\(V\)で\(U \cap V = \emptyset\)をみたすものがとれるとき,
\(X\)を\keyword{Hausdorff空間}(Hausdorff space)という.
\end{definition}
\end{thmbox}

\figref{hausdorff-space}に図示したように,この空間は\(2\)点を分離できるような空間と捉えることができる.

\begin{figure}
    \centering
    \begin{tikzpicture}
        \tikzset{%
            Point/.style={%
                circle,
                draw=black,
                fill=black,
                inner sep=0pt,
                outer sep=1pt,
                minimum size=3pt
            }
        }
        \node[Point] at (1.6, 0) {};
        \node at (1.6, 0.5) {\(x\)};
        \node at (0.2, 1.1) {\(U\)};
        \node[Point] at (2.8, 0) {};
        \node at (2.8, 0.5) {\(y\)};
        \node at (4, 1.3) {\(V\)};
        \draw[thick, dotted] (1.1, 0) ellipse (1cm and 1cm);
        \draw[thick, dotted] (3.3, 0) ellipse (1cm and 1.1cm);
    \end{tikzpicture}
    \caption{Hausdorff空間.}\figlabel{hausdorff-space}
\end{figure}

\begin{thmbox}
\begin{proposition}
Hausdorff空間においてネットが収束するならば,その極限はただ\(1\)つである.
\end{proposition}
\end{thmbox}

\begin{proof} \(X\)をHausdorff空間とする.
任意の収束するネット\(\Familyl{x}\)をとり,その極限を\(x\)とする.
また\(\Familyl{x}\)が\(x\)とは異なる点\(y\)にも収束すると仮定する.
\(X\)がHausdorff空間なので\(x\)の開近傍\(U\)と\(y\)の開近傍\(V\)で\(U \cap V = \emptyset\)であるようなものをとることができる.\(x_\lambda \to x\)より,ある\(\lambda_0 \in \Lambda\)が存在して,
任意の\(\lambda \succeq \lambda_0\)で\(x_\lambda \in U\)が成り立つ.
また\(x_\lambda \to y\)より,ある\(\lambda_0' \in \Lambda\)が存在して,
任意の\(\lambda \succeq \lambda_0\)で\(x_\lambda \in V\)が成り立つ.
したがって\(\lambda \succeq \max \{\lambda_0, \lambda_0'\}\)ならば
\(x_\lambda \in U \cap V\)となるが,これは\(U \cap V = \emptyset\)であったことと矛盾する.
\end{proof}

\begin{thmbox}
\begin{definition}
\(X\)を位相空間,\(Y\)を\(X\)の部分集合とする.
ネット\(f \colon \Lambda \to X\)は,どのような\(\lambda_0 \in \Lambda\)に対しても
\(f(\lambda) \in Y\)となるような\(\lambda \succeq \lambda_0\)が存在するとき,
\keyword{頻繁に\(Y\)に入る}\((f\) is frequently in \(Y)\))という.
一般的な表記ではないが以降はこのことを\(\isfrequentlyin{f}{Y}\)と書く.
ネット\(f\)が点\(x \in X\)の任意の近傍\(U\)について\(\isfrequentlyin{f}{U}\)をみたすとき,
\(x\)を\(f\)の\keyword{集積点}(cluster point)という.
\end{definition}
\end{thmbox}

\begin{exa}
ネット\(f\colon \PositiveInteger \to \symbb{R}\)を\(x_n = n + (-1)^n n\)と定義する.
このとき\(0\)は\(f\)の集積点である.
\end{exa}


列\(\Sequencen{x}\)の部分列(subsequence)は,狭義単調増加関数\(\varphi\colon\PositiveInteger\to\PositiveInteger\)を用いて\(\Subsequence{x}\)と表されるような列のことである.
あるいは列\(f\)の部分列は合成写像\(f\circ \varphi\)であると言い換えてもよい.
ネットにおいて部分列に相当するものはサブネットと呼ばれ,以下のように定義される.

\begin{thmbox}
\begin{definition}[(サブネット)] ネット\(g \colon \Xi \to X\)がネット\(f \colon \Lambda \to X\)の\keyword{サブネット}(subnet)であるとは,写像\(\varphi \colon \Xi \to \Lambda\)が存在して,次の2つをみたすことをいう.
    \begin{conditions}
        \item\label{subnet-comp} \(g = f \circ \varphi\).
        \item\label{subnet-exists-beta} (共終性)任意の\(\kappa \in \Lambda\)に対して,ある\(\xi_0 \in \Xi\)が存在して,\(\xi \succeq \xi_0\)ならば\(\kappa \preceq \varphi(\xi)\)が成り立つ.
    \end{conditions}\definitionlabel{subnet}
\end{definition}
\end{thmbox}

一般に前順序集合\(\tuple{\Lambda, \mathord{\preceq}}\)の部分集合\(A \subseteq \Lambda\)が\keyword{共終}(cofinal)であるとは,任意の\(\lambda \in \Lambda\)に対してある\(\alpha \in A\)が存在して\(\lambda \preceq \alpha\)となることをいう.また関数\(\varphi \colon \Xi \to \Lambda\)が共終であるとは像\(\varphi(\Xi) \subseteq \Lambda\)が共終であることをいう.したがって\ref{subnet-exists-beta}は\(\varphi(\{ \xi \in \Xi \mid \xi \succeq \xi_0\})\)が共終であると言い換えることができる.
以下では写像\(\varphi\)であるといった場合,\definitionref{subnet}の\ref{subnet-exists-beta}が成り立つことを意味するものとする.

列\(\Sequencen{x}\)をネットとみなしたときの部分列\(\Subsequence{x}\)はサブネットである.
このことは以下のように確認できる.
\(\varphi\)の単調性から,任意の\(n \in \PositiveInteger\)に対して,\(n' \geq n\)ならば\(\varphi(n) \leq \varphi(n')\)が成り立つ.
また\propositionref{n-leq-phi-of-n}より\(n \leq \varphi(n)\)が成り立つので,\(\varphi \{ n \in \PositiveInteger \mid n' \geq n\}\)は共終であり,\(\Subsequence{x}\)がサブネットであることが確認できた.

\begin{thmbox}
\begin{proposition} \(\tuple{N, \mathord{\leq}}\)を整列集合,\(\varphi\colon N \to N\)を狭義単調増加関数とする.このとき任意の\(n \in N\)について\(n \leq \varphi(n)\)が成り立つ.\propositionlabel{n-leq-phi-of-n}
\end{proposition}
\end{thmbox}

\begin{proof} \(n_0 > \varphi(n_0)\)となる\(n_0 \in N\)が存在すると仮定する.
このとき\(S = \{ n \in N \mid n > \varphi(n)\}\)とすると,\(S\)は\(N\)の空でない部分集合であるから,最小元\(\submin{n}\)をもつ.
すると\(\submin{n} > \varphi(\submin{n})\)であり,単調性から\(\varphi(\varphi(\submin{n})) < \varphi(\submin{n}) < \submin{n}\)であるが,これは\(\submin{n}\)の最小性に反する.
\end{proof}

\begin{thmbox}
\begin{definition}
\(S\)を\(X\)の任意の部分集合とする.
\(X\)のネット\(f\)が最終的に\(S\)または\(X \setminus S\)に入るとき,\(f\)は\keyword{普遍的なネット}(universal net)であるという.
\end{definition}
\end{thmbox}

\begin{thmbox}
\begin{theorem}
\(X\)を位相空間,\(f \colon \Lambda \to X\)をネットとする.
以下の2つは同値である.
\begin{conditions}
    \item\label{compact-equiv-cluster} \(x \in X\)が\(f\)の集積点である.
    \item\label{compact-equiv-subnet} \(f\)のサブネットで\(x\)に収束するものが存在する.
\end{conditions}
\end{theorem}
\end{thmbox}

\begin{proof} \proofimplies{compact-equiv-cluster}{compact-equiv-subnet}\(\symcal{D} = \{\tuple{\lambda, U} \mid \text{\(\lambda \in \Lambda\), \(U \in \OpenNeighborhood{x}\), and \(f(\lambda) \in U\)}\}\)と定義する.
\(\symcal{D}\)の前順序\(\mathord{\preceq}\)を\(\tuple{\lambda, U} \preceq \tuple{\lambda', U'} \Leftrightarrow \text{\(\lambda \preceq \lambda'\) and \(U \supseteq U'\)}\)と定める.
このとき\(\symcal{D}\)は有向集合になることを示す.
反射律と推移律は明らかである.
\definitionref{directed-set-uppper}~\ref{directed-set-upper}を示す.
任意の\(\tuple{\lambda_1, U_1}, \tuple{\lambda_2, U_2} \in \symcal{D}\)をとる.
\propositionref{neighborhood-properties}の\ref{neighborhood-cap}で見たように,
\(U_3 := U_1 \cap U_2\)は\(U_3 \in \OpenNeighborhood{x}\), \(U_3 \succeq U_1\), \(U_3 \succeq U_2\)をみたす.
\(x\)は\(f\)の集積点であるから,この\(U_3\)に対して,ある\(\lambda_3 \in \Lambda\)が存在して,
\(\lambda_3 \succeq \lambda_1\), \(\lambda_3 \succeq \lambda_1\), \(f(\lambda_3) \in U_3\)が成り立つ.
ゆえに\(\tuple{\lambda_3, U_3}\)は\(\tuple{\lambda_3, U_3} \succeq \tuple{\lambda_3, U_3}\)かつ\(\tuple{\lambda_3, U_3} \succeq \tuple{\lambda_2, U_2}\)をみたす.

\(\varphi \colon \symcal{D} \to \Lambda\)を\(\varphi\colon \tuple{\lambda, U} \mapsto \lambda\)と定義すると\(g := f \circ \varphi\)は\(f\)のサブネットになることを示す.
任意の\(\kappa \in \Lambda\)について\(\tuple{\kappa, X} \in \symcal{D}\)をとる.
このとき\(\tuple{\xi, U} \succeq \tuple{\kappa, X}\)をみたす任意の\(\tuple{\xi, U}\)について\(\kappa \preceq \xi = \varphi(\xi, U)\)から共終性が成り立つ.

\(g \to x\)を示す.
\(x\)が\(f\)の集積点なので,任意の\(U \in \OpenNeighborhood{x}\)に対して,ある\(\lambda_0 \in \Lambda\)で\(f(\lambda_0) \in U\)となるものが存在する.
したがって\(\tuple{\xi, V} \succeq \tuple{\lambda_0, U}\)をみたす任意の\(\tuple{\xi, V} \in \symcal{D}\)について\(g(\xi, V) = \varphi(\xi) \in V \subseteq U\)となる.

\proofimplies{compact-equiv-subnet}{compact-equiv-cluster}任意の\(\kappa \in \Lambda\)と\(U \in \OpenNeighborhood{x}\)をとる.
\(f\)のあるサブネット\(g \colon \Xi \to X\)が\(x\)に収束するので,
ある\(\xi_1 \in \Xi\)が存在して,任意の\(\xi \succeq \xi_1\)で\(g(\xi) \in U\)となる.

\(g\)は共終な写像\(\varphi \colon \Lambda \to \Xi\)を用いて\(g = f \circ \varphi\)と表される.
共終性から,ある\(\xi_2 \in \Xi\)が存在して,
\(\xi \succeq \xi_2\)をみたす任意の\(\xi \in \Xi\)について\(\varphi(\xi) \succeq \kappa\)が成り立つ.

\(\Xi\)が有向集合であるから\(\xi \succeq \xi_1\), \(\xi \succeq \xi_2\)をみたす\(\xi \in \Xi\)が存在する.
この\(\xi\)について\(\varphi(\xi) \succeq \kappa\)かつ\(f(\varphi(\xi)) = g(\xi)\in U\)が成り立つ.
したがって\(x\)は\(f\)の集積点である.
\end{proof}

\(X\)を位相空間とする.\(X\)の部分集合族\(\symcal{C}\)は
その有限部分集合\(\{C_1, \ldots, C_n\}\)で\(\bigcap_{i = 1}^n C_i  = \emptyset\)をみたすものが存在するとき,\keyword{有限交叉性}(finite intersection property)をもつという.

\begin{thmbox}
\begin{proposition}
\(X\)を位相空間,\(\symcal{F}\)をその閉集合全体の集合とする.
このとき次の2つは同値である.
\begin{conditions}
    \item\label{fip-compact} \(X\)がコンパクトである.
    \item\label{fip-fip} \(\bigcap \symcal{A} = \emptyset\)をみたす任意の\(\symcal{A} \subseteq \symcal{F}\)が有限交叉性をもつ.
\end{conditions}
\end{proposition}
\end{thmbox}

\begin{proof} \proofimplies{fip-compact}{fip-fip}\(\symcal{F}\)の部分集合\(\symcal{A}\)で
\begin{align*}
    \bigcap \symcal{A} = \emptyset
\end{align*}
をみたすようなものをとる.
\(\symcal{V} = \{ X \setminus A \mid A \in \symcal{A}\}\)とすると,これは開集合を元としてもつ集合である.
またDe Morgan則から
\begin{align}
    X \setminus \bigcup \symcal{V}
    = \bigcap \{X \setminus V \mid V \in \symcal{V}\}
    = \bigcap \{A \mid A \in \symcal{A}\}
    = \bigcap \symcal{A}
    = \emptyset \eqlabel{fip-de-morgan}
\end{align}
が成り立つ.ゆえに\(X = \bigcup \symcal{V}\)であり,仮定から\(\{V_1, \ldots, V_n\} \subseteq \symcal{V}\)で
\begin{align*}
    X = \bigcup_{i = 1}^n V_i
\end{align*}
となるものが存在する.
各\(i \in \{1, \ldots, n\}\)について\(A_i = X \setminus V_i\)とすると
\begin{align*}
    \bigcap_{i = 1}^n A_i
    = \bigcap_{i = 1}^n (X \setminus V_i)
    = X \setminus \bigcup_{i = 1}^n V_i
    = \emptyset.
\end{align*}

\proofimplies{fip-fip}{fip-compact}\(\bigcap \symcal{A} = \emptyset\)をみたす任意の\(\symcal{A} \subseteq \symcal{F}\)をとる.
\(\symcal{V} = \{X \setminus A \mid A \in \symcal{A}\}\)とすると,\eqref{fip-de-morgan}と同様,\(X \setminus \bigcup \symcal{V} = \bigcap \symcal{A} = \emptyset\),すなわち\(X = \bigcup \symcal{V}\)が成り立つ.有限交叉性から\(\{A_1, \ldots, A_n\} \subseteq \symcal{A}\)で
\begin{align*}
    \bigcap_{i = 1}^n A_i = \emptyset
\end{align*}
となるものがとれる.
各\(i \in \{1, \ldots, n\}\)について\(V_i = X \setminus A_i\)とすると
\begin{align*}
    X \setminus \bigcup_{i = 1}^n V_i
    = \bigcap_{i = 1}^n (X \setminus V_i)
    = \bigcap_{i = 1}^n A_i
    = \emptyset, % \tag*{\qedhere}
\end{align*}
すなわち\(X = \bigcup_{i = 1}^n V_i\)が成り立つ.
\end{proof}

\begin{thmbox}
\begin{theorem}
位相空間\(X\)の任意のネット\(f \colon \Lambda \to X\)は普遍的なサブネットをもつ.
\end{theorem}
\end{thmbox}

\begin{proof}
集合\(\symfrak{A}\)\footnote{\emphchar{\(\symfrak{A}\)}はフラクトゥール(Fraktur)の\emphchar{A}.
手書きの場合はジュッターリン筆記体(Sütterlinschrift)で\emphchar{\itshape\suetterlin A}のように書く.}を次の2つの性質をもつ\(\symcal{F}\)全体の集合とする.
\begin{inproofconditions}
    \item\label{every-net-has-a-universal-subnet} 任意の\(F \in \symcal{F}\)について\(f\)が最終的に\(F\)に入る.
    \item\label{every-net-has-a-universal-subnet-cap-closed} 任意の\(F, F' \in \symcal{F}\)について\(F \cap F' \in \symcal{F}\).
\end{inproofconditions}
\(\{X\} \in \symfrak{A}\)より\(\symfrak{A} \neq \emptyset\)である.
\(\symfrak{A}\)上の半順序\(\mathord{\preceq}\)を\(\symcal{F} \preceq \symcal{F}' \Leftrightarrow \symcal{F} \subseteq \symcal{F}'\)で定義する.
\(\symfrak{C} \subseteq \symfrak{A}\)\footnote{\emphchar{\(\symfrak{C}\)}はフラクトゥールの\emphchar{C}.
手書きの場合は\emphchar{\itshape\suetterlin C}のように書く.}が鎖(定義x)ならば,
\(\bigcup \symfrak{C}\)は\(\symfrak{A}\)における\(\symfrak{C}\)の上界である.
Zornの補題により\(\symfrak{A}\)の極大元\(\symcal{F}_0\)が存在する.
\begin{align*}
    \symcal{D} = \{\tuple{\lambda, F} \mid
       \oxfordcomma{\lambda \in \Lambda}{F \in \symcal{F}_0}{f(\lambda) \in F}
    \}
\end{align*}
\(\symcal{D}\)上の関係\(\mathord{\preceq}\)を\(\tuple{\lambda, F} \preceq \tuple{\lambda', F'} \Leftrightarrow \text{\(\lambda \preceq \lambda'\) and \(F \supseteq F'\)}\)で定義する.
このとき\(\tuple{\symcal{D}, \mathord{\preceq}}\)は有向集合である.
関係\(\mathord{\preceq}\)が反射律と推移律をみたすことは明らかである.
\definitionref{directed-set}~\ref{directed-set-upper}を示す.
任意の\(\tuple{\lambda_1, F_1}, \tuple{\lambda_2, F_2} \in \symcal{D}\)をとる.
\(\symcal{F}_0 \in \symfrak{A}\)は\ref{every-net-has-a-universal-subnet-cap-closed}をみたすので,\(F_3 := F_1 \cap F_2 \in \symcal{F}_0\)が成り立つ.
\ref{every-net-has-a-universal-subnet}より,ある\(\lambda_0 \in \Lambda\)が存在して,任意の\(\lambda \succeq \lambda_0\)で\(f(\lambda) \in F_3\)となる.
\(\Lambda\)が有向集合であることから各\(i \in \{0, 1, 2\}\)について\(\lambda_3 \succeq \lambda_i\)をみたす\(\lambda_3\)が存在する.
このとき\(\tuple{\lambda_3, F_3} \in \symcal{D}\)は\(\tuple{\lambda_3, F_3} \succeq \tuple{\lambda_1, F_1}\)かつ\(\tuple{\lambda_2, F_2} \succeq \tuple{\lambda_2, F_2}\)をみたす.

写像\(\varphi \colon \symcal{D} \to \Lambda\)を\(\tuple{\lambda, F} \mapsto \lambda\)と定義すると\(g := f \circ \varphi\)は\(f\)のサブネットになることを示す.
任意の\(\kappa \in \Lambda\)について\(\tuple{\kappa, X} \in \symcal{D}\)をとる.
このとき\(\tuple{\xi, U} \succeq \tuple{\kappa, X}\)をみたす任意の\(\tuple{\xi, U}\)について\(\kappa \preceq \xi = \varphi(\xi, U)\)から共終性が成り立つ.


\end{proof}


\begin{thmbox}
\begin{theorem}
位相空間\(X\)において以下の
\begin{conditions}
    \item\label{compact-net-compact} \(X\)がコンパクトである.
    \item\label{compact-net-univ} 任意の普遍的なネット\(f\colon \Lambda \to X\)が収束する.
    \item\label{compact-net-subnet} 任意のネット\(f \colon \Lambda \to X\)が収束するサブネットをもつ.
\end{conditions}
\end{theorem}
\end{thmbox}

\begin{proof} \proofimplies{compact-net-compact}{compact-net-univ}\(f\)が\(X\)のどのような元にも収束しないと仮定する.
すなわちどのような\(x\)についても,そのある開近傍\(U_x\)が存在して,\(f\)が頻繁に\(X \setminus U_x\)に入る.
このことと普遍性から\(f\)は最終的に\(X \setminus U_x\)に入る.
したがってある\(\lambda_x \in \Lambda\)が存在して,任意の\(\lambda \succeq \lambda_x\)で\(f(\lambda_x) \in X \setminus U_x\)となる.
\(\{U_x \mid x \in X\}\)は\(X\)の開被覆であり,\(X\)がコンパクトなので,\(\{U_{x_1}, \ldots, U_{x_n}\}\)が存在して
\begin{align*}
    X = \bigcup_{i = 1}^n U_{x_i}
\end{align*}
となる.\(\Lambda\)が有向集合なので各\(i \in \{1, \ldots, n\}\)について\(\kappa \succeq \lambda_{x_i}\)となる\(\kappa\)をとることができる.
このとき
\begin{align*}
    f(\kappa) \in \bigcap_{i = 1}^n (X \setminus U_{x_i})
              = X \setminus \bigcup_{i = 1}^n U_{x_i}
              = \emptyset
\end{align*}
となるが,これは空集合の定義と矛盾する.

\proofimplies{compact-net-univ}{compact-net-subnet}こんにちは


\proofimplies{compact-net-subnet}{compact-net-compact}\(\symcal{F}\)を有限交叉性をもつような\(X\)の閉部分集合族とする.
\(\symcal{F}\)の部分集合族の


\(\symcal{E}\)を
\(E \succeq E' \Leftrightarrow E \subseteq E'\)と定める.
\end{proof}

\begin{thmbox}
\begin{proposition}
\(f\)を\(X\)から\(Y\)への写像,
\(A\)を\(X\)の部分集合
\begin{conditions}
    \item\label{f-increase} \(A \subseteq A' \Rightarrow f(A) \subseteq f(A')\).
    \item\label{f-inv-increase} \(A \subseteq A' \Rightarrow f^{-1}(A) \subseteq f^{-1}(A')\).
    \item\label{f-cup} \(
        f \left(\bigcup_{\lambda \in \Lambda} V_\lambda \right)
        = \bigcup_{\lambda \in \Lambda} f(V_\lambda)
        \).
    \item\label{f-inv-cup} \(
        f^{-1} \left(\bigcup_{\lambda \in \Lambda} V_\lambda \right)
        = \bigcup_{\lambda \in \Lambda} f^{-1}(V_\lambda)
        \).
    \item\label{f-f-inv-a} \(f(f^{-1}(A)) \subseteq A\).
    \item\label{f-inv-f-a} \(f^{-1}(f(A)) \supseteq A\).
\end{conditions}
\propositionlabel{f-set}
\end{proposition}
\end{thmbox}

\begin{thmbox}
\begin{theorem}[(Tikhonovの定理)]
\({\tuple{\tuple{X_\lambda, \symcal{O}_\lambda}}}_{\lambda \in \Lambda}\)を位相空間の族とする.
各\(\lambda \in \Lambda\)で\(X_\lambda\)がコンパクトならば,
\({\tuple{\tuple{X_\lambda, \symcal{O}_\lambda}}}_{\lambda \in \Lambda}\)の直積位相空間もまたコンパクトである.
\end{theorem}
\end{thmbox}

\begin{proof}
\(\Familyl{X}\)をコンパクト空間の族とし,\(f\colon \Lambda \to \prod_{\lambda \in \Lambda} X_\lambda\)を普遍的なネットとする.
射影\(\pi_\xi \colon \prod_{\lambda \in \Lambda} X_\lambda \to X_\xi\)とする.
任意の\(S \subseteq X_\xi\)
\begin{align*}
    & \prod_{\lambda \in \Lambda} X_\lambda \setminus (\pi_\xi^{-1}(S)) \\
    & = \left\{
        g \colon \Lambda \to \bigcup_{\lambda \in \Lambda} X_\lambda
        \relmiddle{|}
        g(\lambda) \in S \text{ for all } \lambda \in \Lambda \text{ and }
        g(\xi) \not\in S
    \right\} \\
    &= \pi_{\xi}^{-1}(X_\xi \setminus S)
\end{align*}

\end{proof}

\end{document}
