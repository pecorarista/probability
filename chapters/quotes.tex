\documentclass[main.tex]{subfiles}
\begin{document}

\chapter{日本語、英語以外の引用の解説}

\section{公理的な確率の定義}
\seclabel{kolmogorov}

Kolmogoroff著\begin{german}\citetitle{kolmogorov}\end{german}(『確率計算の基礎概念』)の第1章より.

\begin{quotebox}
\begin{german}{\itshape
Jede axiomatische (abstrakte) Theorie läßt
bekanntlich unbegrenzt viele konkrete Interpretationen zu.
In dieser Weise hat auch die mathematische Wahrscheinlichkeitstheorie
neben derjenigen ihrer Interpretationen,
aus der sie aufgewachsen ist,
auch zahlreiche andere.}
\end{german}
\end{quotebox}

\noindent \textbf{\gtfamily 訳} よく知られているように,公理的(抽象的)な理論というのはどれも無限に多くの具体的な解釈をすることが可能である.数学的な確率論もまた例外的ではなく,
その揺籃たる元々の解釈に加えて,いくつもの別な解釈をもつ.


    \trigloss{Jede axiomatisch-e (abstrakt-e) Theorie läßt bekanntlich un-begrenz-t}
              {jeːdə aksioˈmaːtɪʃə apsˈtʀakt teoʀiː lɛst bəkantlɪç ʊnbəgʀɛn\t{ts}t}
              {each[\textsc{nom}.\textsc{f}.\textsc{sg}] axiomatic-\textsc{nom}.\textsc{f}.\textsc{sg} abstract-\textsc{nom}.\textsc{f}.\textsc{sg} theory allow[e\textsc{sg}.\textsc{pres}] {publicly{\textunderscore}known} \textsc{neg}-limit-\textsc{pastp}}
              {Each axiomatic abstract theory allows known unlimitedly}

    \trigloss{viel-e konkret-e Interpretation-en zu In diese-r Weise hat auch}
              {fiːlə kɔŋˈkʀeːtə ɪntɐpʀetaˈ\t{ts}ioːn \t{ts}uː ɪn diːzɐ va\invbreve{ɪ}zə hat aʊχ}
              {many-\textsc{pl} concrete-\textsc{pl} interpretation-\textsc{pl} to in this-\textsc{f}.\textsc{sg}.\textsc{dat} manner have-\textsc{3sg}.\textsc{pres} also}
              {many concrete interpretations to. In this manner has also}

    \trigloss{die mathematische Wahrscheinlich-keit-s-theorie neben derjenigen ihr-er Interpretation-en,}
              {diː mateˈmaːtɪʃə vaːʀˈʃaɪnlɪçkaɪ\t{ts}teoʀiː neːbən ˈdeː\invbreve{ɐ}jeːnɪgən her-\textsc{gen} ɪntɐpʀetaˈ\t{ts}ioːn}
              {the-\textsc{f}.\textsc{sg}.\textsc{nom} mathematical probable-\textsc{nomin}-\textsc{link}-theory} {next{\textunderscore}to} those-\textsc{f}.\textsc{sg}.\textsc{dat} iːʀɐ interpretation-\textsc{pl}}
              {The mathematical {probability theory} {next to} those her interpretations}


\section{古典的な確率の定義}
\seclabel{laplace}

Laplace著\citetitle{laplace}第3版の序章より.

\begin{quotebox}
\begin{french}
{\itshape
La théorie des hasards consiste à réduire tous les événements du même genre à un certain nombre
de cas également possibles, c'est-à-dire tels que nous soyons également indécis sur leur existence,
et à déterminer le nombre de cas favorables à l'événement dont on cherche la probabilité.
Le rapport de ce nombre à celui de tous les cas possibles est la mesure de cette probabilité,
qui n'est ainsi qu'une fraction dont le numérateur est le nombre des cas favorables,
et dont le dénominateur est le nombre de tous les cas possibles.}
\end{french}
\end{quotebox}

\noindent \textbf{\gtfamily 訳} 確率の理論を構成しているのは,
同じ種類からなる事象すべてをいくつかの同程度起こりやすい事象,
すなわち私たちがその存在を同程度に確信できないような事象に還元すること,
そして確率を求めたい事象にとって好ましい場合の数を決定することである.
この数と起こりうる全部の場合の数がこの確率の尺度になる.
このように確率は分子が好ましい場合の数,分母が起こりうるすべての場合の数であるような分数に過ぎない.

\documentclass[main.tex]{subfiles}
\begin{document}

\chapter{古典的な確率の定義}


    \trigloss{L-a théorie des hasard-s consist-e à réduire tous}
              {la teɔʁi de azaʁ {k\~{ɔ}sist\liaison} a ʁedɥiʁ tu}
              {the-\textsc{f}.\textsc{sg} theory {of{\textunderscore}the[\textsc{m}.\textsc{pl}]} chance-\textsc{pl} consist-3\textsc{sg}.\textsc{pres} to {reduce[\textsc{inf}]} all}
              {The theory of probability involves  reducing all}

    \trigloss{l-es événement-s du même genre à un certain nombre}
              {{lez\liaison} evɛnm\~{ɑ} dy mɛm {ʒ\~{ɑ}ʁ\liaison} a \~{œ} sɛʁt\~{ɛ} n\~{ɔ}bʁ}
              {the-\textsc{m}.\textsc{pl} event-\textsc{pl} {of{\textunderscore}the[\textsc{m}.\textsc{sg}]} same kind to a certain number}
              {the events of the same kind to a certain number}

    \trigloss{de cas également possible-s c'-est-à-dire tel-s que nous}
              {də kɑ eɡalm\~{ɑ} pɔsibl sɛtadiʁ tɛl kə nu}
              {of {case[\textsc{pl}]} equally possible-\textsc{pl} that-is-to-say such-\textsc{pl} that we}
              {of equally possible cases, that is, such that we}

    \trigloss{soyons également indécis sur leur existence, et}
              {swaj\~{ɔ} egalm\~{ɑ} \~{ɛ}desi syʁ {lœʁ\liaison} ɛgzist\~{ɑ}s e}
              {{be[1\textsc{pl}.\textsc{pres}.\textsc{subj}]} equally {uncertain[\textsc{m}.\textsc{pl}]} on their existence and}
              {are equally uncertain about their occurrence, and}

    \trigloss{à déterminer l-e nombre de cas favorable-s à l'}
              {a detɛʁmine lə n\~{ɔ}bʁ də kɑ {favɔʁabl\liaison} a {l{\liaison}}}
              {to {determine[\textsc{inf}]} the-\textsc{m}.\textsc{sg} number of {case[\textsc{pl}]} favorable-\textsc{pl} to {the[\textsc{m}.\textsc{sg}]}}
              {determining the number of cases favorable to the}

    \trigloss{événement dont on cherch-e l-a probabilité.}
              {evɛnm\~{ɑ} d\~{ɔ} \~{ɔ} ʃɛʁʃə la prɔbabilite}
              {event of{\textunderscore}which {people\textsubscript{3\textsc{sg}}} seek-3\textsc{sg}.\textsc{pres} {the-\textsc{f}.\textsc{sg}} probability.}
              {event whose probability we seek.}

    \trigloss{L-e rapport de ce nombre à celui de tous l-es cas}
              {lə ʁapɔʁ də sə {n\~{ɔ}bʁ{\liaison}} a səlɥi də tu le kɑ}
              {the-\textsc{m}.\textsc{sg} ratio of {this[\textsc{m}.\textsc{sg}]} number to {that[\textsc{m}.\textsc{sg}]} of all {the-\textsc{m}.\textsc{pl}} {case[\textsc{pl}]}}
              {The ratio of this number to that of all possible cases}

    \trigloss{possible-s est l-a mesure de cette probabilité, qui}
              {{pɔsibl{\liaison}} ɛ la məzyʁ də sɛt pʁɔbabilite ki}
              {possible-\textsc{m}.\textsc{pl} {be[3\textsc{sg}.\textsc{pres}]} the-\textsc{f}.\textsc{sg} measure of {this[\textsc{f}.\textsc{sg}]} probability which}
              {is the measure of this probability, which}

    \trigloss{n' est ainsi qu' une fraction dont l-e numérateur}
              {n {ɛt{\liaison}} \~{ɛ}si {k{\liaison}} yn fʁaksj\~{ɔ} d\~{ɔ} lə {nymeʁatœʁ\liaison}}
              {\textsc{neg} {be[3\textsc{sg}.\textsc{pres}]} thus only one fraction whose the-\textsc{m}.\textsc{sg} numerator}
              {is thus only  a fraction whose  numerator}

    \trigloss{est l-e nombre des cas favorable-s, et dont}
              {ɛ lə n\~{ɔ}bʁ de kɑ favɔʁabl e d\~{ɔ}}
              {{be[3\textsc{sg}.\textsc{pres}]} the-\textsc{m}.\textsc{sg} number {of{\textunderscore}the[\textsc{m}.\textsc{pl}]} {case[\textsc{pl}]} favorable-\textsc{pl} and whose}
              {is the number of favorable cases, and whose}

    \trigloss{l-e dénominateur est l-e nombre de tous l-es}
              {lə {denɔminatœʁ\liaison} ɛ lə n\~{ɔ}bʁ də tu le}
              {the-\textsc{m}.\textsc{sg} denominator {be[3\textsc{sg}.\textsc{pres}]} the-\textsc{m}.\textsc{sg} number of all the-\textsc{m}.\textsc{sg}}
              {denominator is the number of all}

    \trigloss{cas possible-s.}
              {kɑ pɔsibl}
              {case[\textsc{pl}] possible-\textsc{pl}}
              {possible cases.}


\end{document}



\end{document}
