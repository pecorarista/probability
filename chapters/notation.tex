\documentclass[../main.tex]{subfiles}
\begin{document}
\chapter*{記法}

\begin{table}[h]
\centering
\begin{tabular}{ll}
\toprule
    \header{表記}     &  \header{意味} \\
\midrule
    \(x := y\)            &  \(x\)を\(y\)と定義する \\
    \(\symbb{N}\)         &  自然数全体の集合\(\{0, 1, \ldots\}\) \\
    \(\PositiveInteger\)  &  正整数全体の集合\(\{1, 2, \ldots\}\) \\
    \(\NonNegReal\)       &  非負実数全体の集合\(\{x \in \symbb{R} \mid x \geq 0\}\) \\
    \(\PositiveReal\)     &  正実数全体の集合\(\{ x \in \symbb{R} \mid x > 0\}\) \\
    \(\symbb{K}\)         &  \(\symbb{R}\)または\(\symbb{C}\) \\
    \(\powerset{X}\)      &  集合\(X\)の冪集合 \\
\bottomrule
\end{tabular}
\end{table}

\end{document}
