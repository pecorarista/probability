\documentclass[main.tex]{subfiles}
\begin{document}
\mainchapter{測度と確率}

\keyword{モデル化}(modeling)とは,現実世界にあるものやそこで生じる現象を,興味のない部分は捨象し,興味のある部分だけを別な単純なものに対応させる行為である.対応させた先の単純なものを\keyword{モデル}(model)という.
力学でボールを\(\symbb{R}^2\)の点に対応させてそれに働く力を議論したり,
経済学で消費者の好みを選好関係\footnote{%
消費者にとって何らかの価値をもっているものを財(good)と呼ぶ.
財の集合を\(G\)とする.消費者\(i\)が\(x \in G\)を\(y \in G\)より好むならば\(y \prec_i x\),どちらも同程度に好ましいと考えるならば\(y \sim_i x\)と表す.また「\(y \prec_i x\)または\(y \sim_i x\)」が成り立つとき\(y \precsim_i x\)と表す.このように定めた各\(R \in \{\prec_i, \sim_i, \precsim_i\}\)は\(G\)上の関係(\(G \times G\)の部分集合)である.
}で表現したりするのはモデル化の例である.
科学において\keyword{確率}(probability)とはあらかじめ結果を知ることができない実験や観測をモデル化するため,起こりうるそれぞれの「結果」にその起こりやすさを割り振る方法である.
実験や観測の「結果」はそれをモデル化する者が適切に設定する必要がある.
サイコロを1度投げる実験を例にとると,それぞれの目が出ることを\(\dice{1}\)のような記号で表すとき,「結果」全体の集合は
\begin{align}
    \Omega = \{\dice{1}, \dice{2}, \dice{3}, \dice{4}, \dice{5}, \dice{6}\} \eqlabel{dice}
\end{align}
のように表すことができるだろう.もちろん記号の選び方は自由であり\(\Omega = \{1, 2, 3, 4, 5, 6\}\)のように表してもよい.
このような「結果」全体の集合を\keyword{標本空間}(sample space)と呼び,しばしば\(\Omega\)で表す.
「結果」という語は\(\dice{1}\)のように単一の記号で表されるものを指す場合と,「目が\(2\)以下である」(\(\{\dice{1}, \dice{2}\}\))のような複数の記号の集合で表されるものを指す場合がある.
これらを区別するため,\(\dice{1}\)のような,これ以上分割ができないとみなせる実験や観測の結果を\keyword{根源事象}(atomic event)と呼び,根源事象を元にもつ集合で表される結果を\keyword{事象}(event)を呼ぶ.
この用語を使うと,確率は各事象に起こりやすさを対応させる関数と定義することができる.
個々の根源事象(例えば\(\dice{1}\))については確率を直接は定義しない.
なぜならば各事象の確率を定めることによって単一の根源事象からなる事象(例えば\(\{\dice{1}\}\))の確率が定まり,結果として根源事象の確率とみなせるものが定まるからである.
応用分野では「事象\(\{\omega\}\)の確率」が単に「\(\omega\)の確率」と書かれたり,\(P(\{\omega\})\)を単に\(P(\omega)\)のように書かれたりすることが多い.
しかしこのような表現は初学者の混乱を招く可能性が高いため,本書では使う際には断りを入れる.

最もよく用いられる確率の定義は\begin{german}Kolmogoroff\end{german}\footnote{%
\pronunciation{\begin{russian}Колмогоров\end{russian}}{ru}{kəlmɐˈɡorəf}{コルモゴロフ}}によるもので,その本質は確率がみたしてほしい性質を列挙したものである.
定義の正確な内容を示す前に,どうしてそれらの性質が必要とされるのかを説明しよう.
1つ目の性質は,任意の事象\(A\)についてその確率\(P(A)\)は\(0 \leq P(A) \leq 1\)であるというものである.その理由としては,古典的な確率がもともと割合を表すものだったということ,またそれと整合的であるほうが便利だということが大きい\footnote{%
もちろん数学においては定義として確立されてしまえば,
その背景にある思想は
}.

古典的な確率の定義\cite{laplace}は付録参照.


1つの代表的な方法として定義域として標本空間の\(\Omega\)の冪集合\(\wp(\Omega)\)を考えるのが普通である.
例えば\(\Omega\)を\eqref{dice}とするとき,多くの場合
\begin{align*}
    & \wp(\Omega) = \{ \\
    & \ \ \ \ \emptyset, \{\dice{1}\}, \{\dice{2}\}, \{\dice{3}\}, \{\dice{4}\}, \{\dice{5}\}, \{\dice{6}\}, \\
    & \ \ \ \ 
        \{\dice{1}, \dice{2}\},
        \{\dice{1}, \dice{3}\},
        \{\dice{1}, \dice{4}\},
        \{\dice{1}, \dice{5}\},
        \{\dice{1}, \dice{6}\}, \\
    & \ \ \ \ 
        \{\dice{2}, \dice{3}\},
        \{\dice{2}, \dice{4}\},
        \{\dice{2}, \dice{5}\},
        \{\dice{2}, \dice{6}\},
        \{\dice{3}, \dice{4}\}, \\
    & \ \ \ \ 
        \{\dice{3}, \dice{5}\},
        \{\dice{3}, \dice{6}\},
        \{\dice{4}, \dice{5}\},
        \{\dice{4}, \dice{6}\},
        \{\dice{5}, \dice{6}\}, \\
    & \ \ \ \ 
        \{\dice{1}, \dice{2}, \dice{3}\},
        \{\dice{1}, \dice{2}, \dice{4}\},
        \{\dice{1}, \dice{2}, \dice{5}\},
        \{\dice{1}, \dice{2}, \dice{6}\},
        \{\dice{1}, \dice{3}, \dice{4}\}, \\
    & \ \ \ \ 
        \{\dice{1}, \dice{3}, \dice{5}\},
        \{\dice{1}, \dice{3}, \dice{6}\},
        \{\dice{1}, \dice{4}, \dice{5}\},
        \{\dice{1}, \dice{4}, \dice{6}\},
        \{\dice{1}, \dice{5}, \dice{6}\}, \\
    & \ \ \ \ 
        \{\dice{2}, \dice{3}, \dice{4}\},
        \{\dice{2}, \dice{3}, \dice{5}\},
        \{\dice{2}, \dice{3}, \dice{6}\},
        \{\dice{2}, \dice{4}, \dice{5}\},
        \{\dice{2}, \dice{4}, \dice{6}\}, \\
    & \ \ \ \ 
        \{\dice{2}, \dice{5}, \dice{6}\},
        \{\dice{3}, \dice{4}, \dice{5}\},
        \{\dice{3}, \dice{4}, \dice{6}\},
        \{\dice{3}, \dice{5}, \dice{6}\},
        \{\dice{4}, \dice{5}, \dice{6}\}, \\
    & \ \ \ \ 
        \Omega \setminus \{\dice{1}, \dice{2}\},
        \Omega \setminus \{\dice{1}, \dice{3}\},
        \Omega \setminus \{\dice{1}, \dice{4}\},
        \Omega \setminus \{\dice{1}, \dice{5}\},
        \Omega \setminus \{\dice{1}, \dice{6}\}, \\
    & \ \ \ \ 
        \Omega \setminus \{\dice{2}, \dice{3}\},
        \Omega \setminus \{\dice{2}, \dice{4}\},
        \Omega \setminus \{\dice{2}, \dice{5}\},
        \Omega \setminus \{\dice{2}, \dice{6}\},
        \Omega \setminus \{\dice{3}, \dice{4}\}, \\
    & \ \ \ \ 
        \Omega \setminus \{\dice{3}, \dice{5}\},
        \Omega \setminus \{\dice{3}, \dice{6}\},
        \Omega \setminus \{\dice{4}, \dice{5}\},
        \Omega \setminus \{\dice{4}, \dice{6}\},
        \Omega \setminus \{\dice{5}, \dice{6}\}, \\
    & \ \ \ \  \Omega \setminus \{\dice{1}\}, \Omega \setminus \{\dice{2}\}, \Omega \setminus \{\dice{3}\}, \Omega \setminus \{\dice{4}\}, \Omega \setminus \{\dice{5}\}, \Omega \setminus \{\dice{6}\}, \Omega \\
    & \}
\end{align*}
をという\(2^6 = 64\)個の元をもつ集合をとる.

次に標本空間が無限集合の場合を考える.
まだサイコロやコインなどを使った実験のモデリングでしか確率を扱ったことのない初学者には
そもそも標本空間が無限集合の場合を考える必要があるのかという疑問が生じるかもしれない.
無限集合の例としてルーレットを回す実験を考える.
ルーレット盤を\(\symbb{R}^2\)における原点\(O = (0, 0)\)を中心とする半径\(1\)の円板とし,
その円周上にルーレットの回転に合わせて動く点\(P = (x, y)\)と,回転とは無関係な点\(A = (0, 1)\)をとる.
我々が興味をもっているのはルーレット盤を適当に回した後の\(\angle AOP\)の大きさだとする.
このとき標本空間は\([-\pi, \pi)\)や\([0, 2\pi)\)のような無限集合になる.

この集合は非常に大きな
そのうち興味があるのは区間や複数の区間の和集合,それらの補集合などごく限られた形のものだろう.
今は極めて単純な実験を考えたが,一般的にも同様であり,

\begin{figure}
    \centering
    \begin{tikzpicture}
        \coordinate (O) at (0, 0);
        \draw[fill] (0, 0) circle (1.4pt) node[left=0.2cm, below=0.2cm] {\(O\)};
        \coordinate (A) at ({2cm * cos(0)}, {2cm * sin(0)});
        \draw[fill] ({2cm * cos(0)}, {2cm * sin(0)}) circle (1.4pt) node[right=0.2cm] {\(A\)};
        \coordinate (P) at ({2cm * cos(120)}, {2cm * sin(120)});
        \draw[fill] ({2cm * cos(120)}, {2cm * sin(120)}) circle (1.4pt) node[above=0.15cm, left=0.13cm] {\(P\)};
        \pic[%
            draw=sRed,
            thick,
            -stealth,
            angle radius=2cm,
        ] {angle=A--O--P};
        \pic[%
            draw,
            thick,
            angle radius=2cm,
        ] {angle=P--O--A};
        \node at ({2.4cm * cos(60)}, {2.4cm * sin(60)}) {\textcolor{sRed}{\(\omega\)}};
        \draw[thick, dotted] (A) -- (O) -- (P);
        \coordinate (X) at (1.6cm, 0);
        \coordinate (Y) at (2.2cm, 0.2cm);
        \coordinate (Z) at (2.2cm, -0.2cm);
        \fill[fill=sBlue, fill opacity=0.5] (X) -- (Y) -- (Z) -- cycle;
    \end{tikzpicture}
    \caption{あ.}\figlabel{darts}
\end{figure}


\begin{thmbox}
\begin{definition}
\(X\)を集合とし,\(\symcal{F}\)をその部分集合族とする.
\(\symcal{F}\)が以下の性質\ref{sigma-algebra-x-in-f}--\ref{sigma-algebra-sigma-cup-closed}をみたすとき,
\(\symcal{F}\)を\(X\)上の\keyword{\(\symbf{\sigma}\)-代数}(\(\sigma\)-algebra)という{\footnotemark}.
\begin{conditions}
    \item\label{sigma-algebra-x-in-f}
        \(X \in \symcal{F}\).
    \item\label{sigma-algebra-comp-closed} (\(\setcomp{\placeholder}\)-閉性) \(A \in \symcal{F}\)ならば\(\setcomp{A} \in \symcal{F}\).
    \item\label{sigma-algebra-sigma-cup-closed} (\(\sigma\)-\(\mathord{\cup}\)-閉性)\(\symcal{F}\)の任意の列\(\Sequence{A}\)について
        \begin{align*}
            \bigcup_{n \in \PositiveInteger} A_n \in \symcal{F}.
        \end{align*}
\end{conditions}
\end{definition}
\end{thmbox}

\footnotetext{\inhibitglue 「\(\sigma\)-加法族」という訳語もよく用いられる.}

\end{document}
