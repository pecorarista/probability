\documentclass[main.tex]{subfiles}
\begin{document}
\mainchapter{測度と確率}

\keyword{モデル化}(modeling)とは,現実世界にあるものやそこで生じる現象を,興味のない部分は捨象し,興味のある部分だけを別な単純なものに対応させる行為である.対応させた先の単純なものを\keyword{モデル}(model)という.
力学でボールを\(\symbb{R}^2\)の点に対応させてそれに働く力を議論したり,
経済学で消費者の好みを選好関係\footnote{%
消費者にとって何らかの価値をもっているものを財(good)と呼ぶ.
財の集合を\(G\)とする.消費者\(i\)が\(x \in G\)を\(y \in G\)より好むならば\(y \prec_i x\),どちらも同程度に好ましいと考えるならば\(y \sim_i x\)と表す.また「\(y \prec_i x\)または\(y \sim_i x\)」が成り立つとき\(y \precsim_i x\)と表す.このように定めた各\(R \in \{\prec_i, \sim_i, \precsim_i\}\)は\(G\)上の関係(\(G \times G\)の部分集合)である.
}で表現したりするのはモデル化の例である.
科学において\keyword{確率}(probability)とはあらかじめ結果を知ることができない実験や観測をモデル化するため,起こりうるそれぞれの「結果」にその起こりやすさを割り振る方法である.
実験や観測の「結果」はそれをモデル化する者が適切に設定する必要がある.
サイコロを1度投げる実験を例にとると,それぞれの目が出ることを\(\dice{1}\)のような記号で表すとき,「結果」全体の集合は
\begin{align}
    \Omega = \{\dice{1}, \dice{2}, \dice{3}, \dice{4}, \dice{5}, \dice{6}\} \eqlabel{dice}
\end{align}
のように表すことができるだろう.もちろん記号の選び方は自由であり\(\Omega = \{1, 2, 3, 4, 5, 6\}\)のように表してもよい.
このような「結果」全体の集合を\keyword{標本空間}(sample space)と呼び,しばしば\(\Omega\)で表す.
「結果」という語は\(\dice{1}\)のように単一の記号で表されるものを指す場合と,「目が\(2\)以下である」(\(\{\dice{1}, \dice{2}\}\))のような複数の記号の集合で表されるものを指す場合がある.
これらを区別するため,\(\dice{1}\)のような,これ以上分割ができないとみなせる実験や観測の結果を\keyword{根源事象}(atomic event)と呼び,根源事象を元にもつ集合で表される結果を\keyword{事象}(event)を呼ぶ.
この用語を使うと,確率は各事象に起こりやすさを対応させる関数と定義することができる.
個々の根源事象(例えば\(\dice{1}\))については確率を直接は定義しない.
なぜならば各事象の確率を定めることによって単一の根源事象からなる事象(例えば\(\{\dice{1}\}\))の確率が定まり,結果として根源事象の確率とみなせるものが定まるからである.
応用分野では「事象\(\{\omega\}\)の確率」が単に「\(\omega\)の確率」と書かれたり,\(P(\{\omega\})\)を単に\(P(\omega)\)のように書かれたりすることが多い.
しかしこのような表現は初学者の混乱を招く可能性が高いため,本書では使う際には断りを入れる.

最もよく用いられる確率の定義は\begin{german}Kolmogorov\end{german}\footnote{%
\pronunciation{\begin{russian}Колмогоров\end{russian}}{ru}{kəlmɐˈɡorəf}{コルモゴロフ.現代の一般的なラテン文字転写ではKolmogorovと綴られるが,出版物の著者名としてはKolmogoroffが使われている}}によるもので,その本質は確率がみたしてほしい性質を列挙したものである.
定義の正確な内容を示す前に,どうしてそれらの性質が必要とされるのかを説明しよう.
もちろん定義にはそのような「気持ち」は含まれていないので筆者による解釈が含まれている.
数学において定義の解釈は無限に存在し,確率の定義もまた例外ではない\footnote{%
詳しくは\secref{kolmogorov}の\begin{german}\citetitle{kolmogorov}\end{german}の引用およびその解説を参照せよ.}.

1つ目の性質は,任意の事象\(A\)についてその確率\(P(A)\)は\(0 \leq P(A) \leq 1\)であるというものである.その理由としては,古典的な確率がもともと割合を表すものだったということ,またそれと整合的であるほうが便利だということが大きいだろう.古典的な確率の定義は例えばLaplace\footnote{%
\pronunciation{\begin{french}Laplace\end{french}}{fr}{laplas}{ラプラス}
}によると,
結果全体を同じくらい起こりやすい場合に分けて,着目する事象に当てはまる場合の数をすべての場合の数で割ったものをその事象の確率とする\footnote{%
\secref{laplace}の\begin{french}\citetitle{laplace}\end{french}の引用およびその解説も参照せよ.}.
例えば2個のサイコロを投げるとき,すべての場合の数は次の36通りとするのが妥当だろう.
\begin{align*}
    & (\dice{1}, \dice{1}), (\dice{1}, \dice{2}), (\dice{1}, \dice{3}), (\dice{1}, \dice{4}), (\dice{1}, \dice{5}), (\dice{1}, \dice{6}), \\
    & (\dice{2}, \dice{1}), (\dice{2}, \dice{2}), (\dice{2}, \dice{3}), (\dice{2}, \dice{4}), (\dice{2}, \dice{5}), (\dice{2}, \dice{6}), \\
    & (\dice{3}, \dice{1}), (\dice{3}, \dice{2}), (\dice{3}, \dice{3}), (\dice{3}, \dice{4}), (\dice{3}, \dice{5}), (\dice{3}, \dice{6}), \\
    & (\dice{4}, \dice{1}), (\dice{4}, \dice{2}), (\dice{4}, \dice{3}), (\dice{4}, \dice{4}), (\dice{4}, \dice{5}), (\dice{4}, \dice{6}), \\
    & (\dice{5}, \dice{1}), (\dice{5}, \dice{2}), (\dice{5}, \dice{3}), (\dice{5}, \dice{4}), (\dice{5}, \dice{5}), (\dice{5}, \dice{6}), \\
    & (\dice{6}, \dice{1}), (\dice{6}, \dice{2}), (\dice{6}, \dice{3}), (\dice{6}, \dice{4}), (\dice{6}, \dice{5}), (\dice{6}, \dice{6}).
\end{align*}
もし着目している事象が「ゾロ目が出る」であればこれに当てはまる場合は
\begin{align*}
    (\dice{1}, \dice{1}), (\dice{2}, \dice{2}), (\dice{3}, \dice{3}), (\dice{4}, \dice{4}), (\dice{5}, \dice{5}), (\dice{6}, \dice{6})
\end{align*}
の6通りである.したがってゾロ目が出る事象の確率は\(\inlinefrac{6}{36} = \inlinefrac{1}{6}\)となる.

2つ目の性質は,同時に起こることがない任意の2つの事象\(A\)と\(B\)について
\begin{align*}
    P(\text{\(A\)または\(B\)}) = P(A) + P(B)
\end{align*}
が成り立つというものである.たとえば\(A\)を「2つの目の合計が3以下」,
\(B\)を「2つの目の合計が10以上」とする.
\(A\)に当てはまる場合は
\begin{align*}
    (\dice{1}, \dice{1}), (\dice{1}, \dice{2}), (\dice{2}, \dice{1})
\end{align*}
の3通りであり,\(B\)に当てはまる場合は
\begin{align*}
    (\dice{4}, \dice{6}), (\dice{6}, \dice{4}), (\dice{5}, \dice{5}),
    (\dice{5}, \dice{6}), (\dice{6}, \dice{5}), (\dice{6}, \dice{6})
\end{align*}
の6通りである.
このとき\(A\)または\(B\)に当てはまるのは\(9\)通りであり
\begin{align*}
    P(\text{\(A\)または\(B\)}) = \frac{10}{36} = \frac{3}{36} + \frac{6}{36} = P(A) + P(B)
\end{align*}
が成り立つ.

1つの代表的な方法として定義域として標本空間の\(\Omega\)の冪集合\(\wp(\Omega)\)を考えるのが普通である.
例えば\(\Omega\)を\eqref{dice}とするとき,多くの場合
をという\(2^6 = 64\)個の元をもつ集合をとる.

次に標本空間が無限集合の場合を考える.
まだサイコロやコインなどを使った実験のモデリングでしか確率を扱ったことのない初学者には
そもそも標本空間が無限集合の場合を考える必要があるのかという疑問が生じるかもしれない.
無限集合の例としてルーレットを回す実験を考える.
ルーレット盤を\(\symbb{R}^2\)における原点\(O = (0, 0)\)を中心とする半径\(1\)の円板とし,
その円周上にルーレットの回転に合わせて動く点\(P = (x, y)\)と,回転とは無関係な点\(A = (0, 1)\)をとる.
我々が興味をもっているのはルーレット盤を適当に回した後の\(\angle AOP\)の大きさだとする.
このとき標本空間は\([-\pi, \pi)\)や\([0, 2\pi)\)のような無限集合になる.

この集合は非常に大きな
そのうち興味があるのは区間や複数の区間の和集合,それらの補集合などごく限られた形のものだろう.
今は極めて単純な実験を考えたが,一般的にも同様であり,

\begin{figure}
    \centering
    \begin{tikzpicture}
        \coordinate (O) at (0, 0);
        \draw[fill] (0, 0) circle (1.4pt) node[left=0.2cm, below=0.2cm] {\(O\)};
        \coordinate (A) at ({2cm * cos(0)}, {2cm * sin(0)});
        \draw[fill] ({2cm * cos(0)}, {2cm * sin(0)}) circle (1.4pt) node[right=0.2cm] {\(A\)};
        \coordinate (P) at ({2cm * cos(120)}, {2cm * sin(120)});
        \draw[fill] ({2cm * cos(120)}, {2cm * sin(120)}) circle (1.4pt) node[above=0.15cm, left=0.13cm] {\(P\)};
        \pic[%
            draw=sRed,
            thick,
            -stealth,
            angle radius=2cm,
        ] {angle=A--O--P};
        \pic[%
            draw,
            thick,
            angle radius=2cm,
        ] {angle=P--O--A};
        \node at ({2.4cm * cos(60)}, {2.4cm * sin(60)}) {\textcolor{sRed}{\(\omega\)}};
        \draw[thick, dotted] (A) -- (O) -- (P);
        \coordinate (X) at (1.6cm, 0);
        \coordinate (Y) at (2.2cm, 0.2cm);
        \coordinate (Z) at (2.2cm, -0.2cm);
        \fill[fill=sBlue, fill opacity=0.5] (X) -- (Y) -- (Z) -- cycle;
    \end{tikzpicture}
    \caption{あ.}\figlabel{darts}
\end{figure}


\section{部分集合族の代数的性質}

\(\symcal{F}\)を集合\(X\)の部分集合族とする.
\(\symcal{F}\)上の演算について
\begin{itemize}
    \item (\(\mathord{\cap}\)-閉性)\(A, B \in \symcal{F}\)ならば\(A \cap B \in \symcal{F}\)が成り立つ.
    \item (\(\mathord{\cup}\)-閉性)\(A, B \in \symcal{F}\)ならば\(A \cup B \in \symcal{F}\)が成り立つ.
    \item (\(\mathord{\setminus}\)-閉性)\(A, B \in \symcal{F}\)ならば\(A \setminus B \in \symcal{F}\)が成り立つ.
    \item (\(\setcomp{\placeholder}\)-閉性)\(A \in \symcal{F}\)ならば\(\setcomp{A} = X \setminus A \in \symcal{F}\)が成り立つ.
    \item (\(\sigma\)-\(\mathord{\cup}\)-閉性)\(\symcal{F}\)の任意の列\(\Sequence{A}\)について
        \begin{align*}
            \bigcup_{n \in \PositiveInteger} A_n \in \symcal{F}.
        \end{align*}
\end{itemize}
\(\mathord{*}\)を\(\symcal{F}\)上の演算とする.
\(\symcal{F}\)が\(\mathord{*}\)-閉性をみたすとき,\(\symcal{F}\)は\(\mathord{*}\)-閉(\(\mathord{*}\)-closed)である,あるいは\(\mathord{*}\)について閉じている(closed under \(\mathord{*}\))ということもある.
「\(\sigma\)-\(\mathord{\cup}\)-閉」の〈\(\sigma\)〉は可算個の集合の和集合に関係しているものにつけられる接辞で,ドイツ語において和や和集合を意味する\begin{german}«Summe»\end{german}\footnote{\begin{german}Summe\end{german} [ˈzʊmə].現代のドイツ語で和集合を意味する語としては\textgerman{«Vereinigung»} [fɛ\invbreve{ɐ}ˈʔa\invbreve{ɪ}nɪgʊŋ]のほうが一般的である.
% 可算個の集合の共通部分に関係しているものには\textgerman{«Durchschnitt»}
% [ˈdʊʁçʃnɪt]
% にちなんで\(\delta\)が使われる.
% 例えば可算個の開集合\textgerman{«Gebiet»} [ɡə\/ˈbiːt]の共通部分であるような集合を\(G_\delta\)と呼ぶ.
}の\emphchar{s}に対応するギリシャ文字である.

\begin{thmbox}
\begin{definition}
集合\(X\)の部分集合族\(\symcal{F}\)が演算\(\mathord{\cap}\)について閉じているとき,\(\symcal{F}\)を
\keyword{\(\symbf{\pi}\)-システム}(\(\pi\)-system)という.
\end{definition}
\end{thmbox}

\begin{thmbox}
\begin{definition}
集合\(X\)の部分集合族\(\symcal{F}\)が次の性質をみたすとき,\(\symcal{F}\)を\keyword{集合半環}(semiring of sets)という:
\begin{conditions}
    \item\label{semiring-emptyset} \(\emptyset \in \symcal{F}\).
    \item (\(\mathord{\cap}\)-閉性)\(A, B \in \symcal{F}\)ならば\(A \cap B \in \symcal{F}\)が成り立つ.
    \item\label{semiring-setminus}(区分的\(\mathord{\setminus}\)-閉性\footnotemark)
        \(A, B \in \symcal{F}\)ならば,互いに素な\(C_1, \ldots, C_n \in \symcal{F}\)が存在して\(A \setminus B = \bigcup_{i = 1}^n C_i\)が成り立つ.
\end{conditions}
\definitionlabel{semiring-of-sets}
\end{definition}
\end{thmbox}

\footnote{「区分的\(\mathord{\setminus}\)-閉性」は便宜的につけたもので一般的な名称ではない.}

\begin{thmbox}
\begin{definition}
\(X\)を集合とし,\(\symcal{F}\)をその部分集合族とする.
\(\symcal{F}\)が以下の性質\ref{sigma-algebra-x-in-f}--\ref{sigma-algebra-sigma-cup-closed}をみたすとき,
\(\symcal{F}\)を\(X\)上の\keyword{\(\symbf{\sigma}\)-代数}(\(\sigma\)-algebra)という{\footnotemark}.
\begin{conditions}
    \item\label{sigma-algebra-x-in-f}
        \(X \in \symcal{F}\).
    \item\label{sigma-algebra-comp-closed} (\(\setcomp{\placeholder}\)-閉性) \(A \in \symcal{F}\)ならば\(\setcomp{A} \in \symcal{F}\).
    \item\label{sigma-algebra-sigma-cup-closed} (\(\sigma\)-\(\mathord{\cup}\)-閉性)\(\symcal{F}\)の任意の列\(\Sequence{A}\)について
        \begin{align*}
            \bigcup_{n \in \PositiveInteger} A_n \in \symcal{F}.
        \end{align*}
\end{conditions}
\end{definition}
\end{thmbox}

\footnotetext{\inhibitglue 「\(\sigma\)-加法族」という訳語もよく用いられる.}

一般に集合半環は代数学の意味での半環ではないが,よく似た性質をもっている.
まず\(\symcal{F}\)上の加法を\keyword{対称差}(symmetric difference)とする.
対称差とは\(\symdiff{A}{B} = (A \setminus B) \cup (B \setminus A)\)と定義されるもので,図示すると\figref{symmetric-difference}のようになる.
\(\symcal{F}\)が集合半環の条件をみたすならば,\(\symcal{F}\)は\(\symdiffsymbol\)について閉じている.
また\(\symdiffsymbol\)について結合律と可換律が成り立つ.
この演算の単位元\(\emptyset\)は定義から\(\symcal{F}\)に含まれている.
よって\(\langle \symcal{F}, \mathord{\triangle}, \emptyset \rangle\)は可換モノイドである.
次に\(\symcal{F}\)上の乗法を\(\mathord{\cap}\)とする.
これは結合律・可換律・分配律をみたしている.
また加法単位元\(\emptyset\)は乗法\(\mathord{\cap}\)によって\(\symcal{F}\)を零化する.
しかし乗法単位元\(X\)が\(\symcal{F}\)に含まれるかどうかは上の定義からわからない.
したがって集合半環は全体集合を含んでいれば代数学の意味での半環でもあり,
そうでなければ代数学の意味での半環にはあと一歩足りないということになる.

\begin{figure}[h]
    \centering
    \begin{tikzpicture}[draw opacity=1, fill opacity=0.4, text opacity=1]
        \draw[thick] (-2.35, 1.9) rectangle (2.35, -1.9);
        \draw[thick, fill=sRed, even odd rule] (-0.65, 0) circle (1) (0.65, 0) circle (1);
        \node at (0, 1.3) {\(\symdiff{A}{B}\)};
        \node at (-2, 1.5) {\(X\)};
        \node at (-1.0, 0.6) {\(A\)};
        \node at (0.9, 0.6) {\(B\)};
    \end{tikzpicture}
    \caption{\(X\)の部分集合\(A\)と\(B\)の対称差\(\symdiff{A}{B}\).}
    \figlabel{symmetric-difference}
\end{figure}

\begin{example}
\(\symbb{R}\)の左開右閉区間の集合\(\symcal{I} = \{(a, b] \mid a \leq b\}\)は集合半環である.
\examplelabel{interval-semiring}
\end{example}

\begin{proof} \(\emptyset = (a, a] \in \symcal{I}\)である.
\(\mathord{\cap}\)-閉であることは
\begin{align*}
    (a, b] \cap (c, d] =
    \begin{cases}
        \emptyset          & \text{\((a, b] \cap (c, d] = \emptyset\)のとき,} \\
        (a, b]             & \text{\((a, b] \subseteq (c, d]\)のとき,} \\
        (c, d]             & \text{\((a, b] \supseteq (c, d]\)のとき,} \\
        (c, b]             & \text{それ以外のとき}
    \end{cases}
\end{align*}
からわかる.また区分的\(\mathord{\setminus}\)-閉であることは
\begin{align}
    (a, b] \setminus (c, d] =
    \begin{cases}
        (a, b]             & \text{\((a, b] \cap (c, d] = \emptyset\)のとき,} \\
        \emptyset          & \text{\((a, b] \subseteq (c, d]\)のとき,} \\
        (a, c] \cup (d, b] & \text{\((a, b] \supseteq (c, d]\)のとき,} \\
        (a, c] \cup (b, d] & \text{それ以外のとき}
        \eqlabel{setminus-cases}
    \end{cases}
\end{align}
からわかる.
\end{proof}

\end{document}
