\documentclass[../main.tex]{subfiles}
\begin{document}
\mainchapter{解析}

\begin{thmbox}
\begin{theorem}[(Raabe{\protect\footnotemark}の判定法)]
正項級数\(\sum_{n = 1}^\infty a_n\)について,
\(\lim_{n\to \infty} \inlinefrac{a_{n + 1}}{a_n} = 1\)(なのでratio testでは判別できない)かつ
\(b_n := n(1 - \inlinefrac{a_{n + 1}}{a_n})\)の極限\(\beta := \lim_{n \to \infty} b_n\)が存在するとき,
\(\sum_{n = 1}^\infty a_n\)は\(\beta > 1\)ならば収束し,\(\beta < 1\)ならば発散する.
\theoremlabel{raabe}
\end{theorem}
\end{thmbox}

\footnotetext{\pronunciation{Raabe}{de}{ˈʁaːbə}{ラーベ}}

証明略.


\begin{thmbox}
\begin{theorem}[(Abel{\protect\footnotemark}の連続性定理)]
冪級数\(\sum_{n = 0}^\infty a_n x^n\)が収束半径\(1\)で関数\(f\colon (-1, 1) \to \mathbb{R}\)に収束し,\(\sum_{n = 0}^\infty a_n\)が絶対収束するならば
\begin{align}
  \sum_{n = 0}^\infty a^n = \lim_{x \to 1 - 0}f(x)
\end{align}
が成り立つ.
\theoremlabel{abel}
\end{theorem}
\end{thmbox}

\footnotetext{\pronunciation{Abel}{no}{ˈ\`{ɑ}ːbəl}{アーベル}}

\begin{proof}
各\(n \in \mathbb{N}\)について\(s_n = \sum_{k = 0}^n a_k\)とし,\(s = \lim_{n \to \infty} s_n = \sum_{k = 0}^\infty a_k\)とする.
Abelの変形により
\begin{align*}
    \sum_{k = 0}^n a_k x^k
    &= \sum_{k = 0}^{0} a_k
    + \left(\sum_{k = 0}^1 a_k - \sum_{k = 0}^{0} a_k \right) x
    + \cdots +
    \left(\sum_{k = 0}^n a_k - \sum_{k = 0}^{n - 1} a_k \right) x^n \\
    &=
    \left(\sum_{k = 0}^{0} a_k \right) (1 - x)
    +
    \left(\sum_{k = 0}^{1} a_k \right) (x - x^2)
    + \cdots \\
    &\mathbin{\hphantom{=}} \cdots +
    \left(\sum_{k = 0}^{n - 1} a_k \right) (x^{n - 1} - x^n)
    + \left(\sum_{k = 0}^n a_k \right) x^n \\
    &=
    (1 - x) \sum_{j = 0}^{n - 1} s_j x^j
    + s_n x^n
\end{align*}
となるから
\begin{align}
  \sum_{k = 0}^n a_k x^k - s
  &= (1 - x) \sum_{j = 0}^{n - 1} s_j x^j + s_n x^n - s (1 - x^n) - s x^n \notag \\
  &= (1 - x) \sum_{j = 0}^{n - 1} s_j x^j - (1 - x)\frac{1 - x^n}{1 - x}s \notag \\
  &= (1 - x) \sum_{j = 0}^{n - 1} (s_j - s) x^j \eqlabel{abel}
\end{align}
が成り立つ.

\(x \in (-1, 1)\)において\(\sum_{k = 0}^n a_k x^k \)が\(f\)に収束するので,任意の\(\varepsilon > 0\)に対して,ある\(N \in \mathbb{N}\)が存在して,\(n \geq N\)ならば任意の\(x \in (-1, 1)\)で
\begin{align}
  \lvert f(x)  - \sum_{k = 0}^n a_k x^k\rvert \leq \frac{\varepsilon}{2} \eqlabel{abel2}
\end{align}
が成り立つ.
\eqref{abel},\eqref{abel2}より
\begin{align*}
    \lvert f(x) - s \rvert
    &= \lvert f - \sum_{k = 0}^N a_k x^k + \sum_{k = 0}^N a_k x^k - s \rvert \\
    &= \lvert f - \sum_{k = 0}^N a_k x^k \rvert + \lvert \sum_{k = 0}^N a_k x^k - s \rvert \\
    &\leq \frac{\varepsilon}{2} + (1 - x) \sum_{j = 0}^{N - 1}\lvert s_j - s\rvert x^j \\
\end{align*}
となるから,\(x\)を十分に\(1\)に近くとる,具体的には
\begin{align*}
  \delta = 1 - \frac{\varepsilon}{2\sum_{k = 0}^{N - 1} \lvert s_j  - s \rvert}
\end{align*}
として任意の\(x \in (\delta, 1)\)をとれば\(\lvert f(x) - s \rvert\leq \varepsilon\)となる.
よって
\begin{align*}
  \lim_{x \to 1 - 0}f(x) = s = \sum_{n = 0}^\infty a_k
\end{align*}
が成り立つ.
\end{proof}

この定理は特定の条件下で
\begin{align*}
\lim_{x \to 1 - 0} \lim_{n \to \infty}\sum_{k = 0}^n a_n x^n = \lim_{n \to \infty} \lim_{x \to 1 - 0}\sum_{k = 0}^n a_n x^n
\end{align*}
が成り立つということを述べている.
\end{document}
