\documentclass[../main.tex]{subfiles}
\begin{document}
\mainchapter{関数解析}

\begin{thmbox}
\begin{definition}[(代数)]
\(V\)を\(\symbb{K}\)上のベクトル空間とする.
積または乗法と呼ばれる\(V\)上の二項演算\(\mathord{\ast}\colon (x, y) \mapsto x y\)が以下の性質をみたすとき,
組\((V, \mathord{\ast})\)は\(\symbb{K}\)上の\keyword{代数}(algebra)であるという:
\begin{conditions}
    \item\label{algebra-mult-assoc} 任意の\(x, y, z \in V\)について\(x (yz) = (xy) z\).
    \item 任意の\(x, y, z \in V\)について\((x + y) z = xz + yz, x (y + z) = xz + xz\).
    \item\label{algebra-mult-scalar-assoc} 任意の\(\alpha \in \symbb{K}\)と任意の\(x, y \in X\)について
        \(\alpha (xy) = (\alpha x) y = x (\alpha y)\).
\end{conditions}
どういう演算であるかを強調したい場合以外は組\((V, \mathord{\ast})\)を単に\(V\)と書くことが多い.
上の\ref{algebra-mult-assoc}--\ref{algebra-mult-scalar-assoc}に加えて次をみたすとき,
代数\(V\)は可換であるという:
\begin{conditions}[resume]
    \item 任意の\(x, y\in V\)について\(x y  = y x\).
\end{conditions}
また\ref{algebra-mult-assoc}--\ref{algebra-mult-scalar-assoc}に加えて次をみたすとき,
代数\(V\)は単位元をもつ(with unit)または単位的(unital)であるという:
\begin{conditions}[resume]
    \item ある元\(\symbf{1} \in V \setminus \{\symbf{0}\}\)が存在して,
        任意の\(x \in V\)に対して\(\symbf{1} x  = x \symbf{1} = x\)が成り立つ.
        これを乗法単位元と呼ぶ.
\end{conditions}
\end{definition}
\end{thmbox}

\noindent 混乱のおそれがない場合は\(\symbf{0}, \symbf{1}\)をそれぞれ単に\(0, 1\)と書く.

\begin{thmbox}
\begin{proposition}[(コンパクトHausdorff空間上の連続関数全体の集合から作られる代数)]
\(X\)をコンパクトHausdorff空間とし,
\(X\)から\(\symbb{K}\)への連続関数全体の集合を\(C(X, \symbb{K})\)または単に\(C(X)\)と書く.
\(C(X)\)における加法,スカラー倍,ゼロベクトル(加法単位元)をそれぞれ
\begin{gather*}
    (f + g)(x) := f(x) + g(x),\\
    (\alpha f)(x) := \alpha f(x), \\
    \symbf{0} \colon X \to \mathbb{K}, \text{\(\symbf{0} (x) = 0\)  for all \(x \in X
\)}
\end{gather*}
で定義するとき,\(C(X)\)は\(\symbb{K}\)上のベクトル空間となる.さらに乗法と乗法単位元を
\begin{gather*}
    (fg)(x) := f(x)g(x), \\
    \symbf{1} \colon X \to \symbb{K},\text{\(\symbf{1}(x) = 1\) for all \(x \in X\)}
\end{gather*}
と定義するとき,\(C(X)\)は\(\symbb{K}\)上の代数となる.
\propositionlabel{compact-hausdorff-continuous-function}
\end{proposition}
\end{thmbox}

\begin{proof} \(C(X)\)が加法,スカラー倍,乗法について閉じていることのみを示す.

\paragraph{加法について閉じていること} 任意の\(f, g \in C(X)\)について,\(f + g \in C(X)\)を示す.\(f\)が連続なので,任意の\(x_0 \in X\)について,任意の\(\varepsilon > 0\)をとったとき,\(x_0\)の近傍\(V_1\)が存在して,\(x \in V_1\)ならば
\begin{align*}
    |f(x) - f(x_0)| < \frac{\varepsilon}{2}
\end{align*}
が成り立つ.\(g\)についても同様に,\(x_0\)の近傍\(V_2\)が存在して,\(x \in V_2\)ならば
\begin{align*}
    |g(x) - g(x_0)| < \frac{\varepsilon}{2}
\end{align*}
が成り立つ.よって\(x \in V_1 \cap V_2\)ならば
\begin{align*}
  |(f + g)(x) - (f + g)(x_0)|
  &= |f(x) + g(x) - (f(x_0) + g(x_0))| \\
  &= |f(x) - f(x_0) + g(x) - g(x_0)| \\
  &\leq |f(x) - f(x_0) | + | g(x) - g(x_0)| \\
  &< \frac{\varepsilon}{2} + \frac{\varepsilon}{2} = \varepsilon
\end{align*}
となり,\(f + g\)も\(x_0 \in X\)で連続である.したがって\(f + g \in C(X)\)である.

\paragraph{スカラー倍について閉じていること} \(\alpha\)を任意の実数とする.任意の\(f \ \in C(X)\)について\(\alpha f \in C(X)\)となることを示す.\(f\)が任意の\(x_0 \in X\)で連続なので,任意の\(\varepsilon > 0\)に対して,\(x_0\)の近傍\(V\)が存在して,\(x \in V\)ならば
\begin{align*}
  |f(x) - f(x_0)| < \frac{\varepsilon}{|\alpha| + 1}
\end{align*}
が成り立つ.したがって
\begin{align*}
  |(\alpha f)(x) - (\alpha f)(x_0)|
  &= |\alpha f(x) - \alpha f(x_0)| \\
  &\leq |\alpha | |f(x) - f(x_0)| \\
  &< |\alpha | \frac{\varepsilon}{|\alpha| + 1} \\
  & \leq \varepsilon
\end{align*}
となり,\(\alpha f \in C(X)\)である.

\paragraph{乗法について閉じていること} 任意の\(f, g \in X\)について\(fg \in C(X)\)を示す.任意の\(x_0 \in X\)をとる.\(f\)が\(x_0\)で連続なので,\(x_0\)の近傍\(V_1\)が存在して,\(x \in V_1\)ならば
\begin{gather*}
  |f(x) - f(x_0)| \leq 1 \\
  |f(x) | \leq | f(x_0)| + 1
\end{gather*}
が成り立つ.同じく連続であることから,任意の\(\varepsilon > 0\)に対し,\(x_0\)の近傍\(V_2\)が存在して,\(x \in V_2\)ならば
\begin{align*}
  |f(x) - f(x_0)| < \frac{\varepsilon}{2(|g(x_0) + 1|)}
\end{align*}
が成り立つ.同様に\(g\)が連続なことから,\(x_0\)の近傍\(V_3\)が存在して,\(x \in V_3\)ならば
\begin{align*}
  |g(x) - g(x_0)| < \frac{\varepsilon}{2(|f(x_0)| + 1)}
\end{align*}
となる.したがって任意の\(x \in V_1 \cap V_2 \cap V_3\)について
\begin{align*}
  |(fg)(x) - (fg)(x_0)|
  &= |f(x)g(x) - f(x_0)g(x_0)| \\
  &= |f(x)g(x) - f(x)g(x_0) + f(x)g(x_0) - f(x_0)g(x_0)| \\
  &= |f(x)(g(x) - g(x_0)) + (f(x) - f(x_0))g(x_0)| \\
  &\leq |f(x)| |g(x) - g(x_0)| + |f(x) - f(x_0)| | g(x_0)| \\
  &< (|f(x_0)| + 1) \frac{\varepsilon}{2(|f(x_0)| + 1)} + \frac{\varepsilon}{2(|g(x_0)| + 1)} |g(x_0)| \\
  &\leq \frac{\varepsilon}{2} + \frac{\varepsilon}{2} = \varepsilon
\end{align*}
が成り立つ.よって\(fg \in C(X)\)である.
\end{proof}

\begin{thmbox}
\begin{definition}[(ノルム)]
\(\symbb{K}\)上のベクトル空間\(V\)において,
\(V\)から\(\NonNegReal\)への写像\(\lVert\placeholder\rVert\colon x \mapsto \lVert x \rVert\)が
以下の性質をみたすとき,\(\lVert \placeholder \rVert\)を\(V\)上のノルムと呼ぶ:
\begin{conditions}
    \item 任意の\(x \in V\)について\(\|x\| = 0\;\Leftrightarrow\;x = 0\).
    \item 任意の\(x \in V\)と\(\alpha \in \symbb{K}\)について\(\lVert \alpha x\rVert = |\alpha|\lVert x \rVert\).
    \item (三角不等式,劣加法性)任意の\(x, y \in V\)について\(\lVert x + y \rVert \leq \lVert x\rVert + \lVert y \rVert\).
\end{conditions}
\end{definition}
\end{thmbox}

\noindent 代数のノルムは以下のように定義される.

\begin{thmbox}
\begin{definition}[(代数のノルム,ノルム付き代数,Banach代数)]
代数\(V\)上の写像\(\lVert\placeholder\rVert\)が,ベクトル空間\(V\)のノルムであり,さらに以下の\ref{banach-algebra-submultiplicativity}, \ref{banach-algebra-norm-1}を満たすとき,\(\lVert\placeholder\rVert\)を代数\(V\)のノルムという:
\begin{conditions}
    \item\label{banach-algebra-submultiplicativity}(劣乗法性)任意の\(x, y \in X\)について\(\lVert xy \rVert \leq \lVert x \rVert \lVert y \rVert\).
    \item\label{banach-algebra-norm-1} \(\lVert \symbf{1} \rVert = 1\).
\end{conditions}
代数\(V\)とそのノルムの組\((V, \lVert\placeholder\rVert)\)を\keyword{ノルム付き代数}(normed algebra)という.
どういうノルムであるかを強調したい場合以外は組\((V, \lVert \placeholder \rVert)\)を単に\(V\)と書く事が多い.
ノルム付き代数\((V, \lVert \placeholder \rVert)\)がノルム\(\lVert \placeholder \rVert\)について完備であるとき,\((V, \lVert \placeholder \rVert)\)を\keyword{Banach代数}(Banach Algebra)と呼ぶ.
\end{definition}
\end{thmbox}

\begin{thmbox}
\begin{proposition}
代数\(C(X)\)に一様ノルム(ここでは最大値ノルムに一致)を入れる,すなわち任意の\(f \in C(X)\)について
\begin{align*}
    \lVert f \rVert_\infty = \sup_{x \in X} |f(x)|
\end{align*}
と定義する.このとき\((C(X), \lVert \placeholder \rVert_\infty)\)はBanach代数となる.
\end{proposition}
\end{thmbox}

\noindent 以下では\(C(X)\)のノルムとしては一様ノルムのみを考え,\(\lVert \placeholder \rVert_\infty\)を単に\(\lVert \placeholder \rVert\)と書く.

\begin{proof} 一様ノルムが代数のノルムになっていることの証明は省略し,完備であることのみを示す.
\(C(X)\)上の任意のCauchy列\((f_n)_{n \in \PositiveInteger}\)をとる.このとき,任意の\(\varepsilon > 0\)に対して,ある\(N_1 \in \PositiveInteger\)が存在して,\(m, n \geq N_0\)ならば
\begin{align*}
  \lVert f_m - f_n \rVert < \varepsilon,
\end{align*}
すなわち,任意の\(x \in X\)で
\begin{align*}
  |f_m(x) - f_n(x)| < \varepsilon
\end{align*}
が成り立つ.\(\symbb{K}\)の完備性により\((f_n(x))_{n \in \PositiveInteger}\)はある\(\alpha_x \in \symbb{K}\)に収束する.すなわち\(f(x) := \alpha_x\)とすると,\(f\colon X \to \symbb{K}\)であり,ある\(N \in \PositiveInteger\)が存在して
\begin{align}
  |f_N(x) - f(x)| < \frac{\varepsilon}{3}
  \label{complete}
\end{align}
を満たす.\(x \in X\)は任意だったので\((f_n)_{n \in \PositiveInteger}\)は\(f\)に一様収束する.この\(f\)が連続であることを示す.\(f_N\)は連続なので,\(x\)の近傍\(V_x\)が存在し,\(y \in V_x\)ならば
\begin{align}
  |f_N(y) - f_N(x)| < \frac{\varepsilon}{3}
  \label{continuous-fn}
\end{align}
が成り立つ.\((\ref{complete})\)と\((\ref{continuous-fn})\)より
\begin{align*}
  |f(y) - f(x)|
  &= |f(y) - f_N(y) + f_N(y) - f_N(x) + f_N(x) - f(x)| \\
  &\leq |f(y) - f_N(y)| + |f_N(y) - f_N(x)| + |f_N(x) - f(x)| \\
  &< \frac{\varepsilon}{3} + \frac{\varepsilon}{3} + \frac{\varepsilon}{3} = \varepsilon.
\end{align*}
よって\(f\)は\(x\)で連続であり,\(x\)は任意だったので,\(f \in C(X)\)が示せた.
\end{proof}


\end{document}
