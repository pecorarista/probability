\documentclass[../main.tex]{subfiles}
\begin{document}
\mainchapter{多様体}

\(X\)を位相空間とする.
\([a, b]\)を\(\symbb{R}\)の区間とする.
連続写像\(\gamma\colon [a, b] \to X\)を\keyword{曲線}(curve)という.
\(\gamma(a) = p\), \(\gamma(b) = q\)であるとき,\(\gamma\)を\(p\)から\(q\)への曲線という.
% 区間\(I\)の\keyword{分割}(partition)とは\(a = t_0 < t_1 < \cdots < t_n = b\)を満たすような点の集合\(\{t_0, \ldots, t_n\}\)のことをいう.
\(\tuple{M, g}\)をRiemann多様体とし,\(\gamma\)を区分的に滑らかな曲線とする.
このとき\(\gamma\)の\keyword{長さ}(length)を
\begin{align*}
    L(\gamma) = \int_{a}^{b} \sqrt{g\left(\frac{d\gamma}{dt}(t), \frac{d\gamma}{dt}(t)\right)} dt
\end{align*}
と定義する.\(M\)が連結であるとき
\begin{align*}
    d_g(p, q) = \inf \{L(\gamma) \mid \text{\(\gamma \colon [a, b] \to M\)は\(p\)から\(q\)への曲線}\}
\end{align*}
で定義とすると\(d_g\)は距離の公理をみたし,したがって\(\tuple{M, d_g}\)は距離空間となる.


確認したいことは以下の\(2\)つである.
\begin{conditions}
    \item\label{great-circle-distance-is-metric} \eqref{great-circle-distance}が距離の公理をみたすこと.
    \item \(p\)から\(q\)へ球面に沿って移動する経路で長さが最小となるのは\(p, q\)で定まる大円の劣弧であること.
\end{conditions}
\ref{great-circle-distance-is-metric}を直接確認してもよいが,一般に図形(正確にはRiemann多様体)に沿った曲線で最も短いものが距離となることを示せば十分である.


北極点を\(\symup{N}\),南極点\(\symup{S}\)と表す.
任意の点\(p \in S^2 \setminus \{\symup{N}, \symup{S}\}\)はパラメーター表示\(X \colon \symbb{R}^2 \to S^2\)と包含写像\(\iota \colon S^2 \to \symbb{R}^3\)を用いて
\begin{align*}
    (\iota \circ X) \colon
    \begin{pmatrix}
        \theta \\
        \varphi
    \end{pmatrix}
    \mapsto
    \begin{pmatrix}
        \sin \theta \cos \varphi \\
        \sin \theta \sin \varphi \\
        \cos \theta
    \end{pmatrix}
\end{align*}

\begin{align*}
    \iota \circ \varphi^{-1}\colon
    \begin{pmatrix}
        \theta \\
        \varphi
    \end{pmatrix}
    \mapsto
    \begin{pmatrix}
        \sin \theta \cos \varphi \\
        \sin \theta \sin \varphi \\
        \cos \theta
    \end{pmatrix}
\end{align*}
aa
\begin{align*}
    \frac{\partial X}{\partial \theta}
    =
    \begin{pmatrix}
        \cos \theta \cos \varphi \\
        \cos \theta \sin \varphi \\
        - \sin \theta
    \end{pmatrix},
    \quad
    \frac{\partial \gamma}{\partial \varphi}
    =
    \begin{pmatrix}
        - \sin \theta \sin \varphi \\
        \sin \theta \cos \varphi \\
        0
    \end{pmatrix},
\end{align*}
\begin{gather*}
    u \colon [0, 1] \to [0, \pi] \times [0, 2 \pi) , \quad
    u \colon t \mapsto
    \begin{pmatrix}
        \theta(t) \\
        \varphi(t)
    \end{pmatrix}, \\
    c \colon
    [0, \pi] \times [0, 2 \pi) \to S, \quad
    c \colon
    \begin{pmatrix}
        \theta \\
        \varphi
    \end{pmatrix}
    \mapsto
    \begin{pmatrix}
        \sin \theta \cos \varphi \\
        \sin \theta \sin \varphi \\
        \cos \theta
    \end{pmatrix}
\end{gather*}
曲線\(\gamma = c \circ u\)
\begin{align*}
    \gamma(t) =
    \begin{pmatrix}
        \sin \theta(t) \cos \varphi(t) \\
        \sin \theta(t) \sin \varphi(t) \\
        \cos \theta(t)
    \end{pmatrix}
\end{align*}
と定義する.
ここで\(\theta(t), \varphi(t)\)はそれぞれ時刻\(t\)における\(\gamma(t)\)の余緯度\((0 \leq \theta(t) \leq \pi)\)と経度\((0 \leq \varphi(t) < 2\pi)\)である.
\(\gamma\)上の点\(p\)において
局所座標\((\theta, \varphi)\)
である.したがって被積分関数の\(2\)乗は
\begin{align*}
    \lVert \dot{\gamma}(t) \rVert^2
        &= \dot{\theta}^2 \cos^2 \theta \cos^2 \varphi
        - 2 \dot{\theta} \dot{\varphi} \cos \theta \cos \varphi \sin \theta  \sin \varphi
        + \dot{\varphi}^2 \sin^2 \theta \sin^2 \varphi \\
        &\mathrel{\hphantom{=}}{\relax} \mathbin{+} \dot{\theta}^2 \cos^2 \theta \sin^2 \varphi
             + 2 \dot{\theta} \dot{\varphi} \cos \theta \sin \varphi \sin \theta  \cos \varphi
            + \dot{\varphi}^2 \sin^2 \theta \cos^2 \varphi \\
        &\mathrel{\hphantom{=}}{\relax} \mathbin{+} \dot{\theta}^2 \sin^2 \theta \\
        &= \dot{\theta}^2 \cos^2 \theta (\cos^2\varphi + \sin^2 \varphi) + \dot{\varphi}^2 \sin^2 \theta (\sin^2 \varphi + \cos^2 \varphi) + \dot{\theta}^2 \sin^2 \theta \\
        &= \dot{\theta}^2  + \dot{\varphi}^2 \sin^2 \theta
\end{align*}
となる.\(F \colon [0, 1] \times \symbb{R}^2 \times \symbb{R}^2 \to \symbb{R}\)を
\begin{align*}
    F\left(t,
        \begin{pmatrix} \theta(t) \\ \varphi(t) \end{pmatrix},
        \begin{pmatrix} \dot{\theta}(t) \\ \dot{\varphi}(t) \end{pmatrix}\right) =
        \sqrt{(\dot{\theta}(t))^2 + (\dot{\varphi}(t))^2 \sin^2 \theta(t)}
\end{align*}
と定義する.この問題のEuler--Lagrange方程式は
\begin{gather*}
    \begin{pmatrix}
        \displaystyle
        \frac{\partial}{\partial \theta} F
        &
        \displaystyle
        \frac{\partial}{\partial \varphi} F
    \end{pmatrix}
    -
    \frac{d}{dt}
    \begin{pmatrix}
        \displaystyle
        \frac{\partial}{\partial \dot{\theta}} F
        &
        \displaystyle
        \frac{\partial}{\partial \dot{\varphi}} F
    \end{pmatrix}
    = 0
    \\
    \begin{pmatrix}
        \displaystyle
        \frac{\dot{\varphi}^2 \sin \theta \cos \theta}{\sqrt{\dot{\theta}^2 + \dot{\varphi}^2 \sin^2 \theta}}
        &
        0
    \end{pmatrix}
    =
    \begin{pmatrix}
        \displaystyle
        \frac{d}{dt}
        \frac{\dot{\theta}}{\sqrt{\dot{\theta}^2 + \dot{\varphi}^2 \sin^2 \theta}}
        &
        \displaystyle
        \frac{d}{dt}
        \frac{\dot{\varphi} \sin^2\theta}{\sqrt{\dot{\theta}^2 + \dot{\varphi}^2 \sin^2 \theta}}
    \end{pmatrix}
\end{gather*}
で与えられる.第\(2\)成分について
\begin{align*}
    \int_0^t 0 dt =
        \frac{\dot{\varphi}(t) \sin^2\theta(t)}{\sqrt{(\dot{\theta}(t))^2 + (\dot{\varphi}(t))^2 \sin^2 \theta(t)}}
        -
        \frac{\dot{\varphi}(0) \sin^2\theta(0)}{\sqrt{(\dot{\theta}(t))^2 + (\dot{\varphi}(t))^2 \sin^2 \theta(0)}}
\end{align*}
\(t = 0\)を北極にとれば
\begin{align*}
    \frac{\dot{\varphi}(t) \sin^2\theta(t)}{\sqrt{(\dot{\theta}(t))^2 + (\dot{\varphi}(t))^2 \sin^2 \theta(t)}}
    = 0
\end{align*}
戻らないので\(\theta(t) > 0\)としてよい.
\(\dot{\varphi}(t) = 0\)となる.

\end{document}
